\section{Introduction}\label{sec:intro}

This workbook has two primary goals.
First, it aims to provide sufficient background to understand state-of-the-art papers on cryptographic signatures, with a focus on discrete-logarithm-based multi-party signatures, including their security proofs.
Second, it seeks to develop the skills needed to formalize security notions for cryptographic primitives.
This skill is crucial for selecting appropriate primitives when proposing and reviewing cryptographic protocols, and for defining precisely what a protocol aims to achieve.

The concepts introduced in this section may at first seem abstract, but this level of abstraction is important: it provides the rigorous mathematical framework on which modern provable cryptography is built.
Throughout this workbook we will study algorithms, analyse their running time and success probability, both for explicit algorithms and for hypothetical adversarial algorithms.
In later sections we will sometimes study these algorithms within idealised computational models.

This workbook primarily contains definitions, propositions, and lemmas, with limited intuitive explanations.
Good cryptography papers should include intuition, but it's never complete: intuition is inherently subjective and what makes sense to one reader may not resonate with another.
What you need to do is develop your own explanations while reading the mathematics very precisely.
The goal of this workbook is to build intuition step-by-step through the exercises, which provide hands-on experience with the theoretical concepts.

Note that there is rarely a single standard definition for a concept in cryptography.
Sometimes we encounter equivalent definitions in the literature, while other times we find definitions that differ in subtle but important ways.
This is why contemporary papers in cryptography include a preliminaries section that rigorously defines the concepts they use.

% REQUIREMENTS:
% basic sets
% modular arithmetic
% proof by contraposition
% basic probability and union bound
% good to have seen before:
% negligible functions
% \ppt algorithms

\subsection{Algorithms}

\say{In mathematics, \emph{algorithm} is commonly understood to be an exact prescription, defining a computational process, leading from various initial data to the desired result.} (A. Markov, 1954)

\begin{definition}
An algorithm $\adv$ is
\begin{itemize}
\item \emph{probabilistic} if it gets random bits as input (in addition to its regular input).
We write $\adv(x; \rho)$ to denote running algorithm $\adv$ on input $x$ with randomness $\rho$.
When the randomness is clear from context or unnecessary, we simply write $\adv(x)$.
\item \emph{polynomial-time} if there exists a polynomial $p$ such that for every $x \in \{0,1\}^\lambda$, $\adv(x)$ terminates in time $p(\lambda)$.
\item \emph{probabilistic polynomial-time (\ppt)} if it is probabilistic and polynomial-time.
\end{itemize}
\end{definition}

Note that the runtime of an algorithm is characterized relative to the size of the input.
All adversaries considered in cryptography are algorithms.

\subsection{Negligible Functions}



\begin{definition}
A function $f: \ZZ \rightarrow \mathbb{R}$ is \emph{negligible} if for every constant $c > 0$ there exists $N \in \NN$ such that for all integers $\secpar > N$, $f(\secpar) < \secpar^{-c}$.
% Or equivalently.
% - We denote the set of \emph{polynomially-bounded} functions in the security parameter $\secpar$ by $\poly = \{ f : \exists a \in \NN,\ f(\secpar) \in O(\lambda^a) \}$,. functions in the security parameter $\secpar$ by $\negl = \{ f : f(\secpar)^{-1} \not\in \poly \}$
% - for all $a \in \NN$, there exists $N \in \NN$ such that for all $\secpar \ge N$, $f(\secpar) \le \frac{1}{\lambda^a}$.
\end{definition}

\begin{example}
  $2^{-\secpar}$ is negligible. $1/\secpar$ is not negligible.
\end{example}

If a function $f$ is negligible, we write $f(\secpar) = \negl$.


\begin{lemma}[Properties of negligible functions]
  \label{lem:negl}
  \hfill
  \begin{enumerate}
  \item Let $f$ be a negligible function and $p$ be a polynomial. Then $f(\secpar) \cdot p(\secpar)$ is negligible.
  \item Let $f_1, f_2$ be negligible functions. Then $f_1 + f_2$ is negligible.
  \item Let $f$ be a negligible function $f(\secpar) \ge 0$ and $k > 0$ be a constant. Then $f + k$ is not negligible.
  \item Let $f$ be a non-negligible function and $g$ be a negligible function. Then $f - g$ is not negligible.
  \end{enumerate}
\end{lemma}

The proof of property (1) is left as an exercise (see \autoref{ex:negl-property-proof}).
The proofs of properties (2)-(4) can be found in Appendix~\ref{sec:appendix-negl-proofs}.

\subsection{Games \& Asymptotic Security}

In this section we will introduce the concept of security games (sometimes called ``experiments'') and asymptotic security through a series of toy examples.

The $\game{\Game~\algo{Guess}}{}$ defined in \autoref{fig:guessing-game} takes an algorithm $\adv$ and the security parameter $\secpar$ as input and outputs $1$ or $0$.
The game first sets $n = 2^\lambda$ and samples $x$ uniformly at random from $\ZZ_n$.
Then it runs algorithm $\adv$ without input, sets its output to $x'$ and returns $1$ if $x = x'$.

\begin{figure}[tbhp]
  \begin{center}
    \begin{tcolorbox}[width=3cm]
      \begin{pchstack}[center]
          \procedure[headlinesep=1pt]{$\game{\Game~\algo{Guess}}{}$}{%
            n \defeq 2^\lambda\\
            x \sample \ZZ_n \\
            x' \gets \adv() \\
            \pcreturn x = x'
          }
      \end{pchstack}
    \end{tcolorbox}
  \end{center}
  \caption{Guessing game\label{fig:guessing-game}}
\end{figure}

\begin{definition}[Guessing game advantage]
  For $\game{\Game~\algo{Guess}}{}$ as defined in \autoref{fig:guessing-game} we define the advantage of algorithm $\adv$ as
 \[
  \advantage{Guess}{\adv} \defeq \pr{\game{\algo{Guess}}{} = 1}.
 \]
\end{definition}

\begin{proposition}\label{prop:guessing-game}
  For any algorithm $\adv$, $\advantage{Guess}{\adv} = \negl$.
\end{proposition}

\begin{proof}
Let $x'$ be the value returned by $\adv$.
The probability that $x$ sampled independently and uniformly at random from $[1, 2^\lambda]$ is equal to $x'$ is $2^{-\lambda} = \negl$.
\end{proof}

While the guessing game above provides a clean security definition, formalizing the security of concrete hash functions like SHA256 requires a fundamentally different approach.
To illustrate this, consider the $\game{\Game~\algo{SHAPreimgZ}}{}$ defined in \autoref{fig:sha-preimg-0}, which outputs $1$ if the algorithm $\adv$ outputs a preimage of 0 under the hash function $\algo{SHA256}: \{0, 1\}^* \rightarrow \{0, 1\}^{256}$.
Note that $\adv$ gets $1^{\secpar}$ (a string of 1s of length $\secpar$) as input in order to characterise the running time of $\adv$ as a function of the security parameter $\secpar$.

\begin{figure}[tbhp]
  \begin{center}
    \begin{tcolorbox}[width=5cm]
      \begin{pchstack}[center]
          \procedure[headlinesep=1pt]{$\game{\Game~\algo{SHAPreimgZ}}{}$}{%
            y \defeq 0 \\
            x \gets \adv(\secparam) \\
            \pcreturn \algo{SHA256}(x) = y
          }
      \end{pchstack}
    \end{tcolorbox}
  \end{center}
  \caption{Game for finding SHA256 preimage of zero \label{fig:sha-preimg-0}}
\end{figure}

\begin{proposition}\label{prop:sha-preimg-z}
 For $\game{\Game~\algo{SHAPreimgZ}}{}$ as defined in \autoref{fig:sha-preimg-0}, let the advantage of $\adv$ be defined as
 \[
  \advantage{SHAPreIZ}{\adv} \defeq \pr{\game{\algo{SHAPreimgZ}}{} = 1}.
 \]
 If there exists $x$ such that $\algo{SHA256}(x) = 0$, there exists a \ppt algorithm $\adv$ such that
  \[
  \advantage{SHAPreIZ}{\adv} = 1.
  \]
\end{proposition}

\begin{proof}
  Let $\adv$ be an algorithm
  \[
   \pcreturn x
  \]
  where $\algo{SHA256}(x) = 0$.
\end{proof}

Next, we consider a variant of preimage resistance for SHA256 where the target value is chosen uniformly at random, preventing the existence of trivial algorithms like the one in Proposition~\ref{prop:sha-preimg-z}.
In $\game{\Game~\algo{SHAPreimg}}{}$ defined in \autoref{fig:sha-preimg} the value $x$ is a uniformly random bitstring of length $256$.

\begin{figure}[tbhp]
  \begin{center}
    \begin{tcolorbox}[width=5cm]
      \begin{pchstack}[center]
          \procedure[headlinesep=1pt]{$\game{\Game~\algo{SHAPreimg}}{}$}{%
            x \sample \{0, 1\}^{256}\\
            y \gets \algo{SHA256}(x) \\
            x' \gets \adv(y) \\
            \pcreturn \algo{SHA256}(x') = y
          }
      \end{pchstack}
    \end{tcolorbox}
  \end{center}
  \caption{Game for finding the SHA256 preimage of a random value \label{fig:sha-preimg}}
\end{figure}

\begin{proposition}
 For $\game{\Game~\algo{SHAPreimg}}{}$ as defined in \autoref{fig:sha-preimg}, let the advantage of $\adv$ be defined as
 \[
  \advantage{SHAPreI}{\adv} \defeq \pr{\game{\algo{SHAPreimg}}{} = 1}.
 \]
 There exists a \ppt algorithm $\adv$ such that
 \[
 \advantage{SHAPreI}{\adv} = 1.
 \]
\end{proposition}

\begin{proof}
  Let $\adv$ be an algorithm that queries $\algo{SHA256}$ until it finds a solution as shown in \autoref{fig:sha-preimg-adv}.
  \begin{figure}[tbhp]
  \begin{center}
    \begin{tcolorbox}[width=5cm]
      \begin{pchstack}[center]
          \procedure[headlinesep=1pt]{$\adv(y)$}{%
          x \defeq 0 \\
          \pcwhile \algo{SHA256}(x) \neq y \pcdo \\
            \t x \defeq x + 1 \\
          \pcendwhile \\
          \pcreturn x
          }
      \end{pchstack}
    \end{tcolorbox}
  \end{center}
  \caption{Algorithm for finding the SHA256 preimage of a given value \label{fig:sha-preimg-adv}}
  \end{figure}
  $\adv$ will return the correct solution with probability $1$.
  The running time of $\adv$ is bounded by a constant independent of the security parameter $\secpar$ and, therefore, $\adv$ is \ppt
\end{proof}

% There is also an algorithm which just has all preimages of SHA256 hardcoded and stored in order. Then use binary search.

We observe that if the output space of the hash function is independent of the security parameter, then the hash function is not preimage-resistant.

\subsection{Hash Functions}

\begin{definition}[Hash function]
  A \emph{hash function} $\algo{H}$ is a pair of polynomial-time algorithms $(\algo{Gen}, \algo{Eval})$ where:
  \begin{itemize}
  \item $\algo{Gen}(\secparam) \rightarrow \kappa$ is a probabilistic algorithm that takes the security parameter as input and outputs a hashing key $\kappa$\footnote{We assume $\kappa$ includes $\secparam$ so that $\algo{Eval}$ can run in time polynomial in $\secpar$}.
  \item $\algo{Eval}(\kappa, x) \rightarrow y$ is a deterministic algorithm that takes a hashing key $\kappa$ and an input $x \in \{0,1\}^*$ as input and outputs a hash value $y \in S$. When not specified otherwise, we assume $S = \{0,1\}^{\lambda}$.\footnote{For notational simplicity, $\algo{H.Eval}(\kappa, x)$ is often written as $\algo{H}(x)$, leaving the hashing key implicit.}
  \end{itemize}
\end{definition}

Note that the hashing key $\kappa$ is not secret because it is required to evaluate the hash function.

\begin{remark}[Encoding of Arbitrary Inputs]
  We extend the notation to arbitrary inputs $x$ (such as group elements, integers, or tuples) and simply write $\algo{H.Eval}(\kappa, x)$, implicitly assuming the existence of a fixed injective encoding function that maps the input $x$ into $\{0,1\}^*$.
  We omit explicit reference to the encoding function in our notation.
\end{remark}

\begin{figure}[tbhp]
  \begin{center}
    \begin{tcolorbox}[width=5cm]
      \begin{pchstack}[center]
        \procedure[headlinesep=1pt]{$\game{\Game~\algo{Preimage}}{\algo{H}}$}{%
          \kappa \gets \algo{H.Gen}(\secpar) \\
          x \sample \{0, 1\}^\lambda\\
          y \gets \algo{H.Eval}(\kappa, x) \\
          x' \gets \adv(\kappa, y) \\
          \pcreturn \algo{H.Eval}(\kappa, x') = y
        }
      \end{pchstack}
    \end{tcolorbox}
  \end{center}
  \caption{Game for finding the preimage of a given value under the hash function \label{fig:break-hash}}
\end{figure}

Similar to the algorithm in \autoref{fig:sha-preimg-adv}, there exists an algorithm that wins $\game{\Game~\algo{Preimage}}{\algo{H}}$ with probability 1 by just evaluating $\algo{H.Eval}$ on fresh inputs until $\algo{H.Eval}(x) = y$.
However, for adequate hash functions this algorithm runs in time exponential in the security parameter.
Hence, the definition of preimage-resistance only considers \ppt adversaries.

\begin{definition}[Preimage-resistance]\label{def:preimage-resistance}
  A hash function $\algo{H}$ is \emph{preimage-resistant} if for any \ppt algorithm $\adv$,
 \[
  \advantage{PreI}{\adv, \algo{H}} \defeq \pr{\game{\algo{Preimage}}{\algo{H}} = 1} = \negl.
 \]
\end{definition}

% The key is necessary to make the security definitions work in a non-uniform model (where we have a different adversary for every security parameter).
% If there was no key, then, given the security parameter, there exists an algorithm where the state transitions encode a sorted map from all $2^\secpar$ possible hash function outputs y to a preimage x.
% Then the algorithm could look up the challenge y in $O(\log(2^\secpar)) = O(\secpar)$ time.

\begin{remark}[Asymptotic Security Framework]
  Definition~\ref{def:preimage-resistance} is an example of a definition in the asymptotic security framework.
  The hash function $\algo{H}$ consists of algorithms that are polynomial-time in $\secpar$, and preimage resistance requires that any \ppt adversary's success probability is negligible in $\secpar$.
  Hence, by increasing $\lambda$ we can make the success probability of any \ppt adversary arbitrarily small.
\end{remark}


\subsection{Exercises}

\begin{exercise}\label{ex:negligible-functions}
  Determine which of the following functions are negligible in $\secpar$:
  \begin{enumerate}
    \item $f(\secpar) = 0$
    \item $f(\secpar) = \frac{1}{2}$
    \item $f(\secpar) = \frac{1}{\secpar^2}$
    \item $f(\secpar) = \secpar^{4096} \cdot 2^{-\secpar}$
    \item $f(\secpar) = \sqrt{2^{-\secpar}}$
  \end{enumerate}
\end{exercise}

\ifsolutions
\begin{mysolution}
  We analyze each function using the definition of negligible functions and Lemma~\ref{lem:negl}:
  \begin{enumerate}
    \item $f(\secpar) = 0$ is \textbf{negligible}.
    For any constant $c > 0$, we have $0 < \secpar^{-c}$ for all $\secpar > 0$.
    
    \item $f(\secpar) = \frac{1}{2}$ is \textbf{not negligible}.
    By Lemma~\ref{lem:negl}(3), since $0$ is negligible and $\frac{1}{2}$ is a positive constant, $0 + \frac{1}{2} = \frac{1}{2}$ is not negligible.
    
    \item $f(\secpar) = \frac{1}{\secpar^2}$ is \textbf{not negligible}.
    For $c = 3$, we have $\frac{1}{\secpar^2} > \secpar^{-3}$ for all $\secpar > 1$, so the definition is not satisfied.
    
    \item $f(\secpar) = \secpar^{4096} \cdot 2^{-\secpar}$ is \textbf{negligible}.
    First, $2^{-\secpar}$ is negligible: for any $c > 0$, we need $2^{-\secpar} < \secpar^{-c}$, which is equivalent to $\secpar^c < 2^{\secpar}$.
    Taking logarithms: $c \ln(\secpar) < \secpar \ln(2)$.
    We have $\lim_{\secpar \to \infty} \frac{c \ln(\secpar)}{\secpar \ln(2)} = \frac{c}{\ln(2)} \cdot \lim_{\secpar \to \infty} \frac{\ln(\secpar)}{\secpar}$ (constant multiple rule).
    Since $\lim_{\secpar \to \infty} \frac{\ln(\secpar)}{\secpar} = 0$ (by L'Hôpital's rule: $\lim_{\secpar \to \infty} \frac{1/\secpar}{1} = 0$), we get $\frac{c}{\ln(2)} \cdot 0 = 0$.
    Therefore, there exists $N$ such that the inequality holds for all $\secpar > N$.
    By Lemma~\ref{lem:negl}(1), the product of a negligible function with a polynomial is negligible.
    
    \item $f(\secpar) = \sqrt{2^{-\secpar}}$ is \textbf{negligible}.
    Since $2^{-\secpar}$ is negligible, for any $c > 0$ there exists $N$ such that for all $\secpar > N$,
    $2^{-\secpar} < \secpar^{-2c}$.
    Taking square roots preserves the inequality, so for all $\secpar > N$,
    $\sqrt{2^{-\secpar}} < \sqrt{\secpar^{-2c}} = \secpar^{-c}$.
    Hence $f(\secpar)$ is negligible.
  \end{enumerate}
\end{mysolution}
\fi

\begin{exercise}\label{ex:guessing-game-equivalence}
  Consider the $\game{\Game~\algo{Guess2}}{}$ which is identical to $\game{\Game~\algo{Guess}}{}$ except that $x$ is sampled \emph{after} $\adv$ is run.
  That is, the game first runs $x' \gets \adv()$, then samples $x \sample \ZZ_n$, and returns $1$ if $x = x'$.
  \begin{enumerate}
    \item Are these two games equivalent? That is, does $\advantage{Guess}{\adv} = \advantage{Guess2}{\adv}$ for all algorithms $\adv$?
    \item Does this change affect the proof of the proposition above?
  \end{enumerate}
  
  This exercise illustrates an important principle in game-based proofs: the order of operations does not necessarily affect game outcomes when the operations are independent.
\end{exercise}

\ifsolutions
\begin{mysolution}
  \begin{enumerate}
    \item Yes, the games are equivalent.
    Since $\adv$ receives no input and $x$ is sampled independently of $\adv$'s execution, the order doesn't matter.
    In both cases, $\adv$ outputs some fixed value $x'$ (determined by its internal randomness), and then we check whether $x = x'$ for a uniformly random $x$.
    The probability is $2^{-\lambda}$ in both cases.
    
    \item No, the proof remains the same.
    The key insight is that $\adv$'s output $x'$ is fixed before we consider the probability over the random choice of $x$.
    Whether $x$ is sampled before or after running $\adv$ doesn't affect this probability since $\adv$ has no information about $x$.
  \end{enumerate}
  
  This illustrates that when operations are independent (here, $\adv$'s execution and the sampling of $x$), their order doesn't affect the outcome, which is a fundamental principle in game-based proofs.
\end{mysolution}
\fi


\begin{exercise}\label{ex:sha-preimage-problem}
  Why is the definition of $\advantage{SHAPreIZ}{\adv}$ in Proposition~\ref{prop:sha-preimg-z} problematic from a cryptographic perspective?
  What fundamental issue does it illustrate about concrete hash functions like SHA256?
\end{exercise}

\ifsolutions
\begin{mysolution}
  The definition is problematic because it doesn't depend on the security parameter $\secpar$ in a meaningful way.
  The adversary $\adv$ that returns a hardcoded preimage of 0 (if one exists) always succeeds with probability 1, regardless of $\secpar$.
  This illustrates that we cannot achieve asymptotic security for concrete, fixed hash functions like SHA256: their output size doesn't grow with the security parameter.
  This is why we define abstract hash function families where the output length grows with $\secpar$.
\end{mysolution}
\fi

\begin{exercise}\label{ex:sha-random-preimage-problem}
  Why is the definition of $\advantage{SHAPreI}{\adv}$ following Proposition~\ref{prop:sha-preimg-z} also problematic?
  Consider the adversary described in the proof.
\end{exercise}

\ifsolutions
\begin{mysolution}
  The adversary's running time is indeed polynomial in $\secpar$: it's a constant independent of $\secpar$, which is $O(1)$ and therefore polynomial.
  However, this constant is potentially $2^{256}$, which is astronomically large.
  The problem is that the output size (256 bits) doesn't grow with $\secpar$, so even exhaustive search becomes ``polynomial time'' in $\secpar$.
  This shows that asymptotic security only makes sense when the problem size grows with the security parameter.
\end{mysolution}
\fi

\begin{exercise}\label{ex:hash-function-definition}
  How does the definition of preimage resistance for hash functions (Definition~\ref{def:preimage-resistance}) avoid the issues encountered with the SHA256 examples?
\end{exercise}

\ifsolutions
\begin{mysolution}
  The definition avoids the issues by considering hash function families rather than fixed functions like SHA256:
  \begin{enumerate}
    \item The definition uses a hash function family where the output size can grow with $\secpar$, ensuring that exhaustive search takes time exponential in $\secpar$ rather than constant time.
    \item The hash function is indexed by a key $\kappa$ selected by $\algo{Gen}(\secparam)$, preventing hardcoding of specific preimages since the adversary doesn't know which function from the family will be used.
  \end{enumerate}
  These changes allow the adversary's success probability to meaningfully depend on $\secpar$ and be made negligible by increasing it.
  Note that this doesn't make SHA256 itself secure in the asymptotic sense - it remains a fixed function with constant output size.
\end{mysolution}
\fi

\begin{exercise}[Optional]\label{ex:negl-property-proof}
  Prove property (1) of Lemma~\ref{lem:negl}: If $f$ is a negligible function and $p$ is a polynomial, then $f(\secpar) \cdot p(\secpar)$ is negligible.
  
  \textbf{Hints:}
  \begin{itemize}
    \item Any polynomial $p(\lambda)$ can be bounded by $\lambda^{a}$ for some positive integer $a$ and sufficiently large $\lambda$.
    \item To show $f(\lambda) \cdot p(\lambda)$ is negligible, for any $c > 0$, use the fact that $f$ is negligible with constant $c + a$.
  \end{itemize}
\end{exercise}

\ifsolutions
\begin{mysolution}
  We need to show that $f(\lambda) \cdot p(\lambda)$ is negligible, i.e., for any $c > 0$, we have $f(\lambda) \cdot p(\lambda) < \lambda^{-c}$ for sufficiently large $\lambda$.
  
  First, since $p$ is a polynomial, there exists a positive integer $a$ and $N_p \in \NN$ such that $p(\lambda) < \lambda^{a}$ for all $\lambda > N_p$.
  
  Let $c > 0$ be arbitrary. Since $f$ is negligible, for the constant $c + a > 0$, there exists $N_f \in \NN$ such that:
  \[
  f(\secpar) < \secpar^{-(c+a)} \quad \text{for all } \secpar > N_f
  \]
  
  Let $N = \max(N_p, N_f)$. Then for all $\secpar > N$:
  \[
  f(\secpar) \cdot p(\secpar) < \secpar^{-(c+a)} \cdot \secpar^{a} = \secpar^{-c}
  \]
  
  Since $c > 0$ was arbitrary, this proves that $f(\secpar) \cdot p(\secpar)$ is negligible.
\end{mysolution}
\fi

\begin{exercise}[Optional]\label{ex:concrete-security}
  An alternative to the asymptotic approach is the \emph{concrete security} approach.
  In this approach, we define $(t,\epsilon)$-hardness as follows:
  The $\game{\Game~\algo{Guess}}{}$ as defined in \autoref{fig:guessing-game} is $(t,\epsilon)$-hard if for any algorithm $\adv$ running in time $t$, we have $\pr{\game{\Game~\algo{Guess}}{} = 1} \le \epsilon$.
  
  Give a proposition about the concrete hardness of the guessing game, analogous to Proposition~\ref{prop:guessing-game}.
\end{exercise}

\ifsolutions
\begin{mysolution}
  \textbf{Proposition:} For any $t$ and $\lambda$, the $\game{\Game~\algo{Guess}}{}$ is $(t, 2^{-\lambda})$-hard.
  
  \textbf{Proof:} The proof is identical to Proposition~\ref{prop:guessing-game}, but we emphasize that the bound $2^{-\lambda}$ holds for \emph{all} $t$, not just polynomial time.
  
  Note: This shows that in the concrete setting, we explicitly state the security level ($2^{-\lambda}$) rather than saying it's ``negligible''.
  The concrete approach gives precise bounds rather than asymptotic statements.
\end{mysolution}
\fi

\begin{exercise}[Optional]\label{ex:foundations-hashing}
  This question explores problems when defining security for hash functions.
  Read Section 1 and Section 4 of ``Formalizing Human Ignorance''~\cite{VIETCRYPT:Rogaway06}.
  \begin{enumerate}
    \item What is the ``Foundations-of-Hashing dilemma''?
    \item To get around this, what alternative approach to defining security is suggested?
    What does ``explicitly given'' mean in this context?
    \item (Optional) See also the discussion in Appendix B.7 and the proposed approaches in Appendix B.5 of~\cite{AC:BerLan13}.
  \end{enumerate}
\end{exercise}

\ifsolutions
\begin{mysolution}
  \begin{enumerate}
    \item The Foundations-of-Hashing dilemma: For any fixed, unkeyed hash function (like SHA256), there always exists an adversary that simply prints a collision.
    This adversary exists mathematically even if we don't know how to construct it.
    This makes it impossible to prove collision resistance in the standard model.
    \item The alternative approach requires ``explicitly given'' reductions.
    This means the security proof must provide an explicit, constructive reduction that converts a collision-finding adversary into a solution to some other hard problem.
    The adversary that prints SHA256 collisions is not ``explicitly given'' because we cannot actually write down its code (we don't know any collisions).
  \end{enumerate}
\end{mysolution}
\fi
