\section{Recap Quiz}\label{sec:recap-quiz}

\begin{exercise}
  Let's say you have a game $\Game~G$ that you want to prove is hard, and you have the option to upper bound $\Pr[\game{G}{\adv} = 1]$ by $\advantage{AOMDL}{\bdv, \grgen}$, $\advantage{OMDL}{\cdv, \grgen}$, $\advantage{OMDL}{\ddv, \grgen}$ in the ROM, or $\advantage{DL}{\edv, \grgen}$.
  What is your order of preference?
  \begin{enumerate}
    \item $\advantage{DL}{\edv, \grgen}$, $\advantage{AOMDL}{\bdv, \grgen}$, $\advantage{OMDL}{\cdv, \grgen}$, $\advantage{OMDL}{\ddv, \grgen}$ in the ROM
    \item $\advantage{DL}{\edv, \grgen}$, $\advantage{OMDL}{\cdv, \grgen}$, $\advantage{AOMDL}{\bdv, \grgen}$, $\advantage{OMDL}{\ddv, \grgen}$ in the ROM
    \item $\advantage{OMDL}{\ddv, \grgen}$ in the ROM, $\advantage{DL}{\edv, \grgen}$, $\advantage{AOMDL}{\bdv, \grgen}$, $\advantage{OMDL}{\cdv, \grgen}$
    \item $\advantage{AOMDL}{\bdv, \grgen}$, $\advantage{DL}{\edv, \grgen}$, $\advantage{OMDL}{\cdv, \grgen}$, $\advantage{OMDL}{\ddv, \grgen}$ in the ROM
  \end{enumerate}
\end{exercise}

\ifsolutions
\begin{mysolution}
  The answer is (1).
\end{mysolution}
\fi

\begin{exercise}
  Consider the following game hop from $\game{\Game~\algo{Guess}}{}$ to $\game{\Game~\algo{Guess}_2}{}$:
  
  \begin{figure}[h!]
    \begin{center}
      \begin{tcolorbox}[width=3cm]
        \begin{pchstack}[center]
            \procedure[headlinesep=1pt]{$\game{\Game~\algo{Guess}_2}{}$}{%
              n \defeq 2^\lambda\\
              x \sample \ZZ_n \\
              \gamechange{$\pcassert x \neq 0$} \\
              x' \gets \adv() \\
              \pcreturn x = x'
            }
        \end{pchstack}
      \end{tcolorbox}
    \end{center}
    \caption{Without the \gamechange{highlighted} change, this is the guessing game from \autoref{fig:guessing-game}. With the change, this is $\game{\Game~\algo{Guess}_2}{}$.}
  \end{figure}
  
  Which of the following is correct?
  \begin{enumerate}
    \item $|\advantage{Guess}{\adv} - \Pr[\game{\algo{Guess}_2}{} = 1]| \leq \Pr[x = 0]$
    \item $|\advantage{Guess}{\adv} - \Pr[\game{\algo{Guess}_2}{} = 1]| = \Pr[x = 0]$
    \item $|\advantage{Guess}{\adv} - \Pr[\game{\algo{Guess}_2}{} = 1]| \geq \Pr[x = 0]$
  \end{enumerate}
\end{exercise}

\ifsolutions
\begin{mysolution}
  The answer is (1).
  By the Difference Lemma (Lemma~\ref{lem:difference}), when the games are identical except when event $E$ occurs (here $E$ is ``$x = 0$''), the difference in success probabilities is bounded by $\Pr[E]$.
\end{mysolution}
\fi

\begin{exercise}
  Using the result from the previous exercise, what can we conclude about $\Pr[\game{\algo{Guess}_2}{} = 1]$?
  \begin{enumerate}
    \item $\Pr[\game{\algo{Guess}_2}{} = 1] \geq \advantage{Guess}{\adv} + 2^{-\lambda}$
    \item $\Pr[\game{\algo{Guess}_2}{} = 1] \leq \advantage{Guess}{\adv} - 2^{-\lambda}$
    \item $\Pr[\game{\algo{Guess}_2}{} = 1] \leq \advantage{Guess}{\adv} + 2^{-\lambda}$
    \item $\Pr[\game{\algo{Guess}_2}{} = 1] = \advantage{Guess}{\adv} + 2^{-\lambda}$
  \end{enumerate}
\end{exercise}

\ifsolutions
\begin{mysolution}
  The answer is (3).
  
  From the previous exercise, we have $|\advantage{Guess}{\adv} - \Pr[\game{\algo{Guess}_2}{} = 1]| \leq 2^{-\lambda}$.
  
  This means:
  $-2^{-\lambda} \leq \advantage{Guess}{\adv} - \Pr[\game{\algo{Guess}_2}{} = 1] \leq 2^{-\lambda}$
  
  From the left inequality:
  $-2^{-\lambda} \leq \advantage{Guess}{\adv} - \Pr[\game{\algo{Guess}_2}{} = 1]$
  
  Rearranging:
  $\Pr[\game{\algo{Guess}_2}{} = 1] \leq \advantage{Guess}{\adv} + 2^{-\lambda}$
\end{mysolution}
\fi

\begin{exercise}
  Let $\adv$ be an algorithm with access to a random oracle $\pcoracle{H}: \{0,1\}^* \to \{0,1\}^\secpar$.
  The algorithm queries the random oracle with $x \in \{0,1\}^\secpar$.
  What is $\Pr[\pcoracle{H}(x) = x]$?
  \begin{enumerate}
    \item $0$
    \item $2^{-\secpar}$
    \item $1$
    \item $2^{-2\secpar}$
  \end{enumerate}
\end{exercise}

\ifsolutions
\begin{mysolution}
  The answer is (2).
  The random oracle returns a uniformly random value from $\{0,1\}^\secpar$ on the first query to any input.
  Since there are $2^\secpar$ possible outputs and exactly one of them equals $x$, we have $\Pr[\pcoracle{H}(x) = x] = 2^{-\secpar}$.
\end{mysolution}
\fi

\begin{exercise}
  Let $p$ be a prime and let $\adv$ be an algorithm with access to a random oracle $\pcoracle{H}: \{0,1\}^* \to \ZZ_p$.
  Before returning, the adversary queries the random oracle with distinct $x_1, x_2 \in \ZZ_p$ (in that order) and receives $y_1 = \pcoracle{H}(x_1)$, $y_2 = \pcoracle{H}(x_2)$.
  What is $\Pr[y_2 = x_1 y_1 x_2^{-1}]$?
  \begin{enumerate}
    \item $0$
    \item $1/p$
    \item $1/(p-1)$
    \item $1$
  \end{enumerate}
\end{exercise}

\ifsolutions
\begin{mysolution}
  The answer is (2).
  
  Since $x_1 \neq x_2$, the queries are to different inputs, so $y_1$ and $y_2$ are independent uniform random values from $\ZZ_p$. 
  Given any fixed $x_1, x_2, y_1$, the value $x_1 y_1 x_2^{-1}$ is a fixed element of $\ZZ_p$.
  Since $y_2$ is uniformly random and independent of $y_1$, we have $\Pr[y_2 = x_1 y_1 x_2^{-1}] = 1/p$.
\end{mysolution}
\fi

\begin{exercise}
  Let $\gparam$ be a group description.
  Let $\adv$ be an algorithm with access to a random oracle $\pcoracle{H}: \{0,1\}^* \to \ZZ_p$ where $p$ is prime.
  The adversary makes two distinct queries $(R_0, m_0)$ and $(R_1, m_1)$ where $R_i \in \GG$ and $m_i \in \{0,1\}^*$, receiving $y_0 = \pcoracle{H}(R_0, m_0)$ and $y_1 = \pcoracle{H}(R_1, m_1)$.
  What is $\Pr[R_0 \cdot g^{y_0} \cdot R_1^{-1} = g^{y_1}]$?
  \begin{enumerate}
    \item $0$
    \item $1/p$
    \item $1/(p-1)$
    \item $1$
  \end{enumerate}
\end{exercise}

\ifsolutions
\begin{mysolution}
  The answer is (2).
  
  Since the queries $(R_0, m_0)$ and $(R_1, m_1)$ are distinct, $y_0$ and $y_1$ are independent uniform random values from $\ZZ_p$.
  
  For any fixed $R_0, R_1 \in \GG$ and $y_0 \in \ZZ_p$, the value $R_0 \cdot g^{y_0} \cdot R_1^{-1}$ is a fixed element of $\GG$.
  Since $y_1$ is uniformly random in $\ZZ_p$, $g^{y_1}$ is uniformly random in $\GG$.
  Therefore, $\Pr[R_0 \cdot g^{y_0} \cdot R_1^{-1} = g^{y_1}] = 1/p$.
  
  Note: If this event occurred, we would have $R_0 \cdot g^{\pcoracle{H}(R_0, m_0)} = R_1 \cdot g^{\pcoracle{H}(R_1, m_1)}$, meaning
  \[
    \algo{TapCom.Commit}(R_0, m_0) = \algo{TapCom.Commit}(R_1, m_1)
  \] and the adversary would have won $\game{\Game~\algo{ComBind}}{\algo{TapCom}}$ in the random oracle model.
  In \autoref{lem:taproot-binding}, we generalize this to $q$ random oracle queries.
\end{mysolution}
\fi

\begin{exercise}
  TODO: Insert question about reading random oracle queries (table $T$), which is not possible with hash functions.
\end{exercise}

\begin{exercise}
  Which of the following represents a gap between security proofs and real-world security?
  \begin{enumerate}
    \item In practice, $\lambda$ is a fixed value (e.g., 128 or 256)
    \item Concrete hash functions like SHA256 are not hash functions
    \item Unfalsifiable assumptions cannot be validated experimentally
    \item Security models may not capture all attack vectors
    \item Random oracles do not exist
    \item Security proofs don't model users writing keys on post-it notes
    \item All of the above
  \end{enumerate}
\end{exercise}

\ifsolutions
\begin{mysolution}
  The answer is (7).
\end{mysolution}
\fi
