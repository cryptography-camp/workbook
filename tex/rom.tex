\section{Random Oracle Model}\label{sec:rom}

\subsection{Random Oracles}

A \emph{random oracle} (RO) is an oracle that can be provided to an algorithm that takes inputs in $\{0, 1\}^*$ and returns elements from some set $S$ where $|S| \geq 2^\secpar$.
The key property is that the oracle responds consistently: identical queries always yield identical responses, while distinct queries yield independent uniformly random values from $S$.

Formally, a random oracle can be described as follows.
Consider the random oracle $\pcoracle{H}$ in \autoref{fig:ro-example}.
The oracle maintains a table $T$ (initially empty) to ensure consistency.
When queried with input $x$, it checks whether $T[x]$ exists. If not, it samples a uniformly random element from $S$ and stores it as $T[x]$.
The oracle returns $T[x]$.


\begin{figure}[tbhp]
  \begin{center}
    \begin{tcolorbox}[width=8cm]
      \begin{pchstack}
        \procedure[headlinesep=1pt]{$\game{\Game~\algo{ROExample}}{}$}{%
          T \defeq \emptyset \\
          r \sample \{0,1\}^\secpar \\
          h \defeq \pcoracle{H}(r) \\
          \pcreturn \adv^{\pcoracle{H}}(h)
        }
        \pchspace
        \procedure[headlinesep=1pt]{Oracle $\pcoracle{H}(x)$}{%
          \pcif T[x] = \bot \pcthen \\
          \t T[x] \sample S \\
          \pcreturn T[x]
        }
      \end{pchstack}
    \end{tcolorbox}
  \end{center}
  \caption{Random oracle example game\label{fig:ro-example}}
\end{figure}

Before examining the consequences of random oracles, we present a simple lemma that will be useful for analyzing games that differ only when certain events occur.

\begin{lemma}[Difference Lemma]\label{lem:difference}
  Let $A$, $B$ and $E$ be events in a probability space. If $\Pr[A \wedge \neg E] = \Pr[B \wedge \neg E]$, then
  \[
  |\Pr[A] - \Pr[B]| \leq \Pr[E].
  \]
\end{lemma}

\begin{proof}
  \begin{align*}
    |\Pr[A] - \Pr[B]| &= |\Pr[A \wedge E] + \Pr[A \wedge \neg E] - (\Pr[B \wedge E] + \Pr[B \wedge \neg E])| \\
    &= |\Pr[A \wedge E] - \Pr[B \wedge E]|
  \end{align*}
  Since $\Pr[A \wedge E] \leq \Pr[E]$ and $\Pr[B \wedge E] \leq \Pr[E]$, both terms are in $[0, \Pr[E]]$, so their difference is at most $\Pr[E]$.
\end{proof}


\begin{lemma}\label{lem:ro-indistinguishable}
  No adversary against $\algo{ROExample}$ can distinguish between being given $h = \pcoracle{H}(r)$ for a random $r$ and being given a uniformly random value from $S$, except with probability negligible in the number of oracle queries.
  More precisely, let $\game{\Game~\algo{ROExample}}{}$ be as defined in \autoref{fig:ro-example}.
  For any adversary $\adv$ making $q$ queries to the random oracle $\pcoracle{H}$, we have
  \[
  \left|\Pr[\game{\algo{ROExample}}{} = 1] - \Pr[\game{\algo{ROExample}_1}{} = 1]\right| \leq \frac{q}{2^\secpar}
  \]
  where $\game{\algo{ROExample}_1}{}$ is defined in \autoref{fig:ro-game-hop}.
\end{lemma}

\begin{figure}[tbhp]
  \begin{center}
    \begin{tcolorbox}[width=8cm]
      \begin{pchstack}
        \procedure[headlinesep=1pt]{$\game{\Game~\algo{ROExample}_1}{}$}{%
          T \defeq \emptyset \\
          r \sample \{0,1\}^\secpar \\
          h \defeq \pcoracle{H}(r) \\
          \gamechange{$h' \sample S$} \\
          \pcreturn \adv^{\pcoracle{H}}(\gamechange{$h'$})
        }
        \pchspace
        \procedure[headlinesep=1pt]{Oracle $\pcoracle{H}(x)$}{%
          \gamechange{$\pcassert x \neq r$} \\
          \pcif T[x] = \bot \pcthen \\
          \t T[x] \sample S \\
          \pcreturn T[x]
        }
      \end{pchstack}
    \end{tcolorbox}
  \end{center}
  \caption{Game hop for random oracle indistinguishability\label{fig:ro-game-hop}}
\end{figure}

\begin{proof}
  By inspection of the code, we observe that $\game{\algo{ROExample}_1}{}$ is identical to $\game{\algo{ROExample}}{}$ except that:
  \begin{enumerate}
  \item The input to $\adv$ is changed from $h = \pcoracle{H}(r)$ to a uniformly random $h' \sample S$.
  \item The oracle $\pcoracle{H}$ asserts that no query equals $r$.
  \end{enumerate}
  
  Since $h'$ is uniformly random in $S$ and has the same distribution as $\pcoracle{H}(r)$ (when $r$ is not queried), the games are identical unless the adversary queries $r$ to the oracle.
  
  We apply Lemma~\ref{lem:difference}. Let $E$ denote the event that $\adv$ queries $r$ to the oracle. Let $A$ be the event that $\game{\algo{ROExample}}{} = 1$ and $B$ be the event that $\game{\algo{ROExample}_1}{} = 1$. 
  
  When $\neg E$ occurs (i.e., $r$ is not queried), the two games behave identically, so $\Pr[A \wedge \neg E] = \Pr[B \wedge \neg E]$. 
  
  By Lemma~\ref{lem:difference}:
  \begin{align*}
    \left|\Pr[\game{\algo{ROExample}}{} = 1] - \Pr[\game{\algo{ROExample}_1}{} = 1]\right| &\leq \Pr[E] \\
    &= \Pr[\exists i \in [q] : x_i = r] \\
    &\leq \sum_{i=1}^q \Pr[x_i = r] \tag{union bound} \\
    &= \sum_{i=1}^q \frac{1}{2^\secpar} = \frac{q}{2^\secpar}
  \end{align*}
  where $x_1, \ldots, x_q$ are the queries made by $\adv$ to $\pcoracle{H}$.
\end{proof}

\subsection{ROM}

  For a large class of cryptographic schemes, the properties of hash functions seen so far are insufficient to prove security.
  Instead, we prove security of a modified scheme where hash function evaluations are replaced with random oracle calls.
  This is the \emph{random oracle model} (ROM)~\cite{CCS:BelRog93}.
  Random oracles cannot be instantiated in the real world because their output must be distributed uniformly at random unless explicitly queried.
  On the other hand, the output of a hash function is deterministic once the key is known.
  Moreover, there's no way to formally define an ``explicit query'' for public hash functions.

  There exist contrived schemes provably secure in the ROM that become insecure when instantiated with any concrete hash function (see optional exercise).
  Despite these theoretical concerns, the ROM has proven very useful in practice for analyzing real-world cryptographic protocols~\cite{DCC:KobMen15}.

\subsection{Hash Commitments}

\begin{definition}[Hash Commitment Scheme]
  Let $\algo{H}$ be a hash function.
  The hash commitment scheme $\algo{HashCom}[\algo{H}]$ is defined as follows:
  \begin{itemize}
    \item $\algo{Setup}(1^\secpar) \rightarrow \params$: Output $\params = (\mathcal{M} \defeq \{0, 1\}^*, \mathcal{R} \defeq \{0, 1\}^\secpar)$.
    \item $\algo{Commit}(\params, m; r) \rightarrow C$: Output $C = \algo{H.Eval}(m \|\| r)$.
  \end{itemize}
\end{definition}

\begin{lemma}\label{lem:hash-com-hiding}
  The hash commitment scheme $\algo{HashCom}[\algo{H}]$ is hiding in the random oracle model for $\algo{H}$.
  More precisely, for any algorithm $\adv$ making at most $q$ queries to $\algo{H}$,
  \[
  \advantage{Hide}{\adv, \algo{HashCom}} \leq \frac{2q}{2^\secpar}.
  \]
\end{lemma}

The proof is left as an exercise (see \autoref{ex:hash-com-hiding}).

\subsection{Taproot Commitments}

\begin{definition}[Taproot Commitment Scheme~\cite{add:bip-taproot}]
  Let $\grgen$ be a group generation algorithm and $\algo{H}$ be a hash function with output in $\ZZ_p$ where $p$ is the group order. 
  The Taproot commitment scheme $\algo{TapCom}[\grgen, \algo{H}]$ is defined as follows:
  \begin{itemize}
    \item $\algo{Setup}(1^\secpar) \rightarrow \params$: Run $\gparam \gets \grgen(1^\secpar)$ and $\kappa \gets \algo{H.Gen}(1^\secpar)$. Parse $\gparam = (\GG, p, g)$. Output $\params = (\gparam, \kappa, \mathcal{M} \defeq \{0,1\}^*, \mathcal{R} \defeq \GG)$.
    \item $\algo{Commit}(\params, m; R) \rightarrow C$: Parse $\params$ to obtain $g$ and $\kappa$. Output $C = R \cdot g^{\algo{H.Eval}(\kappa, R \| m)}$.
  \end{itemize}
\end{definition}

\begin{lemma}\label{lem:taproot-binding}
  Let $\grgen$ be a group generation algorithm and $\algo{H}$ be a hash function.
  The Taproot commitment scheme $\algo{TapCom}[\grgen, \algo{H}]$ is binding in the random oracle model for $\algo{H}$.
  More precisely, for any algorithm $\adv$ making at most $q$ queries to $\algo{H}$,
  \[
  \advantage{Bind}{\adv, \algo{TapCom}} \leq \frac{(q+2)^2}{2^\secpar}.
  \]
\end{lemma}

The proof is left as an exercise (see \autoref{ex:taproot-binding}).

\subsection{Exercises}

\begin{exercise}\label{ex:rom-contrived-hashcom}
  Consider a modified hash commitment scheme $\algo{HashCom}_y[\algo{H}]$ defined as follows:
  \begin{itemize}
    \item $\algo{Setup}(1^\secpar) \rightarrow \params$: Same as $\algo{HashCom}[\algo{H}]$.
    \item $\algo{Commit}(\params, m; r) \rightarrow C$: 
    \begin{align*}
      C = \begin{cases}
        \algo{H.Eval}(m) & \text{if } \algo{H.Eval}(0) = y \\
        \algo{H.Eval}(m \| r) & \text{otherwise}
      \end{cases}
    \end{align*}
  \end{itemize}
  
  \begin{enumerate}
    \item Using Lemma~\ref{lem:hash-com-hiding}, argue that for all $y \in \{0,1\}^\secpar$, $\algo{HashCom}_y[\algo{H}]$ is hiding in the ROM.
    \item Show that there exists $y$ such that $\algo{HashCom}_y$ instantiated with SHA256 is not hiding.
  \end{enumerate}
\end{exercise}

\ifsolutions
\begin{mysolution}
  \begin{enumerate}
    \item In the ROM, $\algo{H}$ behaves as a random oracle. For any fixed $y$, we have $\Pr[\algo{H.Eval}(0) = y] = \frac{1}{2^\secpar}$, which is negligible. With overwhelming probability, the scheme behaves identically to the standard $\algo{HashCom}[\algo{H}]$, which is hiding by Lemma~\ref{lem:hash-com-hiding}.
    
    More formally, the hiding advantage is bounded by:
    \[
    \advantage{Hide}{\adv, \algo{HashCom}_y} \leq \Pr[\algo{H.Eval}(0) = y] + \advantage{Hide}{\adv, \algo{HashCom}} \leq \frac{1}{2^\secpar} + \frac{2q}{2^\secpar}
    \]
    
    \item Let $y = \text{SHA256}(0)$. When instantiated with SHA256, we always have $\algo{H.Eval}(0) = y$, so the commitment becomes $C = \text{SHA256}(m)$ without any randomness. This is deterministic and therefore not hiding: given two messages $m_0, m_1$, the adversary can compute $\text{SHA256}(m_0)$ and $\text{SHA256}(m_1)$ and compare with $C$ to determine which message was committed to.
  \end{enumerate}
\end{mysolution}
\fi

\begin{exercise}\label{ex:hash-com-hiding}
  Prove Lemma~\ref{lem:hash-com-hiding} using a similar technique as in Lemma~\ref{lem:ro-indistinguishable}.
\end{exercise}

\ifsolutions
\begin{mysolution}
\autoref{fig:hashcom-hiding-rom} shows the hiding game for $\algo{HashCom}[\algo{H}]$ where $\algo{H}$ is modeled as a random oracle.

\begin{figure}[h]
  \begin{center}
    \begin{tcolorbox}[width=8cm]
      \begin{pchstack}
        \procedure[headlinesep=1pt]{$\game{\Game~\algo{ComHide}}{\algo{HashCom}}$}{%
          \params \gets \algo{HashCom.Setup}(1^\secpar) \\
          (m_0, m_1) \gets \adv(\params) \\
          T \defeq \emptyset \\
          b \sample \{0,1\} \\
          r \sample \{0,1\}^\secpar \\
          c \defeq \pcoracle{H}(m_b \| r) \\
          b' \gets \adv^{\pcoracle{H}}(c) \\
          \pcreturn b = b'
        }
        \pchspace
        \procedure[headlinesep=1pt]{Oracle $\pcoracle{H}(x)$}{%
          \pcif T[x] = \bot \pcthen \\
          \t T[x] \sample \{0,1\}^\secpar \\
          \pcreturn T[x]
        }
      \end{pchstack}
    \end{tcolorbox}
  \end{center}
  \caption{The hiding game for $\algo{HashCom}[\algo{H}]$ in the random oracle model.}
  \label{fig:hashcom-hiding-rom}
\end{figure}

Now consider the modified game shown in \autoref{fig:hashcom-hiding-rom-modified}:

\begin{figure}[h]
  \begin{center}
    \begin{tcolorbox}[width=8cm]
      \begin{pchstack}
        \procedure[headlinesep=1pt]{$\game{\Game~\algo{ComHide}_1}{\algo{HashCom}}$}{%
          \params \gets \algo{HashCom.Setup}(1^\secpar) \\
          (m_0, m_1) \gets \adv(\params) \\
          T \defeq \emptyset \\
          b \sample \{0,1\} \\
          r \sample \{0,1\}^\secpar \\
          c \defeq \pcoracle{H}(m_b \| r) \\
          \gamechange{$c' \sample \{0,1\}^\secpar$} \\
          b' \gets \adv^{\pcoracle{H}}(\gamechange{$c'$}) \\
          \pcreturn b = b'
        }
        \pchspace
        \procedure[headlinesep=1pt]{Oracle $\pcoracle{H}(x)$}{%
          \gamechange{$\pcassert x \neq m_0 \| r$} \\
          \gamechange{$\pcassert x \neq m_1 \| r$} \\
          \pcif T[x] = \bot \pcthen \\
          \t T[x] \sample \{0,1\}^\secpar \\
          \pcreturn T[x]
        }
      \end{pchstack}
    \end{tcolorbox}
  \end{center}
  \caption{Modified hiding game for $\algo{HashCom}[\algo{H}]$ with changes \gamechange{highlighted}.}
  \label{fig:hashcom-hiding-rom-modified}
\end{figure}

Note that in $\game{\algo{ComHide}_1}{\algo{HashCom}}$, the value $b$ is independent of the adversary's view, so:
\[
\Pr[\game{\algo{ComHide}_1}{\algo{HashCom}} = 1] = \frac{1}{2}
\]

The games differ only when $\adv$ queries either $m_0 \| r$ or $m_1 \| r$ to the oracle.
By the Difference Lemma (\autoref{lem:difference}), we can bound the difference between the games by the probability of this event:

\begin{align*}
  \advantage{Hide}{\adv, \algo{HashCom}} &= \left|\Pr[\game{\algo{ComHide}}{\algo{HashCom}} = 1] - \frac{1}{2}\right| \\
  &= \left|\Pr[\game{\algo{ComHide}}{\algo{HashCom}} = 1] - \Pr[\game{\algo{ComHide}_1}{\algo{HashCom}} = 1]\right| \\
  &\leq \Pr[\exists i \in [q] : x_i \in \{m_0 \| r, m_1 \| r\}]
\end{align*}

Since $r$ is uniformly random from $\{0,1\}^\secpar$ and unknown to $\adv$, and the adversary knows both $m_0$ and $m_1$, we apply the union bound:

\begin{align*}
  \Pr[\exists i \in [q] : x_i \in \{m_0 \| r, m_1 \| r\}] &\leq \sum_{i=1}^q \Pr[x_i = m_0 \| r \text{ or } x_i = m_1 \| r] \\
  &\leq \sum_{i=1}^q \left(\Pr[x_i = m_0 \| r] + \Pr[x_i = m_1 \| r]\right) \\
  &= \sum_{i=1}^q \left(\frac{1}{2^\secpar} + \frac{1}{2^\secpar}\right) \\
  &= \sum_{i=1}^q \frac{2}{2^\secpar} = \frac{2q}{2^\secpar}
\end{align*}
\end{mysolution}
\fi

\begin{exercise}\label{ex:taproot-binding}
  Prove Lemma~\ref{lem:taproot-binding}.
\end{exercise}

\ifsolutions
\begin{mysolution}
  Let $\adv$ be an adversary against the binding property of $\algo{TapCom}[\grgen, \algo{H}]$. The adversary wins if it outputs $(m_0, m_1, R_0, R_1)$ with $m_0 \neq m_1$ such that:
  \[
  R_0 \cdot g^{\algo{H}(R_0 \| m_0)} = R_1 \cdot g^{\algo{H}(R_1 \| m_1)}
  \]

  Rearranging, this equality holds if and only if:
  \[
  R_0 \cdot g^{\algo{H}(R_0 \| m_0)} \cdot R_1^{-1} = g^{\algo{H}(R_1 \| m_1)}
  \]

  Our strategy is to check for binding collisions as queries are made to the random oracle.
  Consider the distinct queries to the random oracle of the form $R_i \| m_i$ and define the collision predicate:
  \[
  c(i, j) = \text{true} \iff R_i \cdot g^{\algo{H}(R_i \| m_i)} \cdot R_j^{-1} = g^{\algo{H}(R_j \| m_j)}
  \]
  
  The probability that the adversary breaks binding is:
  \[
  \advantage{Bind}{\adv, \algo{TapCom}} = \Pr\left[\bigvee_{i=1}^{q'} \bigvee_{j=i+1}^{q'} c(i,j) = \text{true}\right]
  \]
  where $q'$ is the number of distinct queries.
  
  For all $j > i$, we have $\Pr[c(i,j) = \text{true}] = \frac{1}{p}$, since $\algo{H}(R_j \| m_j)$ is a fresh uniformly random value in $\ZZ_p$, independent of $R_i$, $\algo{H}(R_i \| m_i)$, and $R_j$.
  
  By the union bound:
  \begin{align*}
    \advantage{Bind}{\adv, \algo{TapCom}} &= \Pr\left[\bigvee_{i=1}^{q'} \bigvee_{j=i+1}^{q'} c(i,j) = \text{true}\right] \\
    &\leq \sum_{i=1}^{q'} \sum_{j=i+1}^{q'} \Pr[c(i,j) = \text{true}] \\
    &= \sum_{i=1}^{q'} \sum_{j=i+1}^{q'} \frac{1}{p} \\
    &= \binom{q'}{2} \cdot \frac{1}{p} < \frac{(q')^2}{2p}
  \end{align*}
  
  The total number of queries includes the adversary's queries (at most $q$) plus the 2 queries made by the binding game when checking $C_0 = C_1$. Thus $q' \leq q + 2$.
  
  Since $p \geq 2^\secpar$, we have:
  \[
  \advantage{Bind}{\adv, \algo{TapCom}} < \frac{(q+2)^2}{2p} \leq \frac{(q+2)^2}{2 \cdot 2^\secpar} = \frac{(q+2)^2}{2^{\secpar+1}} < \frac{(q+2)^2}{2^\secpar}
  \]
\end{mysolution}
\fi

\begin{exercise}[Optional]
  \begin{enumerate}
    \item Let $\algo{H}$ be a hash function with output in $\{0,1\}^\secpar$ and let $\kappa$ be chosen uniformly at random from a set of size $2^\secpar$. Define $A = \algo{H.Eval}(\kappa, 0) \| \algo{H.Eval}(\kappa, 1)$. What is the size of the set over which $A$ is uniformly distributed?
    \item Let $\algo{H}$ be a random oracle with output in $\{0,1\}^\secpar$. Define $B = \algo{H}(0) \| \algo{H}(1)$. What is the size of the set over which $B$ is uniformly distributed?
  \end{enumerate}
\end{exercise}

\ifsolutions
\begin{mysolution}
  \begin{enumerate}
    \item There are $2^\secpar$ possibilities for $\kappa$, and each determines a unique value of $A$. Therefore, $A$ is uniformly distributed over a set of size at most $2^\secpar$.
    \item Since $\algo{H}(0)$ and $\algo{H}(1)$ are independent uniformly random values in $\{0,1\}^\secpar$, $B$ is uniformly distributed over a set of size $2^{2\secpar}$.
  \end{enumerate}
\end{mysolution}
\fi

\begin{exercise}[Optional]
  Let $\algo{H}$ be a random oracle with output in $\{0,1\}^\secpar$. Define the function $G: (\{0,1\}^*)^{\secpar+1} \rightarrow \{0,1\}^\secpar$ as:
  \[
  G(x_0, x_1, \ldots, x_\secpar) = \algo{H}(x_0) \oplus \algo{H}(x_1) \oplus \cdots \oplus \algo{H}(x_\secpar)
  \]
  where $\oplus$ denotes bitwise XOR. Is $G$ preimage resistant? That is, given $y \in \{0,1\}^\secpar$, is it hard for a \ppt adversary with oracle access to $\algo{H}$ to find $(x_0, \ldots, x_\secpar)$ such that $G(x_0, \ldots, x_\secpar) = y$?
\end{exercise}

\ifsolutions
\begin{mysolution}
  No, $G$ is not preimage resistant. Wagner's algorithm for the generalized birthday problem~\cite{C:Wagner02} can find a preimage efficiently.
\end{mysolution}
\fi

\begin{exercise}[Optional]
  Why is the ROM not considered adequate to model quantum computation?
\end{exercise}

\begin{exercise}[Optional]
  Study the diagonalization argument in Section 4 of Canetti, Goldreich, and Halevi~\cite{STOC:CanGolHal98}.
  Explain how this argument demonstrates the existence of a contrived signature scheme that is provably secure in the random oracle model but becomes insecure when the random oracle is instantiated with any concrete hash function (not just a specific fixed hash function as considered in \autoref{ex:rom-contrived-hashcom}).
\end{exercise}
