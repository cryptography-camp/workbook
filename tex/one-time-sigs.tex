\section{One-Time Signatures}\label{sec:one-time-sigs}

\subsection{Syntax}

\begin{definition}[One-Time Signature]
    A one-time signature scheme $\algo{OTS}$ is a tuple of p.p.t algorithms $(\algo{Setup}, \algo{KeyGen}, \algo{Sign}, \algo{Verify})$ with message space $\mathcal{M}$, where
    \begin{itemize}
        \item $\algo{Setup}(1^\secpar)\rightarrow \params$ is a p.p.t algorithm that generates the public parameters $\params$ for the one-time signature scheme.
        \item $\algo{KeyGen}(\params)\rightarrow (\sk, \pk)$ is a p.p.t algorithm that generates the key pair $(\sk, \pk)$ of the one-time signature scheme.
        \item $\algo{Sign}(\params, \sk, m) \rightarrow \sigma$ is a p.p.t algorithm that outputs a signature $\sigma$ for the message $m$ under the one-time signature scheme with secret key $\sk$.
        \item $\algo{Verify}(\params, \pk, m, \sigma) \rightarrow b$ is a p.p.t algorithm that outputs $b = 1$ if the signature $\sigma$ is valid for the message $m$ under the one-time signature scheme with public key $\pk$. Otherwise, it outputs $b = 0$.
    \end{itemize}
\end{definition}


\subsection{Correctness}

\subsection{Security}

\begin{definition}[EUF-CMA Security]\label{def:euf-cma-ot}
    A one-time signature scheme $\algo{OTS}$ is \emph{EUF-CMA secure} (Existential Unforgeability under Chosen Message Attack) if for all p.p.t. adversaries $\adv$, it holds that
    \[
    \advantage{EUF-CMA}{\adv, \algo{OTS}} \defeq \pr{\game{\Game~\algo{EUF-CMA}}{\algo{OTS}} = 1} = \negl.
    \]
\end{definition}

\begin{figure}[tbh]
 \begin{tcolorbox}%[left=0mm,right=0mm]
  \begin{pchstack}[center]
   \procedure{$\game{\Game~\algo{EUF-CMA}}{\algo{OTS}}$}{%
     \params \gets \algo{Setup}(1^\secpar) \\
     (\sk,\pk) \gets \algo{KeyGen}(\params) \\
     \mathcal{Q} \defeq \emptyset\\
     (m^*, \sigma) \gets \adv^{\pcoracle{Sign}}(\params, \pk) \\
     \pcreturn m^* \notin \mathcal{Q} \wedge \algo{Verify}(\params, \pk, m^*, \sigma) = 1
   }
   \pchspace
   \procedure{Oracle $\pcoracle{Sign}(m)$}{%
     \pcassert \mathcal{Q} = \emptyset \\
     \mathcal{Q} \defeq \{m\} \\
     \pcreturn \algo{Sign}(\params, \sk, m)
   }
  \end{pchstack}
 \end{tcolorbox}
 \caption{The EUF-CMA security game for a one-time signature scheme.}
 \label{fig:euf-cma-ots}
\end{figure}

\begin{figure}[t!]
 \begin{tcolorbox}
  \begin{pchstack}[center]
  \begin{pcvstack}
    \procedure{$\algo{Setup}(1^\secpar)$}{%
      \params \defeq \algo{H.Gen}(\secparam) \\
      \pcreturn \params
    }
    \pcvspace
    \procedure{$\algo{KeyGen}(\params)$}{%
      x_{i, j} \sample \{0, 1\}^\secpar \text{ for } i \in [1, k], j \in \{0, 1\} \\
      \sk \defeq (x_{1, 0}, x_{1, 1}, \dots, x_{k, 0}, x_{k, 1}) \\
      \pk \defeq (\algo{H.Eval}(x_{1, 0}), \algo{H.Eval}(x_{1, 1}), \dots, \\
      \t \t \algo{H.Eval}(x_{k, 0}), \algo{H.Eval}(x_{k, 1})) \\
      \pcreturn (\sk, \pk)
    }
  \end{pcvstack}
  \pchspace
  \pchspace
  \begin{pcvstack}
    \procedure{$\algo{Sign}(\params, \sk, m)$}{%
        \pccomment{$\algo{Sign}$ must be called at most once per $\sk$.}\\
        (m_1, \ldots, m_k) \defeq m \\
        (\sk_{1, 0}, \sk_{1, 1}), \ldots, (\sk_{k, 0}, \sk_{k, 1}) \defeq \sk \\
        \sigma \defeq (\sk_{1, m_1}, \ldots, \sk_{k, m_k}) \\
        \pcreturn \sigma
    }
    \pcvspace
    \procedure{$\algo{Verify}(\params, \pk, m, \sigma)$}{%
      (m_1, \ldots, m_k) \defeq m \\
      (\pk_{1, 0}, \pk_{1, 1}), \ldots, (\pk_{k, 0}, \pk_{k, 1}) \defeq \pk \\
      \pcreturn \bigwedge_{i=1}^k \algo{H.Eval}(\sigma_i) = \pk_{i, m_i}
    }
  \end{pcvstack}
  \end{pchstack}
 \end{tcolorbox}
 \caption{The one-time Lamport signature scheme $\algo{LS}[\algo{H, k}]$.}
 \label{fig:ots-lamport}
\end{figure}


\begin{example}[Lamport signatures]
    The Lamport one-time signature scheme $\algo{LS}[\algo{H, k}]$ is defined in \autoref{fig:ots-lamport}.
    It is parameterized by a hash function $\algo{H}$ and an integer $k$ that determines the message space $\mathcal{M} = \{0, 1\}^k$.
\end{example}

\begin{theorem}[Lamport signatures are EUF-CMA secure]\label{thm:ots-lamport-euf-cma}
    Let $\algo{H}$ be a preimage-resistant hash function.
    Then the Lamport one-time signature scheme $\algo{LS}[\algo{H, k}]$ is EUF-CMA secure.
    More precisely, for any p.p.t. adversary $\adv$ against the EUF-CMA security of $\algo{LS}[\algo{H, k}]$,
    there exists a p.p.t. adversary $\bdv$ against the preimage resistance of $\algo{H}$ such that
    \[
    \advantage{EUF-CMA}{\adv, \algo{LS}[\algo{H, k}]} \leq 2k\advantage{Preimage}{\bdv, \algo{H}}.
    \]
\end{theorem}

\begin{figure}[tbh]
 \begin{tcolorbox}%[left=0mm,right=0mm]
  \begin{pchstack}[center]
  \begin{pcvstack}
   \procedure{$\game{\Game~\algo{EUF-CMA}}{\algo{LS}[\algo{H, k}]}$}{%
     \params \gets \algo{Setup}(1^\secpar) \\
     (\sk,\pk) \gets \algo{KeyGen}(\params) \\
     \mathcal{Q} \defeq \emptyset\\
     (m^*, \sigma) \gets \adv^{\pcoracle{Sign}}(\params, \pk) \\
     \pcreturn m^* \notin \mathcal{Q} \wedge \\
      \t \algo{Verify}(\params, \pk, m^*, \sigma) = 1
   }
   \pcvspace
   \procedure{Oracle $\pcoracle{Sign}(m)$}{%
     \pcassert \mathcal{Q} = \emptyset \\
     \mathcal{Q} \defeq \{m\} \\
     \pcreturn \algo{Sign}(\params, \sk, m)
   }
  \end{pcvstack}
  \pchspace
   \begin{pcvstack}
    \procedure{$\game{\Game~\algo{G_1}}{}$}{%
      \params \gets \algo{Setup}(1^\secpar) \\
      (\sk,\pk) \gets \algo{KeyGen}(\params) \\
      \gamechange{$i \sample [1, k] \pcsc j \sample \{0, 1\}$} \\
      \mathcal{Q} \defeq \emptyset\\
      (m^*, \sigma) \gets \adv^{\pcoracle{Sign}}(\params, \pk) \\
      \pcif \mathcal{Q} = \emptyset \pcthen \\
      \t m \sample \{0, 1\}^k \\
      \t \pcoracle{Sign}(m) \\
      \pcreturn m^* \notin \mathcal{Q} \wedge \\
      \t \algo{Verify}(\params, \pk, m^*, \sigma) = 1
    }
    \pcvspace
    \procedure{Oracle $\pcoracle{Sign}(m)$}{%
      \pcassert \mathcal{Q} = \emptyset \\
      \gamechange{$\pcassert m_i \neq j$} \\
      \mathcal{Q} \defeq \{m\} \\
      \pcreturn \algo{Sign}(\params, \sk, m)
    }
   \end{pcvstack}
  \pchspace
   \begin{pcvstack}
    \procedure{$\game{\Game~\algo{G_2}}{\algo{OTS}}$}{%
      \params \gets \algo{Setup}(1^\secpar) \\
      (\sk,\pk) \gets \algo{KeyGen}(\params) \\
      i \sample [1, k]  \\
      j \sample [0, 1]  \\
      \mathcal{Q} \defeq \emptyset\\
      (m^*, \sigma) \gets \adv^{\pcoracle{Sign}}(\params, \pk) \\
      \pcif \mathcal{Q} = \emptyset \pcthen \\
      \t m \sample \{0, 1\}^k \\
      \t \pcoracle{Sign}(m) \\
      \gamechange{$\pcassert m^*_i = j$} \\
      \pcreturn m^* \notin \mathcal{Q} \wedge \\
      \t \algo{Verify}(\params, \pk, m^*, \sigma) = 1
    }
    \pcvspace
    \procedure{Oracle $\pcoracle{Sign}(m)$}{%
      \pcassert \mathcal{Q} = \emptyset \\
      \pcassert m_i \neq j \\
      \mathcal{Q} \defeq \{m\} \\
      \pcreturn \algo{Sign}(\params, \sk, m)
    }
   \end{pcvstack}
  \end{pchstack}
 \end{tcolorbox}
 \caption{Sequence of games for the proof that Lamport signatures are EUF-CMA secure. Differences are \gamechange{highlighted}.}
 \label{fig:euf-cma-ots-lamport}
\end{figure}


\begin{proof}
    \autoref{fig:euf-cma-ots-lamport} shows a sequence of games for the proof that Lamport signatures are EUF-CMA secure.
    The first game is identical to the $\Game~\algo{EUF-CMA}$ game.
    Each subsequent game introduces a small change compared to the previous game.
    Then we can construct an algorithm that runs a $\Game~\algo{EUF-CMA}$ adversary $\adv$ internally and wins $\game{\Game~\algo{Preimage}}{\algo{H}}$ with the same probability as $\adv$ winning the last  $\Game~\algo{G_2}$.
    The rest of the proof is left as an exercise.
\end{proof}

\subsection{Exercises}

\begin{exercise}
  What is "existential unforgeability" in "EUF-CMA" (\autoref{def:euf-cma-ot}) and what is "chosen message attack"?
\end{exercise}

\ifsolutions
\begin{mysolution}
  \textbf{Existential unforgeability:} The adversary wins if it can forge a signature for \emph{any} message of its choice.
  \textbf{Chosen message attack:} The adversary has access to a signing oracle and can obtain a signature for a message of its choice before attempting the forgery.
\end{mysolution}
\fi

\begin{exercise}
  To see why the security definition for signatures includes a signing oracle, consider $\game{\Game~\algo{G}}{\algo{OTS}}$ which is the same as $\game{\Game~\algo{EUF-CMA}}{\algo{OTS}}$ except that the adversary $\adv$ is not given access to the signing oracle.
  Give an example for a one time signature scheme that is correct and $\pr{\game{\Game~\algo{G}}{\algo{OTS}} = 1} = negl$ for all p.p.t. adversaries $\adv$ but is not EUF-CMA secure.
\end{exercise}

\ifsolutions
\begin{mysolution}
  Consider the following scheme:
  \begin{itemize}
      \item $\algo{KeyGen}(\params) = (\sk = x, \pk = \algo{H.Eval}(x))$ where $x \sample \{0,1\}^\secpar$.
      \item $\algo{Sign}(\params, \sk, m) = \sk$ (simply outputs the secret key).
      \item $\algo{Verify}(\params, \pk, m, \sigma) = 1$ if and only if $\algo{H.Eval}(\sigma) = \pk$.
  \end{itemize}
  This scheme is correct and would be secure in $\Game~\algo{G}$ (without signing oracle) since the adversary cannot forge without knowing $\sk$, which requires solving the preimage problem.
  However, this scheme is obviously insecure in practice: anyone who sees one signature learns the secret key and can forge signatures for any message!
  The $\Game~\algo{EUF-CMA}$ game with signing oracle correctly captures this insecurity.
\end{mysolution}
\fi

\begin{exercise}
  Is $(x_{1,0}, x_{2,0}, x_{3,1})$ a valid signature for the message $m = \{0, 0, 1\}$ under the Lamport signature scheme $\algo{LS}[\algo{H, 3}]$?
\end{exercise}

\begin{exercise}
  Is the Lamport signature scheme $\algo{LS}[\algo{H, k}]$ correct?
\end{exercise}

\begin{exercise}
  Prove theorem \autoref{thm:ots-lamport-euf-cma}:
  \begin{enumerate}
      \item Express $\pr{\game{\Game~\algo{G_1}}{} = 1}$ in terms of $\pr{\game{\Game~\algo{EUF-CMA}}{\algo{LS[\algo{H}, k]}} = 1}$.
      \item Express $\pr{\game{\Game~\algo{G_2}}{} = 1}$ in terms of $\pr{\game{\Game~\algo{G_1}}{} = 1}$.
      \item Give an algorithm $\bdv^\adv$ that wins $\game{\Game~\algo{Preimage}}{\algo{H}}$ with probability $\pr{\game{\Game~\algo{G_2}}{} = 1}$.
      \item Put it all together to prove \autoref{thm:ots-lamport-euf-cma}.
  \end{enumerate}
\end{exercise}

\ifsolutions
\begin{mysolution}
   We have
  \begin{align*}
      \pr{\game{\Game~\algo{G_1}}{} = 1} &= \pr{\game{(\Game~\algo{EUF-CMA}}{\algo{LS[\algo{H}, k]}}) = 1) \wedge (m_i \neq j)} \\
      &= \pr{\game{\Game~\algo{EUF-CMA}}{\algo{LS[\algo{H}, k]}} = 1 | m_i \neq j} \pr{m_i \neq j} \\
      &= \frac{1}{2}\pr{\game{\Game~\algo{EUF-CMA}}{\algo{LS[\algo{H}, k]}} = 1} \\
  \end{align*}
  $\game{\Game~\algo{G_2}}{} = 1$ only when $m^* \neq m$ and $m^*_i = j$.
  Therefore,
  \begin{align*}
      \pr{\game{\Game~\algo{G_2}}{} = 1} &= \pr{\game{\Game~\algo{G_1}}{} = 1 \wedge m^*_i = j} \\
      &= \pr{m^*_i = j | \game{\Game~\algo{G_1}}{} = 1} \pr{\game{\Game~\algo{G_1}}{} = 1} \\
      &\geq \frac{1}{k}\pr{\game{\Game~\algo{G_1}}{} = 1} \\
  \end{align*}
  The last inequality holds because when $\Game~\algo{G_1}$ succeeds, we have $m \neq m^*$ (at least one position differs), $m_i \neq j$, and $i$ is uniformly random. The probability that position $i$ is one where $m$ and $m^*$ differ is at least $1/k$. When they differ at position $i$ and $m_i \neq j$, then $m^*_i = j$ (since the message space is binary).
  $\bdv^\adv$ is the same as $\Game~\algo{G_2}$ except that
  \begin{itemize}
      \item it gets a preimage challenge $y$ from a challenger and
      \item replaces $\pk_{i, j}$ with $y$
      \item it returns $\sigma_i$.
  \end{itemize}
  Since $m_i \neq j$, the inputs to the adversary (including the signing oracle call) has the same distribution as in $\Game~\algo{G_2}$.
  Moreover, if $\game{\Game~\algo{G_2}}{}$ returns 1, then $\algo{H.Eval}(\sigma_i) = \pk_{i, j}$
  Therefore,
  \[
  \advantage{Preimage}{\bdv^\adv, \algo{H}} = \pr{\game{\Game~\algo{G_2}}{} = 1}
  \]
  and then combining the inequalities:
  \begin{align*}
    \advantage{EUF-CMA}{\adv, \algo{LS}[\algo{H, k}]} &= \pr{\game{\Game~\algo{EUF-CMA}}{\algo{LS[\algo{H}, k]}} = 1} \\
    &= 2 \cdot \pr{\game{\Game~\algo{G_1}}{} = 1} \\
    &\leq 2k \cdot \pr{\game{\Game~\algo{G_2}}{} = 1} \\
    &= 2k \cdot \advantage{Preimage}{\bdv^\adv, \algo{H}}
  \end{align*}
\end{mysolution}
\fi

\begin{exercise}
  Let $\Game~\algo{G'}$ be the same as $\Game~\algo{EUF-CMA}$ except that the signing oracle can be called twice. Where does your proof break?
\end{exercise}

\begin{exercise}[Optional]
  Why are Lamport signatures considered post-quantum?
\end{exercise}

\begin{exercise}
  TODO: Consider lamport sigs where the public key is a hash, or where the secret keys are derived.
\end{exercise}

