\section{Commitments}\label{sec:commitments}

\subsection{Syntax}
\begin{definition}[Commitment Scheme]
  A \emph{commitment scheme} $\algo{Com}$ is a pair of polynomial-time algorithms $(\algo{Setup}, \algo{Commit})$ where:
  \begin{itemize}
    \item $\algo{Setup}(1^\lambda) \rightarrow \params$\footnote{The security parameter $1^\lambda$ is often left implicit in algorithm inputs to simplify notation, e.g., writing $\algo{Setup}()$ instead of $\algo{Setup}(1^\lambda)$.} is a probabilistic algorithm that takes the security parameter as input and outputs public parameters $\params$ including the message space $\mathcal{M}$ and the randomness space $\mathcal{R}$.
    \item $\algo{Commit}(\params, m; r) \rightarrow C$\footnote{Public parameters $\params$ are often left implicit in algorithm inputs to simplify notation, e.g., writing $\algo{Commit}(m; r)$ instead of $\algo{Commit}(\params, m; r)$.} is a deterministic algorithm that takes public parameters $\params$, a message $m \in \mathcal{M}$, and randomness $r \in \mathcal{R}$ as input and outputs a commitment $C$.
  \end{itemize}
\end{definition}

\begin{example}
  An example is the trivial commitment scheme\label{sec:trivial-com}:
  \begin{itemize}
    \item $\algo{Setup}(1^\lambda) \defeq (\mathcal{M} \defeq \{0, 1\}^*, \mathcal{R} \defeq \{\})$
    \item $\algo{Commit}(\params,m;r) \defeq m$
  \end{itemize}
\end{example}

\subsection{Security}

\begin{figure}[tbhp]
  \begin{center}
    \begin{tcolorbox}[width=8cm]
      \begin{pchstack}[center]
        \procedure[headlinesep=1pt]{$\game{\Game~\algo{ComBind}}{\algo{Com}}$}{%
            \params \sample \algo{Com.Setup}(1^\lambda) \\
            (m_0, m_1, r_0, r_1) \gets \adv(\params) \\
            C_0 \gets \algo{Com.Commit}(\params, m_0; r_0) \\
            C_1 \gets \algo{Com.Commit}(\params, m_1; r_1) \\
            \pcreturn (C_0 = C_1) \wedge (m_0 \neq m_1)
        }
      \end{pchstack}
    \end{tcolorbox}
  \end{center}
  \caption{Binding game \label{fig:break-com-bind}}
\end{figure}

\begin{definition}[Binding]
  A commitment scheme $\algo{Com}$ is binding if for any \ppt algorithm $\adv$,
 \[
  \advantage{Bind}{\adv, \algo{Com}} \defeq \pr{\game{\algo{ComBind}}{\algo{Com}} = 1} = \negl.
 \]
 If $\advantage{Bind}{\adv, \algo{Com}} = 0$, then the commitment scheme is said to be \emph{perfectly binding}.
\end{definition}

\begin{figure}[tbhp]
  \begin{center}
    \begin{tcolorbox}[width=8cm]
      \begin{pchstack}[center]
        \procedure[headlinesep=1pt]{$\game{\Game~\algo{ComHide}}{\algo{Com}}$}{%
            \params \sample \algo{Com}.\algo{Setup}(1^\lambda) \\
            (m_0, m_1) \gets \adv(\params) \\
            r \sample \params.\mathcal{R} \\
            b \sample \{0, 1\} \\
            C \gets \algo{Com}.\algo{Commit}(\params, m_b; r) \\
            b' \gets \adv(\params, C) \\
            \pcreturn (b = b')
        }
      \end{pchstack}
    \end{tcolorbox}
  \end{center}
  \caption{Hiding game \label{fig:break-com-hid}}
\end{figure}

\begin{definition}[Hiding]
  A commitment scheme $\algo{Com}$ is hiding if for any \ppt algorithm $\adv$,
 \[
  \advantage{Hide}{\adv, \algo{Com}} \defeq |\pr{\game{\algo{ComHide}}{\algo{Com}} = 1} - \frac{1}{2}| = \negl.
 \]
 If $\advantage{Hide}{\adv, \algo{Com}} = 0$, then the commitment scheme is said to be \emph{perfectly hiding}.
\end{definition}

Note that in $\game{\algo{ComHide}}{\algo{Com}}$ the adversary $\adv$ is run twice with different arguments, first to output two messages and then to output a bit.

\subsection{Exercises}

\begin{exercise}
  Is the trivial commitment scheme in \cref{sec:trivial-com} binding?
  Prove it.
\end{exercise}

\ifsolutions
\begin{mysolution}
  Yes, the trivial commitment scheme is perfectly binding.
  
  Proof by contradiction: Assume the binding game succeeds, i.e., $\game{\algo{ComBind}}{\algo{Com}} = 1$.
  This means:
  \begin{itemize}
    \item $C_0 = C_1$ (the commitments are equal)
    \item $m_0 \neq m_1$ (the messages are different)
  \end{itemize}
  
  However, by the definition of the trivial commitment scheme:
  \begin{itemize}
    \item $C_0 = \algo{Commit}(\params, m_0; r_0) = m_0$
    \item $C_1 = \algo{Commit}(\params, m_1; r_1) = m_1$
  \end{itemize}
  
  Therefore, $C_0 = C_1$ implies $m_0 = m_1$, which contradicts $m_0 \neq m_1$.
  
  Since no adversary can win the binding game, we have $\advantage{Bind}{\adv, \algo{Com}} = 0$ for all adversaries $\adv$.
  The trivial commitment scheme is therefore perfectly binding.
\end{mysolution}
\fi

\begin{exercise}
  A commitment scheme is \emph{strongly binding} if a commitment cannot be opened with two different randomness values.
  Formally define the strong binding game $\game{\algo{ComStrongBind}}{\algo{Com}}$ by modifying $\game{\algo{ComBind}}{\algo{Com}}$.
\end{exercise}

\ifsolutions
\begin{mysolution}
  The strong binding game $\game{\algo{ComStrongBind}}{\algo{Com}}$ is defined as follows:
  
  \begin{center}
    \begin{tcolorbox}[width=8cm]
      \begin{pchstack}[center]
        \procedure[headlinesep=1pt]{$\game{\Game~\algo{ComStrongBind}}{\algo{Com}}$}{%
            \params \sample \algo{Com.Setup}(1^\lambda) \\
            (m_0, m_1, r_0, r_1) \gets \adv(\params) \\
            C_0 \gets \algo{Com.Commit}(\params, m_0; r_0) \\
            C_1 \gets \algo{Com.Commit}(\params, m_1; r_1) \\
            \pcreturn (C_0 = C_1) \wedge ((m_0 \neq m_1) \vee (r_0 \neq r_1))
        }
      \end{pchstack}
    \end{tcolorbox}
  \end{center}
  
  The key difference from the regular binding game is in the return condition:
  \begin{itemize}
    \item Regular binding: $(C_0 = C_1) \wedge (m_0 \neq m_1)$
    \item Strong binding: $(C_0 = C_1) \wedge ((m_0 \neq m_1) \vee (r_0 \neq r_1))$
  \end{itemize}
  
  Strong binding prevents an adversary from finding the same commitment with either different messages OR different randomness.
\end{mysolution}
\fi

\begin{exercise}[Optional]
  Describe a scenario where strong binding is important.
\end{exercise}

\ifsolutions
\begin{mysolution}
  Strong binding is crucial when the randomness itself has semantic meaning in the protocol.

  Consider the Taproot commitment scheme where the randomness space $\mathcal{R}$ is a group $\GG$.
  For hash function $\algo{H}$ and group generator $g$, the commitment is computed as:
  \[
    \algo{Commit}(m; R) = R \cdot g^{\algo{H.Eval}(\kappa, (R, m))}
  \]

  In this scheme, $m$ represents the root of a Merkle tree containing Tapscripts, and $R$ serves as an internal key.
  Assume that the Tapscript language includes an opcode OP\_INTERNALKEY that pushes $R$ onto the execution stack.

  Without strong binding, an adversary could potentially find an alternative opening $(R', m)$ where $R' \neq R$ but still produces the same commitment $C$.
  This would allow the adversary to execute scripts with OP\_INTERNALKEY pushing their chosen $R'$ onto the stack instead of the original $R$.

  Strong binding prevents this attack by ensuring that each commitment $C$ has a unique valid opening, making it impossible for an adversary to substitute a different internal key while maintaining the same commitment.
\end{mysolution}
\fi

\begin{exercise}
  Design a hash-based commitment scheme $\algo{HashCom}[\algo{H}]$\footnote{The notation $\algo{Scheme}[\algo{Primitive}]$ indicates that the scheme is parameterized by a cryptographic primitive.} that is strongly binding when used with a collision-resistant hash function $\algo{H}$.
  State and prove the security theorem.
  
  Note: We do not yet have the tools to prove that $\algo{HashCom}[\algo{H}]$ is hiding; we will return to this in later chapters.
\end{exercise}

\ifsolutions
\begin{mysolution}
  Define the hash commitment scheme $\algo{HashCom}[\algo{H}]$ as follows:
  \begin{itemize}
    \item $\algo{Setup}(1^\lambda) \defeq (\mathcal{M} \defeq \{0, 1\}^*, \mathcal{R} \defeq \{0, 1\}^\lambda, \kappa \gets \algo{H.Gen}(1^\lambda))$
    \item $\algo{Commit}(\params, m; r) \defeq \algo{H.Eval}(\kappa, r \| m)$
  \end{itemize}
  
  \textbf{Theorem:} Let $\algo{H}$ be a collision-resistant hash function. Then $\algo{HashCom}[\algo{H}]$ is strongly binding.
  More precisely, for any \ppt adversary $\adv$ against the strong binding property of $\algo{HashCom}[\algo{H}]$, there exists a \ppt adversary $\bdv$ against the collision resistance of $\algo{H}$ such that
  \[
  \advantage{StrongBind}{\adv, \algo{HashCom}[\algo{H}]} = \advantage{Coll}{\bdv^\adv_\algo{H}, \algo{H}}.
  \]
  
  \textbf{Proof:} We prove the contrapositive: if $\algo{HashCom}[\algo{H}]$ is not strongly binding, then $\algo{H}$ is not collision-resistant.
  
  Assume $\algo{HashCom}[\algo{H}]$ is not strongly binding.
  Then there exists a \ppt adversary $\adv$ that wins $\game{\algo{ComStrongBind}}{\algo{HashCom}[\algo{H}]}$ with non-negligible probability, finding $(m_0, r_0)$ and $(m_1, r_1)$ such that:
  \begin{itemize}
    \item $\algo{H.Eval}(\kappa, r_0 \| m_0) = \algo{H.Eval}(\kappa, r_1 \| m_1)$
    \item Either $m_0 \neq m_1$ or $r_0 \neq r_1$
  \end{itemize}
  
  If either $m_0 \neq m_1$ or $r_0 \neq r_1$, then $r_0 \| m_0 \neq r_1 \| m_1$.
  This means $\adv$ has found a collision in $\algo{H}$: two different inputs that hash to the same output.
  
  We can construct a \ppt adversary $\bdv$ against the collision resistance of $\algo{H}$ that runs $\adv$ and outputs $(r_0 \| m_0, r_1 \| m_1)$.
  Since $\adv$ succeeds with non-negligible probability and $\bdv$ is \ppt (as it simply runs $\adv$ once), $\bdv$ breaks collision resistance.
  
  Therefore, if $\algo{H}$ is collision-resistant, then $\algo{HashCom}[\algo{H}]$ is strongly binding.
  Note that strong binding implies regular binding, so this scheme is also binding.
\end{mysolution}
\fi

\begin{exercise}
  Is the trivial commitment scheme in \cref{sec:trivial-com} hiding?
  Prove it.
\end{exercise}

\ifsolutions
\begin{mysolution}
  No, the trivial commitment scheme is not hiding.

  We construct a \ppt adversary $\adv$ that wins the hiding game with probability 1:

  \begin{enumerate}
    \item $\adv$ receives $\params$ from the setup.
    \item $\adv$ outputs two distinct messages: $m_0 = 0$ and $m_1 = 1$.
    \item The game samples $b \sample \{0, 1\}$ and computes $C = \algo{Commit}(\params, m_b; r) = m_b$.
    \item $\adv$ receives $C$ and outputs $b' = C$.
    \item Since $C = m_b = b$ (because $m_0 = 0$ and $m_1 = 1$), we have $b' = b$.
  \end{enumerate}

  Therefore, $\pr{\game{\algo{ComHide}}{\algo{Com}} = 1} = 1$, and 
  \[
    \advantage{Hide}{\adv, \algo{Com}} = |1 - \frac{1}{2}| = \frac{1}{2}
  \]
  which is not negligible.

  Intuitively, the trivial commitment scheme completely reveals the message, providing no hiding at all.
\end{mysolution}
\fi

\begin{exercise}
  Let $\algo{H}$ be a collision-resistant hash function and define the commitment scheme $\algo{HashCom'}[\algo{H}]$ as:
  \begin{itemize}
    \item $\algo{Setup}(1^\lambda) \defeq (\mathcal{M} \defeq \{0, 1\}, \mathcal{R} \defeq \{\}, \kappa \gets \algo{H.Gen}(1^\lambda))$
    \item $\algo{Commit}(\params,m;r) \defeq  \algo{H.Eval}(\kappa, m)$
  \end{itemize}
  Note that $\algo{HashCom'}[\algo{H}]$ is binding.
  Is it hiding?
  Prove it.
\end{exercise}

\ifsolutions
\begin{mysolution}
  No, $\algo{HashCom'}[\algo{H}]$ is not hiding.
  The issue is that there's no randomness in the commitment.
  
  We construct a \ppt adversary $\adv$ that wins the hiding game with high probability:
  
  \begin{enumerate}
    \item $\adv$ receives $\params$ containing $\kappa$.
    \item $\adv$ outputs $m_0 = 0$ and $m_1 = 1$.
    \item The game samples $b \sample \{0, 1\}$ and computes $C = \algo{H.Eval}(\kappa, m_b)$.
    \item $\adv$ receives $C$ and computes:
      \begin{itemize}
        \item $h_0 = \algo{H.Eval}(\kappa, 0)$
        \item $h_1 = \algo{H.Eval}(\kappa, 1)$
      \end{itemize}
    \item If $C = h_0$, output $b' = 0$; if $C = h_1$, output $b' = 1$.
  \end{enumerate}
  
  Since $\algo{H}$ is collision-resistant, with overwhelming probability $\algo{H.Eval}(\kappa, 0) \neq \algo{H.Eval}(\kappa, 1)$.
  Therefore, $\adv$ correctly identifies $b$ with overwhelming probability, making the scheme not hiding.
\end{mysolution}
\fi

