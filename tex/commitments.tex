\section{Commitments}\label{sec:commitments}


A commitment scheme consists of algorithm $\mathsf{Com}_\mathsf{Commit}$:
\begin{itemize}
    \item $\mathsf{Com}_\mathsf{Commit}(m;r)\rightarrow C$ outputs a commitment $C$ to message $m$ with randomness $r$.
\end{itemize}

\subsection{Syntax}
A commitment scheme $\mathsf{Com}$ is a tuple of algorithms:

\begin{itemize}
  \item $\mathsf{Com}_\mathsf{Setup}(1^\lambda) \rightarrow \mathsf{pp}$
        On input the security parameter $\lambda$ (in unary) output public
        parameters~$\mathsf{pp}$.

  \item $\mathsf{Com}_\mathsf{Commit}(\mathsf{pp},m;r)\rightarrow C$
        On message $m$ and randomness $r$ output a commitment $C$ to $m$.
\end{itemize}

% Why does this need a setup algorithm
% What are the sets, the types?

\paragraph{Example} An example is the $\mathsf{Trivial}$ commitment scheme:
\begin{itemize}
  \item $\mathsf{Trivial}.\mathsf{Com}_\mathsf{Setup}(1^\lambda) = ()$
  \item $\mathsf{Trivial}.\mathsf{Com}_\mathsf{Commit}(\mathsf{pp},m;r) = m$
\end{itemize}

\subsection{Security}
We define binding security through a game between a challenger and an adversary.
A commitment scheme is binding if no p.p.t adversary can win the following game with non-negligible probability:

\begin{figure}[tbhp]
  \begin{center}
    \begin{tcolorbox}[width=8cm]
      \begin{pchstack}[center]
        \procedure[headlinesep=1pt]{$\game{\Game~\algo{ComBind}}{\algo{Com}}$}{%
            \params \sample \mathsf{Com}_\mathsf{Setup}(1^\lambda) \\
            (m_0, m_1, r_0, r_1) \gets \adv(\params) \\
            C_0 \gets \mathsf{Com}_\mathsf{Commit}(\params, m_0; r_0) \\
            C_1 \gets \mathsf{Com}_\mathsf{Commit}(\params, m_1; r_1) \\
            \pcreturn (C_0 = C_1) \wedge (m_0 \neq m_1)
        }
      \end{pchstack}
    \end{tcolorbox}
  \end{center}
  \caption{Game for finding a binding attack on a commitment scheme \label{fig:break-com-bind}}
\end{figure}

% Why does the adversary get \params? Let say we have a Pedersen commitment scheme and the group is hardcoded in the commitment scheme.
% Then there exists an adversary that just has the collision hardcoded.

% TODO begin definition
More formally, a commitment scheme $\mathsf{Com}$ is binding if for all p.p.t. adversaries $\adv$, there exists a negligible function $\negl$ such that:
\[ \pr{\game{\algo{ComBind}}{\algo{Com}} = \pctrue} \leq \negl \]
If the advantage is 0, then the commitment scheme is perfectly binding.

% collision resistant hash function makes for a binding commitment

% What does this mean? We can make the probability as close to 0 as we want by increasing the security parameter.
% Note that we choose the pp and therefore the group in an pedersen commitment randomly.
% Hence the adversery can't just hardcode the collision. But how many groups are there of a certain order? Or do we mean with n-bit group that the prime has at least n bits, but we can also select a choice of larger groups?



%Informal
%\paragraph{Security} A commitment scheme is \emph{binding} if no PPT adversary produces two distinct messages $m_0, m_1$ and randomness $r_0, r_1$ such that $\com(m_0;r_0) = \com(m_1;r_1)$  (except with probability negligible in $\secpar$).
%A commitment scheme is \emph{hiding} if no PPT adversary obtains any information about the message from the commitment.

% define hiding
% show that hash function is not hiding in the standard model and motivate the ROM in that way??

\begin{figure}[tbhp]
  \begin{center}
    \begin{tcolorbox}[width=8cm]
      \begin{pchstack}[center]
        \procedure[headlinesep=1pt]{$\game{\Game~\algo{ComHid}}{\algo{Com}, b}$}{%
            \params \sample \mathsf{Com}_\mathsf{Setup}(1^\lambda) \\
            (m_0, m_1) \gets \adv(\params) \\
            r \sample \params.\mathcal{R} \\
            b \sample \{0, 1\} \\
            C \gets \mathsf{Com}_\mathsf{Commit}(\params, m_b; r) \\
            b' \gets \adv(\params, C) \\
            \pcreturn (b = b')
        }
      \end{pchstack}
    \end{tcolorbox}
  \end{center}
  \caption{Game for finding a hiding attack on a commitment scheme \label{fig:break-com-hid}}
\end{figure}

A commitment scheme $\mathsf{Com}$ is hiding if for all p.p.t. adversaries $\adv$, there exists a negligible function $\negl$ such that:
\[ \left| \pr{\game{\algo{ComHid}}{\algo{Com}, b} = \pctrue} - \frac{1}{2} \right| \leq \negl \]
If the advantage is 0, then the commitment scheme is perfectly hiding.
