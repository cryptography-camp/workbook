\section{Discrete Logarithm}\label{sec:discrete-log}

\subsection{The Discrete Logarithm Problem}

\begin{definition}[Group Generation Algorithm]
  A \emph{group generation algorithm} $\grgen$ is a \ppt algorithm that takes the security parameter $1^\secpar$ as input and outputs a group description $(\GG, p, g)$, where $\GG$ is a cyclic group of prime order $p$ with $p \geq 2^{\secpar-1}$, and $g$ is a generator of $\GG$.
\end{definition}

\begin{remark}
  The group operation is denoted multiplicatively ($g\cdot g = g^2$).
\end{remark}


\begin{figure}[tbhp]
  \begin{center}
    \begin{tcolorbox}[width=5cm]
      \begin{pchstack}[center]
        \procedure[headlinesep=1pt]{$\game{\Game~\algo{DL}}{\grgen}$}{%
          \gparam \gets \grgen(1^\secpar) \\
          x \sample \ZZ_p \\
          X \defeq g^x \\
          x' \gets \adv(\gparam, X) \\
          \pcreturn x' = x
        }
      \end{pchstack}
    \end{tcolorbox}
  \end{center}
  \caption{The discrete logarithm game\label{fig:dl-game}}
\end{figure}

\begin{definition}[Discrete Logarithm Assumption]
  Let $\grgen$ be a group generation algorithm, and let $\game{\Game~\algo{DL}}{\grgen}$ be as defined in \autoref{fig:dl-game}.
  The discrete logarithm (DL) problem is hard for $\grgen$ if for any \ppt algorithm $\adv$,
  \[
  \advantage{DL}{\adv, \grgen} \defeq \pr{\game{\Game~\algo{DL}}{\grgen} = \pctrue} = \negl.
  \]
\end{definition}

\subsection{Pedersen Commitments}

\begin{definition}[Pedersen Commitment Scheme]
  The Pedersen commitment scheme\\ $\algo{PedCom}[\grgen]$ is parameterized by a group generation algorithm $\grgen$ and defined as follows:
  \begin{itemize}
  \item $\algo{Setup}(1^\secpar) \rightarrow \params$: Run $\gparam \gets \grgen(1^\secpar)$, sample $h \sample \GG \setminus \{1_\GG\}$, and output $\params = (\gparam, h, \mathcal{M} \defeq \ZZ_p, \mathcal{R} \defeq \ZZ_p)$.
  \item $\algo{Commit}(\params, m; r) \rightarrow C$: Parse $\params = ((\GG, p, g), h, \mathcal{M}, \mathcal{R})$. For message $m \in \mathcal{M}$ and randomness $r \in \mathcal{R}$, output $C = g^m h^r$.
  \end{itemize}
\end{definition}

\begin{remark}
  Pedersen commitments are \emph{homomorphic}: for commitments $C_1 = \algo{Commit}(\params, m_1; r_1)$ and $C_2 = \algo{Commit}(\params, m_2; r_2)$, we have
  \[
  C_1 \cdot C_2 = g^{m_1 + m_2} h^{r_1 + r_2} = \algo{Commit}(\params, m_1 + m_2; r_1 + r_2).
  \]
\end{remark}

\begin{theorem}[Pedersen Commitments are Binding]\label{thm:pedersen-binding}
  Let $\grgen$ be a group generation algorithm for which the discrete logarithm problem is hard, then the Pedersen commitment scheme $\algo{PedCom}[\grgen]$ is binding.
  More precisely, for any \ppt adversary $\adv$ against the binding property of $\algo{PedCom}[\grgen]$, there exists a \ppt adversary $\bdv$ against the discrete logarithm problem such that
  \[
  \advantage{Bind}{\adv, \algo{PedCom}} = \advantage{DL}{\bdv, \grgen}.
  \]
\end{theorem}

The proof is left as an exercise (see \autoref{ex:pedersen-binding}).

\begin{theorem}[Pedersen Commitments are Hiding]\label{thm:pedersen-hiding}
  The Pedersen commitment scheme $\algo{PedCom}[\grgen]$ is perfectly hiding.
  More precisely, for any (possibly unbounded) adversary $\adv$,
  \[
  \advantage{Hid}{\adv, \algo{PedCom}} = 0.
  \]
\end{theorem}

The proof is left as an exercise (see \autoref{ex:pedersen-hiding}).

\subsection{The DDH Assumption}

\begin{figure}[tbhp]
  \begin{center}
    \begin{tcolorbox}[width=6cm]
      \begin{pchstack}[center]
        \procedure[headlinesep=1pt]{$\game{\Game~\algo{DDH}}{\grgen}$}{%
          \gparam \gets \grgen(1^\secpar) \\
          a, b \sample \ZZ_p \\
          A \defeq g^a, B \defeq g^b \\
          d \sample \{0, 1\} \\
          \pcif d = 0 \pcthen \\
          \t c \sample \ZZ_p \\
          \t C \defeq g^c \\
          \pcelse \\
          \t C \defeq g^{ab} \\
          d' \gets \adv(\gparam, A, B, C) \\
          \pcreturn d' = d
        }
      \end{pchstack}
    \end{tcolorbox}
  \end{center}
  \caption{The decisional Diffie-Hellman game\label{fig:ddh-game}}
\end{figure}

\begin{definition}[Decisional Diffie-Hellman Assumption]
  Let $\grgen$ be a group generation algorithm, and let $\game{\Game~\algo{DDH}}{\grgen}$ be as defined in \autoref{fig:ddh-game}.
  The decisional Diffie-Hellman (DDH) problem is hard for $\grgen$ if for any \ppt algorithm $\adv$,
  \[
  \advantage{DDH}{\adv, \grgen} \defeq \left|\pr{\game{\Game~\algo{DDH}}{\grgen} = \pctrue} - \frac{1}{2}\right| = \negl.
  \]
\end{definition}

\subsection{ElGamal Commitments}

\begin{definition}[ElGamal Commitment Scheme]
  The ElGamal commitment scheme\\ $\algo{ElGamalCom}[\grgen]$ is parameterized by a group generation algorithm $\grgen$ and defined as follows:
  \begin{itemize}
  \item $\algo{Setup}(1^\secpar) \rightarrow \params$: Run $\gparam \gets \grgen(1^\secpar)$, sample $h \sample \GG \setminus \{1_\GG\}$, and output $\params = (\gparam, h, \mathcal{M} \defeq \ZZ_p, \mathcal{R} \defeq \ZZ_p)$.
  \item $\algo{Commit}(\params, m; r) \rightarrow C$: Parse $\params = ((\GG, p, g), h, \mathcal{M}, \mathcal{R})$. For message $m \in \mathcal{M}$ and randomness $r \in \mathcal{R}$, output $C = (g^r, h^{m+r})$.
  \end{itemize}
\end{definition}

\begin{theorem}[ElGamal Commitments are Hiding]\label{thm:elgamal-hiding}
  Let $\grgen$ be a group generation algorithm for which the DDH problem is hard, then the ElGamal commitment scheme $\algo{ElGamalCom}[\grgen]$ is hiding.
  More precisely, for any \ppt adversary $\adv$ against the hiding property of $\algo{ElGamalCom}[\grgen]$, there exists a \ppt adversary $\bdv$ against DDH such that
  \[
  \advantage{Hide}{\adv, \algo{ElGamalCom}} = 2 \cdot \advantage{DDH}{\bdv, \grgen}.
  \]
\end{theorem}

The proof is left as an exercise (see \autoref{ex:elgamal-hiding}).

\subsection{The OMDL and AOMDL Problems}

\begin{figure}[tbhp]
  \begin{center}
    \begin{tcolorbox}[width=\textwidth]
      \begin{pchstack}[center]
        \begin{pcvstack}
          \procedure[headlinesep=1pt]{$\game{\Game~\algo{OMDL}}{\grgen}$}{%
            \gparam \gets \grgen(1^\secpar) \\
            \ell \defeq 0 \pcsc q \defeq 0 \\
            \vect{y} \gets \adv^{\pcoracle{Chal}, \pcoracle{DL}}(\gparam) \\
            \vect{x} \defeq (x_1, \ldots, x_\ell) \\
            \pcreturn (\vect{y} = \vect{x} \wedge q < \ell)
          }
          \pcvspace
          \procedure[headlinesep=1pt]{Oracle $\pcoracle{DL}(X)$}{%
            q \defeq q+1 \\
            \pcreturn \log_g(X)
          }
        \end{pcvstack}
        \pchspace
        \begin{pcvstack}
          \procedure[headlinesep=1pt]{Oracle $\pcoracle{Chal}()$}{%
            \ell \defeq \ell + 1 \\
            x_\ell \sample \ZZ_p \\
            X_\ell \defeq g^{x_\ell} \\
            \pcreturn X_\ell
          }
        \end{pcvstack}
        \pchspace
        \begin{pcvstack}
          \procedure[headlinesep=1pt]{$\game{\Game~\algo{AOMDL}}{\grgen}$}{%
            \gparam \gets \grgen(1^\secpar) \\
            \ell \defeq 0 \pcsc q \defeq 0 \\
            \vect{y} \gets \adv^{\pcoracle{Chal}, \gamechange{$\oracle{ADL}$}}(\gparam) \\
            \vect{x} \defeq (x_1, \ldots, x_\ell) \\
            \pcreturn (\vect{y} = \vect{x} \wedge q < \ell)
          }
          \pcvspace
          \procedure[headlinesep=1pt]{Oracle $\oracle{ADL}((\alpha, \beta_1, \dots, \beta_{\ell}))$}{%
            q \defeq q+1 \\
            \pcreturn \alpha + \tsum_{i=1}^{\ell} {\beta_i x_i} \\
            \pclinecomment{${} = \log_g\left(g^\alpha \tprod_{i=1}^{\ell} {X_i^{\beta_i}}\right)$}
          }
        \end{pcvstack}
      \end{pchstack}
    \end{tcolorbox}
  \end{center}
  \caption{The one-more discrete logarithm (OMDL) and algebraic one-more discrete logarithm (AOMDL) games. The difference is \gamechange{highlighted}.\label{fig:omdl-aomdl-games}}
\end{figure}

\begin{definition}[One-More Discrete Logarithm Assumption]
  Let $\grgen$ be a group generation algorithm, and let $\game{\Game~\algo{OMDL}}{\grgen}$ be as defined in \autoref{fig:omdl-aomdl-games}.
  The one-more discrete logarithm (OMDL) problem is hard for $\grgen$ if for any \ppt algorithm $\adv$,
  \[
  \advantage{OMDL}{\adv, \grgen} \defeq \pr{\game{\Game~\algo{OMDL}}{\grgen} = \pctrue} = \negl.
  \]
\end{definition}

\begin{definition}[Algebraic One-More Discrete Logarithm Assumption]
  Let $\grgen$ be a group generation algorithm, and let $\game{\Game~\algo{AOMDL}}{\grgen}$ be as defined in \autoref{fig:omdl-aomdl-games}.
  The algebraic one-more discrete logarithm (AOMDL) problem is hard for $\grgen$ if for any \ppt algorithm $\adv$,
  \[
  \advantage{AOMDL}{\adv, \grgen} \defeq \pr{\game{\Game~\algo{AOMDL}}{\grgen} = \pctrue} = \negl.
  \]
\end{definition}

\begin{remark}
  When proving the security of a scheme using the AOMDL assumption, the $\pcoracle{ADL}$ oracle is (typically) used by the reduction and not by the adversary against the scheme.
  Therefore, the security proof is not restricted to adversaries that know the algebraic representations of all group elements that they output.
\end{remark}

\begin{remark}[Falsifiable Assumptions]
  In order to be able to evaluate whether an assumption is true or not, it must be falsifiable, i.e., there must be a (constructive) way to demonstrate that it is false, if this is the case.
  More precisely, a cryptographic assumption is \emph{falsifiable} if there exists a \ppt challenger algorithm that interacts with an adversary and decides whether the adversary breaks the assumption.
\end{remark}

\subsection{Exercises}

\begin{exercise}
  In the framework of asymptotic security, why is the DL problem on secp256k1 not hard? What does $\grgen$ do?
\end{exercise}

\ifsolutions
\begin{mysolution}
  The DL problem on secp256k1 is not hard in the asymptotic security framework because secp256k1 has a fixed 256-bit group order, independent of the security parameter $\secpar$.
  For asymptotic hardness, the group size must grow with $\secpar$.
  The group generation algorithm $\grgen$ generates groups where $p \geq 2^{\secpar-1}$, ensuring that the problem scales appropriately with the security parameter.
\end{mysolution}
\fi

\begin{exercise}\label{ex:pedersen-binding}
  Argue that Pedersen commitments are binding under the DL assumption (\autoref{thm:pedersen-binding}).
\end{exercise}

\ifsolutions
\begin{mysolution}
  Let $\adv$ be an adversary that breaks the binding property of Pedersen commitments.
  That is, $\adv$ outputs $(m_0, r_0, m_1, r_1)$ such that $m_0 \neq m_1$ and $g^{m_0}h^{r_0} = g^{m_1}h^{r_1}$.
  We construct a \ppt reduction $\bdv$ that breaks DL as follows:
  \begin{itemize}
    \item $\bdv$ receives $(\gparam, h)$ where $h = g^z$ for unknown $z$.
    \item $\bdv$ runs $\adv$ with parameters $\params = (\gparam, h)$.
    \item When $\adv$ outputs $(m_0, r_0, m_1, r_1)$, we have $g^{m_0}h^{r_0} = g^{m_1}h^{r_1}$.
    \item This implies $g^{m_0 - m_1} = h^{r_1 - r_0}$, so $g^{m_0 - m_1} = g^{z(r_1 - r_0)}$.
    \item If $r_0 = r_1$, then $g^{m_0 - m_1} = 1$, which contradicts $m_0 \neq m_1$ (since $g$ generates $\GG$).
    \item Therefore $r_0 \neq r_1$, and we can compute $z = \frac{m_0 - m_1}{r_1 - r_0} \bmod p$.
    \item $\bdv$ outputs $z$.
  \end{itemize}
  The reduction is perfect: whenever $\adv$ succeeds in breaking binding, $\bdv$ succeeds in computing the discrete log.
  Therefore, $\advantage{DL}{\bdv, \grgen} = \advantage{Bind}{\adv, \algo{PedCom}}$.
\end{mysolution}
\fi

\begin{exercise}\label{ex:pedersen-hiding}
  Prove that Pedersen commitments are perfectly hiding (\autoref{thm:pedersen-hiding}).
\end{exercise}

\ifsolutions
\begin{mysolution}
  For any message $m \in \ZZ_p$, the commitment $C = g^m h^r$ is uniformly distributed over $\GG$ when $r$ is chosen uniformly from $\ZZ_p$.
  Since $h \in \GG \setminus \{1_\GG\}$ and $\GG$ has prime order $p$, $h$ generates $\GG$ (every non-identity element generates a prime-order group).
  Thus the map $r \mapsto h^r$ is a bijection from $\ZZ_p$ to $\GG$.
  For fixed $m$, the map $r \mapsto g^m h^r$ is also a bijection from $\ZZ_p$ to $\GG$.
  Therefore, for any two messages $m_0, m_1$, the distributions of $\algo{Commit}(\params, m_0; r)$ and $\algo{Commit}(\params, m_1; r)$ for uniform $r$ are both uniform over $\GG$.
\end{mysolution}
\fi

\begin{exercise}\label{ex:elgamal-hiding}
  Prove that ElGamal commitments are hiding under the DDH assumption (\autoref{thm:elgamal-hiding}).
\end{exercise}

\ifsolutions
\begin{mysolution}
   Let $\adv$ be an adversary against the hiding property of ElGamal.
  We construct a \ppt reduction $\bdv$ against DDH:
  \begin{itemize}
    \item $\bdv$ receives $(\gparam, A, B, C)$ where $A = g^a$, $B = g^b$, and either $C = g^{ab}$ or $C = g^c$ for random $c$.
    \item $\bdv$ sets $h = B$ and gives $\params = (\gparam, h)$ to $\adv$.
    \item When $\adv$ outputs $(m_0, m_1)$, $\bdv$ chooses $e \sample \{0, 1\}$ and computes:
          If $C = g^{ab}$, set implicitly $r = a$, so $(g^r, h^{m_e + r}) = (g^a, g^{b(m_e + a)}) = (A, g^{bm_e + ab}) = (A, B^{m_e} \cdot C)$.
          If $C = g^c$ for random $c$, then $(A, B^{m_e} \cdot C) = (g^a, g^{bm_e + c})$.
    \item $\bdv$ gives $(A, B^{m_e} \cdot C)$ to $\adv$.
    \item When $\adv$ outputs $e'$, $\bdv$ outputs $1$ if $e' = e$ and $0$ otherwise.
  \end{itemize}

  Analysis: We have:
  \begin{itemize}
      \item When $C = g^{ab}$ (real tuple): $\bdv$ perfectly simulates the hiding game, so $|\Pr[e = e' | C = g^{ab}] - \frac{1}{2}| = \advantage{Hide}{\adv, \algo{ElGamalCom}}$.
      \item When $C = g^c$ (random tuple): The commitment reveals no information about $e$, so $\Pr[e = e' | C = g^c] = \frac{1}{2}$.
  \end{itemize}
  For the DDH advantage:
  \begin{align*}
  \advantage{DDH}{\bdv, \grgen} &= |\Pr[d = d'] - \frac{1}{2}|\\
  \Pr[d = d'] &= \Pr[d' = 1 \wedge d = 1] + \Pr[d' = 0 \wedge d = 0]\\
  &= \Pr[d' = 1 \wedge C = g^{ab}] + \Pr[d' = 0 \wedge C = g^c]\\
  &= \Pr[d' = 1 | C = g^{ab}] \cdot \Pr[C = g^{ab}] + \Pr[d' = 0 | C = g^c] \cdot \Pr[C = g^c]\\
  &= \frac{1}{2} \Pr[d' = 1 | C = g^{ab}] + \frac{1}{2} \Pr[d' = 0 | C = g^c]\\
  &= \frac{1}{2} \Pr[d' = 1 | C = g^{ab}] + \frac{1}{2}(1 - \Pr[d' = 1 | C = g^c])\\
  &= \frac{1}{2} \Pr[d' = 1 | C = g^{ab}] + \frac{1}{2} - \frac{1}{2}\Pr[d' = 1 | C = g^c]
  \end{align*}
  Therefore:
  \begin{align*}
  \advantage{DDH}{\bdv, \grgen} &= \left|\frac{1}{2} \Pr[d' = 1 | C = g^{ab}] - \frac{1}{2}\Pr[d' = 1 | C = g^c]\right|\\
  &= \frac{1}{2} |\Pr[d' = 1 | C = g^{ab}] - \Pr[d' = 1 | C = g^c]|\\
  &= \frac{1}{2} |\Pr[e = e' | C = g^{ab}] - \Pr[e = e' | C = g^c]|\\
  &= \frac{1}{2} \left|\Pr[e = e' | C = g^{ab}] - \frac{1}{2}\right|\\
  &= \frac{1}{2} \cdot \advantage{Hide}{\adv, \algo{ElGamalCom}}
  \end{align*}

  This gives us: $\advantage{Hide}{\adv, \algo{ElGamalCom}} = 2 \cdot \advantage{DDH}{\bdv, \grgen}$. $\bdv$ is \ppt\ if $\adv$ is.
\end{mysolution}
\fi

\begin{exercise}
  Argue that if OMDL is hard for a group generation algorithm then DL is hard.
\end{exercise}

\ifsolutions
\begin{mysolution}
  If OMDL is hard, then DL must be hard.
  Given a \ppt DL adversary $\adv$, we construct a \ppt OMDL adversary $\bdv$:
  \begin{itemize}
    \item $\bdv$ calls $\pcoracle{Chal}()$ once to get $X_1$.
    \item $\bdv$ runs $\adv(\gparam, X_1)$ to get $x_1'$.
    \item $\bdv$ outputs $(x_1')$.
  \end{itemize}
  If $\adv$ succeeds, then $X_1 = g^{x_1'}$, so $\bdv$ solves one discrete log with zero challenge queries, winning the OMDL game.
\end{mysolution}
\fi

\begin{exercise}
  Consider an algorithm $\adv$ playing the AOMDL game. It queries $\pcoracle{Chal}$ twice to obtain $X_1$ and $X_2$, then samples $\alpha, \beta_1, \beta_2 \sample \ZZ_p$. 
  \begin{enumerate}
    \item How should $\adv$ call the $\oracle{ADL}$ oracle to obtain $\log_g(P)$ where $P = g^\alpha X_1^{\beta_1} X_2^{\beta_2}$?
    \item Assuming $\adv$ makes no more $\pcoracle{Chal}$ queries, how many more $\oracle{ADL}$ queries can it make?
    \item What must $\adv$ return to win the game?
  \end{enumerate}
\end{exercise}

\ifsolutions
\begin{mysolution}
  \begin{enumerate}
    \item $\adv$ should call $\oracle{ADL}((\alpha, \beta_1, \beta_2))$. The oracle will return $\alpha + \beta_1 x_1 + \beta_2 x_2 = \log_g(g^\alpha X_1^{\beta_1} X_2^{\beta_2}) = \log_g(P)$.
    \item Since $\ell = 2$ (two $\pcoracle{Chal}$ queries) and $q = 1$ (one $\oracle{ADL}$ query), and the winning condition requires $q < \ell$, the adversary can make at most 0 more queries (as $1 < 2$ but $2 \not< 2$).
    \item $\adv$ must return $(x_1, x_2)$, the discrete logarithms of $X_1$ and $X_2$.
  \end{enumerate}
\end{mysolution}
\fi

\begin{exercise}
  Is the OMDL assumption falsifiable? Is the AOMDL assumption falsifiable? Do we prefer the security of a scheme to be based on OMDL or AOMDL?
\end{exercise}

\ifsolutions
\begin{mysolution}
  The OMDL assumption is \emph{non-falsifiable} because verifying the adversary's output requires computing discrete logarithms to check that $g^{x_i} = X_i$, which cannot be done in polynomial time classically.

  The AOMDL assumption is \emph{falsifiable} because:
  \begin{itemize}
    \item The ADL oracle can be implemented efficiently: given $(\alpha, \beta_1, \ldots, \beta_\ell)$, it simply returns $\alpha + \sum_{i=1}^\ell \beta_i x_i$
    \item The winning condition can be verified by checking that the returned values equal the stored $x_i$ values
    \item All operations (additions, multiplications) are polynomial time
  \end{itemize}

  We prefer basing security on falsifiable assumptions like AOMDL. Note that AOMDL is a weaker assumption than OMDL---if AOMDL doesn't hold, then OMDL doesn't hold either (since any OMDL adversary is also an AOMDL adversary).
\end{mysolution}
\fi

\begin{exercise}[Optional]
  Why is the DL problem not hard in general?
\end{exercise}

\ifsolutions
\begin{mysolution}
  The DL problem is not hard in general because quantum computers can solve DL in polynomial time using Shor's algorithm.
\end{mysolution}
\fi

\begin{exercise}[Optional]
  Read about the Generic Group Model (GGM) in the introduction of~\cite{EC:Shoup97} and~\cite{IMA:Maurer05}, and argue why the DLP is hard for classical computational models.
\end{exercise}

\begin{exercise}[Optional]
  How does Silent Payments~\cite{add:bip-silentpayments} rely on the DDH assumption?
\end{exercise}

\begin{exercise}[Optional]
  Study the concept of falsifiability of cryptographic assumptions by reading Naor~\cite{C:Naor03}.
  Analyze the following assumptions and explain the extent to which each one is falsifiable:
  \begin{itemize}
    \item Factoring
    \item The RSA assumption
    \item The Decisional Diffie-Hellman assumption
    \item The Knowledge-of-Exponent assumption
  \end{itemize}
\end{exercise}
