\section{Interactive Aggregate Signatures}\label{sec:interactive-aggregate-signatures}

An interactive aggregate signature (IAS) scheme allows $n$ signers, each with their own key pair $(sk_i, pk_i)$ and message $m_i$, to jointly produce a single short signature that proves $m_i$ was signed under $pk_i$ for every $i \in \{1, \ldots, n\}$.

The following exercises are based on the DahLIAS paper~\cite{add:nick2025dahlias}, which presents a discrete logarithm-based IAS scheme with constant-size signatures.

\subsection{Exercises}

\begin{exercise}
  Read the abstract and the main security theorem (Theorem 2) of DahLIAS.
  What does the security theorem tell us about the relationship between breaking DahLIAS and solving the underlying computational problems?
\end{exercise}

\begin{exercise}
  Read the technical introduction (Section 1.3).
  Why does the straightforward adaptation of the MuSig2 two-round technique described in the text not yield a secure IAS scheme?
\end{exercise}

\begin{exercise}
  Read about IAS syntax and security (Sections 3.1 and 3.3).
  What is the difference between EUF-CMA and co-EUF-CMA security for IAS?
\end{exercise}

\begin{exercise}
  What are the applications where strong binding (Section 6) matters?
  Explain why this property is important in these contexts.
\end{exercise}

\begin{exercise}[Optional]
  Read the proof sketch (Section 5.2).
  Outline the main proof technique and explain how the security reduction works.
\end{exercise}
