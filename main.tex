\newif\iffull
\newif\ifnotes
\newif\iflabels
\newif\ifanonymous
\newif\ifspace
\newif\ifcameraready
\newif\ifbibtex
\newif\ifsolutions

%%% Settings
%% Full version of the paper
\fulltrue
%% Show notes and todos
\notestrue
%% Print labels in the margins
% \labelstrue
%% Hide authors
\anonymoustrue
%% Enable hacks to save space
% \spacetrue
%% Disable some tweaks that the publisher may not like
% \camerareadytrue
%% Use legacy BibTeX instead of Biber
% \bibtextrue
%% Include solutions in the output
\solutionstrue

\documentclass[
  %% Uncomment to enable letterpaper
  a4paper,
  orivec,oribibl]{llncs}
\unless\ifcameraready
    \usepackage{llncsstandalone}  % local package
    % This is equivalent to 3cm margins (and the page shifted 0.5cm up)
    % on a4paper but gives an identical textblock on letterpaper.
    \usepackage[twoside=false, textwidth=150mm, textheight=237mm, vmarginratio=5:7, footskip=13mm]{geometry}
\fi
\usepackage[save]{silence}
\usepackage[T1]{fontenc}
\usepackage{lmodern}

%%% Space savings.
%%% The evil stuff. Uncomment manually.
\ifspace
  %% Scale line spacing. This becomes apparent at ~0.985 and below.
  % \renewcommand\baselinestretch{0.99}
  %% Always use inline style for \sum, \prod, ...
  % \PassOptionsToPackage{nosumlimits}{amsmath}
\fi

%%% Misc early packages
\usepackage[english]{babel}
\usepackage{csquotes}
\usepackage[babel,final]{microtype}
\usepackage{xpunctuate}
\usepackage{dirtytalk}
\usepackage{pifont}  % Dingbats
\usepackage{tcolorbox}
\tcbset{colback=white, size=title, boxsep=2mm}

%%% Colors
\usepackage[dvipsnames,svgnames,x11names]{xcolor}
\definecolor{bitcoin-orange}{RGB}{246, 145, 29}

%%% Math
\usepackage{amsmath}  % mostly for \qedhere
\usepackage{fixcmex}
% amsthm's proof environment conflicts with that of llncs
\let\lncsproof\proof
\let\lncsendproof\endproof
\let\proof\relax
\let\endproof\relax
\usepackage{amsthm}
\usepackage{mathtools}  % for \coloneq

%%% Notes and Todos
\NewDocumentCommand\newuser{m m m}{%
  \expandafter\NewDocumentCommand\csname #1note\endcsname{s O{} +m}{
    % Ensure that todos after paragraph headings are properly displayed
    \quitvmode%
    \texorpdfstring{%
      \todo[color=#3, inline, caption={}, ##2]{\textbf{#2:} ##3}%
    }{(Note by #2: ##3)}%
  }}
\ifnotes
  \usepackage{todonotes}
\else
  \usepackage[disable]{todonotes}
\fi

\iflabels
  \usepackage[notref,notcite,color]{showkeys}
\fi

%%% Bibliography
\ifbibtex
  \PassOptionsToPackage{backend=bibtex}{biblatex}
\fi
\PassOptionsToPackage{
  date=year,
  maxnames=3,
  minnames=3,
}{biblatex}
\ifcameraready
  \usepackage[
  style=biblatex-lncs/lncs,
  ]{biblatex}
  % Disable shorthands, see https://github.com/plk/biblatex/issues/1427
  \DeclareFieldInputHandler{shorthand}{\def\NewValue{}}
\else
  \usepackage[
  style=alphabetic,
  maxbibnames=15,
  minbibnames=15,
  ]{biblatex}
\fi

%%% Printing ePrint identifiers
\newcommand\iacreprintname{Cryptology ePrint Archive}
%% Shorter alternative:
% \newcommand\iacreprintname{IACR ePrint}
% TODO Consider just "\textsc{IACR}" (like for URL and DOI).
\DeclareFieldFormat{eprint:iacr}{%
  \iacreprintname\addcolon\space
  \ifhyperref
    {\href{https://eprint.iacr.org/#1}{%
       \nolinkurl{#1}%
       \iffieldundef{eprintclass}
         {}
         {\addspace\texttt{\mkbibbrackets{\thefield{eprintclass}}}}}}
    {\nolinkurl{#1}%
     \iffieldundef{eprintclass}
       {}
       {\addspace\texttt{\mkbibbrackets{\thefield{eprintclass}}}}}}

\addbibresource{vendor/cryptobib/abbrev3.bib}
% "crypto.bib" will be treated specially: We use the biber-wrapper.sh
% script to extract the cited entries from crypto.bib into a temporary
% bib file and then call biber on it, to avoid that biber takes very
% long to process the huge crypto.bib. All of this should happen
% automatically and transparently if you use latexmk.
\addbibresource{vendor/cryptobib/crypto.bib}
\addbibresource{add.bib}

%%% Save space in references
\AtEveryBibitem{
  % See Section 2.2.2 in the BibLaTeX manual for the field types
  \unless\iffull
    \clearfield{doi}
  \fi
  \clearlist{location}
  \clearname{editor}
  \clearfield{pages}
  \ifentrytype{inproceedings}{
      \clearfield{year}
      \clearfield{volume}
      \clearlist{publisher}
      \clearfield{series}
    }{}
}

%%% hyperref
\unless\ifcameraready
  \PassOptionsToPackage{pdfusetitle}{hyperref}
\fi
\usepackage{hyperref}
\hypersetup{
  hypertexnames=false,
}
\unless\ifcameraready
  \hypersetup{
    colorlinks=true,
    allcolors=blue,
    bookmarksopen,
  }
\usepackage{bookmark}
\fi

%%% Cross-references
\usepackage[capitalize, nameinlink]{cleveref}
% Force "eq.", "line" and "item" to be lowercase.
% "eq.~" is a bit involved: We'll \crefformat to include the "~",
% but we don't want to abbreviate at the beginning of a sentence.
% All these definitions are based on "equation" examples in the
% cleveref manual.
\crefformat{equation}{#2eq.~(#1)#3}
\Crefformat{equation}{#2Equation~(#1)#3}
\crefmultiformat{equation}{eqs.~(#2#1#3)}{ and~(#2#1#3)}{, (#2#1#3)}{ and~(#2#1#3)}
\Crefmultiformat{Equation}{Equations~(#2#1#3)}{ and~(#2#1#3)}{, (#2#1#3)}{ and~(#2#1#3)}
\crefrangeformat{Equation}{eqs.~(#3#1#4) to~(#5#2#6)}
\crefrangeformat{Equation}{Equations~(#3#1#4) to~(#5#2#6)}
\Crefname{equation}{Equation}{Equations}  % Don't abbreviate at start of sentence
\crefname{line}{line}{lines}
\Crefname{line}{Line}{Lines}
\crefname{enumi}{item}{items}
\Crefname{enumi}{Item}{Items}

%%% Alternative for cross-references
% \usepackage{zref-clever}
% \zcsetup{cap}
% % Replacement for \namecref
% \newcommand{\zcnameref}[1]{\zcref*[noref,nocap]{#1}}
% \zcRefTypeSetup{assumption}{
%   Name-sg = Assumption ,
%   name-sg = assumption ,
%   Name-pl = Assumptions ,
%   name-pl = assumptions ,
% }
% \zcRefTypeSetup{equation}{
%   abbrev = true ,
% }

%%% Additional theorem-like environments
\newtheorem{assumption}{Assumption}
\Crefname{assumption}{Assumption}{Assumptions}

%%% Exercise environment with section-based numbering
% Reset exercise counter at each section
\makeatletter
\@addtoreset{exercise}{section}
\makeatother
% Redefine the exercise environment to use section numbering
\renewcommand{\theexercise}{\thesection.\arabic{exercise}}
% Configure cleveref for exercises
\Crefname{exercise}{Exercise}{Exercises}
\crefname{exercise}{exercise}{exercises}

%%% Section-based numbering for theorem-like environments
\makeatletter
% Reset all theorem-like counters at each section
\@addtoreset{theorem}{section}
\@addtoreset{lemma}{section}
\@addtoreset{proposition}{section}
\@addtoreset{corollary}{section}
\@addtoreset{definition}{section}
\@addtoreset{remark}{section}
\@addtoreset{assumption}{section}
% Redefine numbering to use section.number format
\renewcommand{\thetheorem}{\thesection.\arabic{theorem}}
\renewcommand{\thelemma}{\thesection.\arabic{lemma}}
\renewcommand{\theproposition}{\thesection.\arabic{proposition}}
\renewcommand{\thecorollary}{\thesection.\arabic{corollary}}
\renewcommand{\thedefinition}{\thesection.\arabic{definition}}
\renewcommand{\theremark}{\thesection.\arabic{remark}}
\renewcommand{\theassumption}{\thesection.\arabic{assumption}}
\makeatother

%%% Configure cleveref for theorem-like environments
\Crefname{theorem}{Theorem}{Theorems}
\crefname{theorem}{theorem}{theorems}
\Crefname{lemma}{Lemma}{Lemmas}
\crefname{lemma}{lemma}{lemmas}
\Crefname{proposition}{Proposition}{Propositions}
\crefname{proposition}{proposition}{propositions}
\Crefname{corollary}{Corollary}{Corollaries}
\crefname{corollary}{corollary}{corollaries}
\Crefname{definition}{Definition}{Definitions}
\crefname{definition}{definition}{definitions}
\Crefname{remark}{Remark}{Remarks}
\crefname{remark}{remark}{remarks}

%%% Section-based numbering for figures and tables
\makeatletter
% Reset figure and table counters at each section
\@addtoreset{figure}{section}
\@addtoreset{table}{section}
% Redefine numbering to use section.number format
\renewcommand{\thefigure}{\thesection.\arabic{figure}}
\renewcommand{\thetable}{\thesection.\arabic{table}}
\makeatother

%%% Configure cleveref for figures and tables (usually already configured by cleveref)
\Crefname{figure}{Figure}{Figures}
\crefname{figure}{figure}{figures}
\Crefname{table}{Table}{Tables}
\crefname{table}{table}{tables}

%%% Solution handling
\newenvironment{mysolution}{\begin{proof}[Solution]}{\end{proof}}

%%% Cryptocode
\WarningFilter*[theH]{latex}{Command `\string\the}%
\ActivateWarningFilters[theH]
\usepackage[
  lambda,
  advantage,
  operators,
  adversary,
  landau,
  probability,
  sets,
  % notions,
  % logic,
  % ff,
  % mm,
  % primitives,
  % oracles,
  events,
  % complexity,
  asymptotics,
  keys,
]{cryptocode}
\DeactivateWarningFilters[theH]

% Better version of \pcfixcleveref:
% (see https://github.com/arnomi/cryptocode/pull/15)
\IfPackageLoadedTF{cleveref}{
  \AtBeginDocument{
    \pcfixhyperref%
    \makeatletter
    \crefalias{@pclinenumber}{line}%
    \makeatother
    % TODO Consider further improvements for the interaction between
    % crytocode and cleveref, e.g., referencing games, see: https://tex.stackexchange.com/questions/623163/cryptocode-cref-should-link-to-a-game-and-write-its-name/623196#623196
  }
}{}

%%% Append "." to subsubsection and paragraph headings
\let\llncssubsubsection\subsubsection
\renewcommand{\subsubsection}[1]{\llncssubsubsection{#1.}}
\let\llncsparagraph\paragraph
\renewcommand{\paragraph}[1]{\llncsparagraph{#1.}}


%%% We typically use \NewDocumentCommand instead of \newcommand(*) in
%%% order to obtain robust commands, which is usual in modern LaTeX.
%%% In many cases below, it would be okay to use \newcommand, e.g., as
%%% the underlying commands are robust but we mostly stick to
%%% \NewDocumentCommand for consistency.
%%%
%%% See https://latex3.github.io/help/documentation/usrguide.pdf for
%%% background.

%%% We use := as assignment operator
\NewDocumentCommand\defeq{}{\coloneq}
\NewCommandCopy\origgets\gets
\RenewCommandCopy\gets\defeq

%%% cryptocode tweaks
% Multi-letter variables, e.g., "ctr"
\NewDocumentCommand\var{m}{\ensuremath{\mathit{#1}}}
% Make keys and polynomials look like other identifiers.
% (No idea why cryptocode treats them specially.)
\RenewDocumentCommand\pckeystyle{m}{\ensuremath{\var{#1}}}
\RenewDocumentCommand\pcpolynomialstyle{m}{\ensuremath{\var{#1}}}
% Add 1pt padding to \gamechange (based on original definition).
\renewcommand{\gamechange}[2][gamechangecolor]{%
  {\setlength{\fboxsep}{1pt}%
    \colorbox{#1}{#2}}}
% Shorthand for \pcalgostyle
\NewDocumentCommand\algo{}{\pcalgostyle}
% Oracles
\NewDocumentCommand{\mathsc}{m}{{\normalfont\textsc{#1}}}
\RenewDocumentCommand\pcoraclestyle{m}{\ensuremath{\mathsc{#1}}}
\NewDocumentCommand\oracle{m}{\pcoraclestyle{#1}}
% Games
\newcommandx{\game}[4][3=\adv,4=(\secpar)]{{\operatorname{#1}_{#2}^{#3}#4}}
\newcommand{\Game}{\algo{Game}}
% Misc
\newcommand{\pcsc}{\,;~}

%%% Algorithms and named schemes
%%%
%%% Uncomment these manually as needed.
% \NewDocumentCommand\grgen{}{\algo{GrGen}}
% \NewDocumentCommand\gparam{}{(\GG,p,g)}

% \NewDocumentCommand{\inpgen}{\algo{InpGen}}
% \NewDocumentCommand{\param}{\var{par}} % public parameters of the multisig scheme

% \NewDocumentCommand\setup{\algo{Setup}}
% \NewDocumentCommand\keygen{\algo{KeyGen}}
% \NewDocumentCommand\sign{\algo{Sign}}
% \NewDocumentCommand\verify{\algo{Verify}}

% \NewDocumentCommand\musigtwo{\algo{MuSig2}}
% \NewDocumentCommand\musigtwoias{\algo{MuSig2\textnormal{-}IAS}}

%% Game Oracles
% \NewDocumentCommand{\dlo}{}{\oracle{DL}}

%% Random oracles
% \NewDocumentCommand{\hash}{\algo{H}}
% \NewDocumentCommand{\Hsig}{\algo{H}_{\mathrm{sig}}}

%%% For easier typing
\NewDocumentCommand{\eg}{}{e.g\xperiod}
\NewDocumentCommand{\ie}{}{i.e\xperiod}
% Caution: \wlog is a TeX builtin that writes to the log.
\NewDocumentCommand{\wlg}{}{w.l.o.g\xperiod}

%%% Dingbats for checkmark and cross, e.g., for feature columns in tables
\NewDocumentCommand\cmark{}{\ding{51}}
\NewDocumentCommand\xmark{}{\ding{55}}

%%% Bitcoin symbol
% https://tex.stackexchange.com/a/112165, slightly improved
\NewDocumentCommand\btcsym{}{%
  \leavevmode
  \vtop{\offinterlineskip\upshape%\bfseries
    \setbox0=\hbox{B}%
    \setbox2=\hbox to\wd0{\hfil\hskip-.03em
    \vrule height .3ex width .15ex\hskip .08em
    \vrule height .3ex width .15ex\hfil}
    \vbox{\copy2\vskip-.01ex\box0}\vskip-.01ex\box2}}

\newcommand{\pr}[1]{\Pr\left[ #1 \right]}
\DeclareMathOperator{\Ima}{Im}
\newcommand{\params}{\var{pp}}




%%% Title
%
% Springer wants:
% All words in titles should be capitalized except for conjunctions, prepositions
% (e.g. on, of, by, and, or, but, from, with, without, under), and definite/indefinite
% articles (the, a, an), unless they appear at the beginning. Formula letters are
% typeset as in the text.
%
% Hint: https://titlecaseconverter.com/
\title{\texorpdfstring{%
    Provable Cryptography for Bitcoin: An Introduction%
  }{% Title for PDF metadata and bookmarks:
    Provable Cryptography for Bitcoin: An Introduction%
  }
}

%%% Authors
\RenewDocumentCommand\email{m}{\href{mailto:#1}{\texttt{\textcolor{black}{#1}}}}
\author{Jonas Nick}\institute{Blockstream Research\\
\email{jonas@n-ck.net}}

%%% Notes and Todos
% Usage: \newuser{a}{Alice}{orange!20}
%
% The "!20" suffix creates a mix of 20% of the preceding color and 80%
% white. See https://latexcolor.com/ for predefined colors.
\newuser{a}{Alice}{orange!20}
\newuser{b}{Bob}{olive!20}

\begin{document}

\maketitle

\begin{center}
  \textbf{License:} This work is released into the public domain under CC0 1.0\\
  \url{https://creativecommons.org/publicdomain/zero/1.0/}
\end{center}

\iffull
  \begin{center}
    \today
  \end{center}
\fi

%%% Keywords
\ifcameraready
  \keywords{a keyword \and another one \and keywords are separated with \texttt{\textbackslash{}and}}
\fi

%%% Table of Contents
\iffull
  \tableofcontents
  \clearpage
\fi

%%% Uncomment this for usage notes on this LaTeX template
% \setcounter{section}{-1}
\section{Template Usage Notes}
This is a brief overview.
See the respective package documentation for more details. (Hint: Try \texttt{texdoc -l <pkg>} if you use TeXLive or check out \url{https://texdoc.org/}.)
The source code of this file (\texttt{template-readme.tex}) may be instructive, too.


\subsection{Compilation}
It's recommended to use \texttt{latexmk}.
A configuration file is included, so simply running \texttt{latexmk} is enough.
Use \texttt{latexmk -C} to clean auxiliary files, which are all stored in the \texttt{latex.out} directory.


\subsection{Toggles}
A few toggles are provided at the top of \texttt{main.tex}:
\begin{quote}
  \begin{description}
    \item [\texttt{\textbackslash{}fulltrue}:]
          Full version of the paper. (Authors can also use \verb|\iffull| where appropriate.)
    \item [\texttt{\textbackslash{}notestrue}:]
          Enable notes and todos
    \item [\texttt{\textbackslash{}labelstrue}:]
          Print labels
    \item [\texttt{\textbackslash{}anonymoustrue}:]
          Hide authors
    \item [\texttt{\textbackslash{}spacetrue}:]
          Enable hacks to save space
    \item [\texttt{\textbackslash{}camerareadytrue}:]
          Disable some tweaks that the publisher may not like
  \end{description}
\end{quote}


\subsection{Theorem-live Environments}
The \texttt{llncs} class defines the environments
\texttt{corollary}, \texttt{definition}, \texttt{lemma}, \texttt{proposition}, and \texttt{theorem},
as well as \texttt{proof} and \texttt{claim} with a different formatting.
This template adds \texttt{assumption}.

\begin{theorem}[My Theorem]\label{thm:my-thm}
  \begin{equation}\label{eq:my-thm}
    C = g^mh^r.
  \end{equation}
  \begin{equation}\label{eq:my-thm2}
    \Pr[\bad] \le \negl.
  \end{equation}
\end{theorem}
\begin{proof}[Proof (handwavy)]
  Believe, me, this \cref{eq:my-thm}, it's true.
  But here's an equation to demonstrate that \verb|\qedhere| works:
  \[ c = m+rx . \qedhere \]
\end{proof}

\begin{lemma}\label{thm:a-lemma}
  $10\,\btcsym$ is a lot of money.
\end{lemma}
\begin{proof}
  Follows from the theorem.
\end{proof}

\begin{assumption}[VERY-HARD-PROB]\label{ass:very-hard}
  It's hard, you know?
\end{assumption}

\begin{remark}\label{rem:add-env}
  There are additional environments
  \texttt{case}, \texttt{conjecture}, \texttt{example}, \texttt{exercise}, \texttt{note}, \texttt{problem}, \texttt{property}, \texttt{question}, \texttt{remark}, and \texttt{solution}.
  These have a different formatting (like this \lcnamecref{rem:add-env}).
\end{remark}


\subsection{Citations and Bibliography (\texttt{biblatex})}
Use \texttt{get-cryptobib.sh} to add the latest \href{https://cryptobib.di.ens.fr/}{crypto.bib} as a git subtree.
The same script is used to update \texttt{crypto.bib}.

This template defaults to BibLaTeX's modern Biber backend, which can handle UTF-8 and which will enable all BibLaTeX features.
But since Biber tends to be slow and will need over a minute to parse the huge \texttt{crypto.bib},\footnote{See \url{https://github.com/plk/biber/issues/371} for background.}
we use a wrapper around Biber that automatically extracts only the cited entries from \texttt{crypto.bib} to a temporary BibTeX file that will then be processed by Biber.
All of this should happen automatically and transparently if you use \texttt{latexmk}.

You can use \verb|\textcite| to typeset author names automatically, \eg, \textcite{CCS:BelNev06,EPRINT:NicRufSeu20}.

It's a good idea to define the \texttt{shorthand} field in the bib file for BIPs, RFCs and similar documents, \eg, \verb|shorthand = {BIP340}|. By the way, have you heard about xonly keys~\cite{add:bip-schnorr}?


\subsection{References (\texttt{cleveref})}
\begin{enumerate}
  \item\label{item:one}
        Use \verb|\cref{label,maybe-another-label}| to insert a reference, \eg, \cref{sec:intro,thm:my-thm,ass:very-hard}.
        The prefix, \eg, ``Theorem'' or ``Appendix'' will be added automatically.
  \item
        References to equations, list items, and line numbers (\cref{line:false,item:one,eq:my-thm}) are not capitalized.
        Use \verb|\Cref| to force uppercase, \eg, at the beginning of a sentence.
        Then you get:
        \Cref{line:false,item:one}. But: \Cref{item:one}. And: \Cref{eq:my-thm,eq:my-thm2,item:one}.
\end{enumerate}


\subsection{Notes and Todos (\texttt{todonotes})}
Set \verb|\notestrue| to display them.
\newuser{readme}{Template:}{orange!20}
\readmenote{
  You can define a user with \texttt{\textbackslash{}newuser}, \eg,  \texttt{\textbackslash{}newuser\textbraceleft{}s\textbraceright{}\textbraceleft{}Satoshi\textbraceright{}\textbraceleft{}orange\textbraceright{}}, see the source code.
  This will create the user-specific command \texttt{\textbackslash{}snote} for inserting notes.

  A note can span multiple paragraphs.
}


\subsection{Macros, \eg, $\btcsym$, $\algo{Sign}$, $\pk$, $\defeq$, and $\xmark$ Are Mostly Robust and Can Thus be Used in Movable Arguments like Section Headings}
If not, try to put a \verb|\protect| in front of them when using them.


\subsection{Pseudocode (\texttt{cryptocode})}
See the source code.
\begin{figure}[tbhp]
  \begin{center}
    % If you omit the width, then it will be 100%.
    % Try to specify boxsep=1mm or smaller if space is tight
    \begin{tcolorbox}[width=10cm]
      \begin{pchstack}[center]
        \begin{pcvstack}
          \procedure[linenumbering]{$\algo{Alg}(\secpar)$}{%
            y \gets \ZZ_p \\
            \pcreturn \pcfalse \label{line:false}
          }
          \pcvspace
          \procedure[headlinesep=1pt]{$\game{\Game~\algo{Coll}}{\algo{H}}$}{%
            \mathellipsis
          }
        \end{pcvstack}
        \pchspace[1em]
        \begin{pcvstack}
          \procedure{$\algo{Foo}(\secpar)$}{%
            x \gets 5\\
            \gamechange{\pcassert \pcfalse}
          }
          \pcvspace
          \procedure[headlinesep=1pt]{$\oracle{Bar}(\sk, \pk)$}{%
            \pcbox{x \sample \ZZ_p \pcsc \pcreturn \sk + x} \\
            y \sample \ZZ_p
          }
        \end{pcvstack}
      \end{pchstack}
    \end{tcolorbox}
  \end{center}
  \caption{My great scheme\label{fig:scheme}}  % \label goes inside \caption
\end{figure}

\clearpage


\section{Introduction}\label{sec:intro}

This workbook has two primary goals.
First, it aims to provide sufficient background to understand state-of-the-art papers on cryptographic signatures, with a focus on discrete-logarithm-based multi-party signatures, including their security proofs.
Second, it seeks to develop the skills needed to formalize security notions for cryptographic primitives.
This skill is crucial for selecting appropriate primitives when proposing and reviewing cryptographic protocols, and for defining precisely what a protocol aims to achieve.

The concepts introduced in this section may at first seem abstract, but this level of abstraction is important: it provides the rigorous mathematical framework on which modern provable cryptography is built.
Throughout this workbook we will study algorithms, analyse their running time and success probability, both for explicit algorithms and for hypothetical adversarial algorithms.
In later sections we will sometimes study these algorithms within idealised computational models.

This workbook primarily contains definitions, propositions, and lemmas, with limited intuitive explanations.
Good cryptography papers should include intuition, but it's never complete---intuition is inherently subjective and what makes sense to one reader may not resonate with another.
What you need to do is develop your own explanations while reading the mathematics very precisely.
The goal of this workbook is to build intuition step-by-step through the exercises, which provide hands-on experience with the theoretical concepts.

Note that there is rarely a single standard definition for a concept in cryptography.
Sometimes we encounter equivalent definitions in the literature, while other times we find definitions that differ in subtle but important ways.
This is why contemporary papers in cryptography include a preliminaries section that rigorously defines the concepts they use.

% REQUIREMENTS:
% basic sets
% proof by contraposition
% basic probability and union bound
% good to have seen before:
% negligible functions
% \ppt algorithms

\subsection{Algorithms}

\say{In mathematics, \emph{algorithm} is commonly understood to be an exact prescription, defining a computational process, leading from various initial data to the desired result.} (A. Markov, 1954)

\begin{definition}
An algorithm $\adv$ is
\begin{itemize}
\item \emph{probabilistic} if it gets random bits as input (in addition to its regular input).
We write $\adv(x; \rho)$ to denote running algorithm $\adv$ on input $x$ with randomness $\rho$.
When the randomness is clear from context or unnecessary, we simply write $\adv(x)$.
\item \emph{polynomial-time} if there exists a polynomial $p$ such that for every $x \in \{0,1\}^\lambda$, $\adv(x)$ terminates in time $p(\lambda)$.
\item \emph{probabilistic polynomial-time (\ppt)} if it is probabilistic and polynomial-time.
\end{itemize}
\end{definition}

Note that the runtime of an algorithm is characterized relative to the size of the input.
All adversaries considered in cryptography are algorithms.

\subsection{Negligible Functions}



\begin{definition}
A function $f: \ZZ \rightarrow \mathbb{R}$ is \emph{negligible} if for every polynomial $p$ there exists $N \in \NN$ such that for all integers $\secpar > N$, $f(\secpar) \le \frac{1}{p(\secpar)}$.
% Or equivalently.
% - We denote the set of \emph{polynomially-bounded} functions in the security parameter $\secpar$ by $\poly = \{ f : \exists a \in \NN,\ f(\secpar) \in O(\lambda^a) \}$,. functions in the security parameter $\secpar$ by $\negl = \{ f : f(\secpar)^{-1} \not\in \poly \}$
% - for all $a \in \NN$, there exists $N \in \NN$ such that for all $\secpar \ge N$, $f(\secpar) \le \frac{1}{\lambda^a}$.
\end{definition}

\begin{example}
  $2^{-\secpar}$ is negligible. $1/\secpar$ is not negligible.
\end{example}

If a function $f$ is negligible, we write $f(\secpar) = \negl$.


\begin{lemma}[Properties of negligible functions]
  \label{lem:negl}
  \hfill
  \begin{enumerate}
  \item Let $f$ be a negligible function and $p$ be a polynomial. Then $f(\secpar) \cdot p(\secpar)$ is negligible.
  \item Let $f_1, f_2$ be negligible functions. Then $f_1 + f_2$ is negligible.
  \item Let $f$ be a negligible function $f(\secpar) \ge 0$ and $c > 0$ be a constant. Then $f + c$ is not negligible.
  \item Let $f$ be a non-negligible function and $g$ be a negligible function. Then $f - g$ is not negligible.
  \end{enumerate}
\end{lemma}

The proof of property (1) is left as an exercise (see \autoref{ex:negl-property-proof}).
The proofs of properties (2)-(4) can be found in Appendix~\ref{sec:appendix-negl-proofs}.

\subsection{Games \& Asymptotic Security}

In this section we will introduce the concept of security games (sometimes called ``experiments'') and asymptotic security through a series of toy examples.

The $\game{\Game~\algo{Guess}}{}$ defined in \autoref{fig:guessing-game} takes an algorithm $\adv$ and the security parameter $\secpar$ as input and outputs $\pctrue$ or $\pcfalse$.
The game first sets $n = 2^\lambda$ and samples $x$ uniformly at random from $\ZZ_n$.
Then it runs algorithm $\adv$ without input, sets its output to $x'$ and returns $\pctrue$ if $x = x'$.

\begin{figure}[tbhp]
  \begin{center}
    \begin{tcolorbox}[width=3cm]
      \begin{pchstack}[center]
          \procedure[headlinesep=1pt]{$\game{\Game~\algo{Guess}}{}$}{%
            n := 2^\lambda\\
            x \sample \ZZ_n \\
            x' \gets \adv() \\
            \pcreturn x = x'
          }
      \end{pchstack}
    \end{tcolorbox}
  \end{center}
  \caption{Guessing game\label{fig:guessing-game}}
\end{figure}

\begin{definition}[Guessing game advantage]
  For $\game{\Game~\algo{Guess}}{}$ as defined in \autoref{fig:guessing-game} we define the advantage of algorithm $\adv$ as
 \[
  \advantage{Guess}{\adv} \defeq \pr{\game{\algo{Guess}}{} = \pctrue}.
 \]
\end{definition}

\begin{proposition}\label{prop:guessing-game}
  For any algorithm $\adv$, $\advantage{Guess}{\adv} = \negl$.
\end{proposition}

\begin{proof}
Let $x'$ be the value returned by $\adv$.
The probability that $x$ sampled independently and uniformly at random from $[1, 2^\lambda]$ is equal to $x'$ is $2^{-\lambda} = \negl$.
\end{proof}

While the guessing game above provides a clean security definition, formalizing the security of concrete hash functions like SHA256 requires a fundamentally different approach.
To illustrate this, consider the $\game{\Game~\algo{SHAPreimgZ}}{}$ defined in \autoref{fig:sha-preimg-0}, which outputs $\pctrue$ if the algorithm $\adv$ outputs a preimage of 0 under the hash function $\algo{SHA256}: \{0, 1\}^* \rightarrow \{0, 1\}^{256}$.
Note that $\adv$ gets $1^{\secpar}$ (a string of 1s of length $\secpar$) as input in order to characterise the running time of $\adv$ as a function of the security parameter $\secpar$.

\begin{figure}[tbhp]
  \begin{center}
    \begin{tcolorbox}[width=5cm]
      \begin{pchstack}[center]
          \procedure[headlinesep=1pt]{$\game{\Game~\algo{SHAPreimgZ}}{}$}{%
            y := 0 \\
            x \gets \adv(\secparam) \\
            \pcreturn \algo{SHA256}(x) = y
          }
      \end{pchstack}
    \end{tcolorbox}
  \end{center}
  \caption{Game for finding SHA256 preimage of zero \label{fig:sha-preimg-0}}
\end{figure}

\begin{proposition}\label{prop:sha-preimg-z}
 For $\game{\Game~\algo{SHAPreimgZ}}{}$ as defined in \autoref{fig:sha-preimg-0}, let the advantage of $\adv$ be defined as
 \[
  \advantage{SHAPreIZ}{\adv} \defeq \pr{\game{\algo{SHAPreimgZ}}{} = \pctrue}.
 \]
 If there exists $x$ such that $\algo{SHA256}(x) = 0$, there exists a \ppt algorithm $\adv$ such that
  \[
  \advantage{SHAPreIZ}{\adv} = 1.
  \]
\end{proposition}

\begin{proof}
  Let $\adv$ be an algorithm
  \[
   \pcreturn x
  \]
  where $SHA256(x) = 0$.
\end{proof}

Next, we consider a variant of preimage resistance for SHA256 where the target value is chosen uniformly at random, preventing the existence of trivial algorithms like the one in Proposition~\ref{prop:sha-preimg-z}.
In $\game{\Game~\algo{SHAPreimg}}{}$ defined in \autoref{fig:sha-preimg} the value $x$ is a uniformly random bitstring of length $256$.

\begin{figure}[tbhp]
  \begin{center}
    \begin{tcolorbox}[width=5cm]
      \begin{pchstack}[center]
          \procedure[headlinesep=1pt]{$\game{\Game~\algo{SHAPreimg}}{}$}{%
            x \sample \{0, 1\}^{256}\\
            y \gets \algo{SHA256}(x) \\
            x' \gets \adv(y) \\
            \pcreturn \algo{SHA256}(x') = y
          }
      \end{pchstack}
    \end{tcolorbox}
  \end{center}
  \caption{Game for finding the SHA256 preimage of a random value \label{fig:sha-preimg}}
\end{figure}

\begin{proposition}
 For $\game{\Game~\algo{SHAPreimg}}{}$ as defined in \autoref{fig:sha-preimg}, let the advantage of $\adv$ be defined as
 \[
  \advantage{SHAPreI}{\adv} \defeq \pr{\game{\algo{SHAPreimg}}{} = \pctrue}.
 \]
 There exists a \ppt algorithm $\adv$ such that
 \[
 \advantage{SHAPreI}{\adv} = 1.
 \]
\end{proposition}

\begin{proof}
  Let $\adv$ be an algorithm that queries $\algo{SHA256}$ until it finds a solution as shown in \autoref{fig:sha-preimg-adv}.
  \begin{figure}[tbhp]
  \begin{center}
    \begin{tcolorbox}[width=5cm]
      \begin{pchstack}[center]
          \procedure[headlinesep=1pt]{$\adv(y)$}{%
          x := 0 \\
          \pcwhile \algo{SHA256}(x) \neq y \pcdo \\
            \t x := x + 1 \\
          \pcendwhile \\
          \pcreturn x
          }
      \end{pchstack}
    \end{tcolorbox}
  \end{center}
  \caption{Algorithm for finding the SHA256 preimage of a given value \label{fig:sha-preimg-adv}}
  \end{figure}
  $\adv$ will return the correct solution with probability $1$.
  The running time of $\adv$ is a constant independent of the security parameter $\secpar$ and, therefore, $\adv$ is \ppt
\end{proof}

% There is also an algorithm which just has all preimages of SHA256 hardcoded and stored in order. Then use binary search.

We observe that if the output space of the hash function is independent of the security parameter, then the hash function is not preimage-resistant.

\subsection{Hash Functions}

\begin{definition}[Hash function]
  A \emph{hash function} $\algo{H}$ is a pair of \ppt algorithms $(\algo{Gen}, \algo{Eval})$ where
  \begin{itemize}
  \item $\algo{Gen}(\secparam) \rightarrow \kappa$ is a probabilistic algorithm that takes the security parameter $\secparam$ and returns a hashing key $\kappa$ \footnote{We assume $\kappa$ includes $\secparam$ so that $\algo{Eval}$ can run in time polynomial in $\secpar$}.
  \item $\algo{Eval}(\kappa, x) \rightarrow y$ takes the hashing key $\kappa$ and $x\in \{0,1\}^*$ as input and outputs a hash $y\in \{0,1\}^{c\lambda}$ for some integer $c > 0$.\footnote{For notational simplicity, $\algo{H.Eval}(\kappa, x)$ is often written as $\algo{H}(x)$, leaving the hashing key implicit.}
  \end{itemize}
\end{definition}

Note that the hashing key $\kappa$ is not secret because it is required to evaluate the hash function.

\begin{figure}[tbhp]
  \begin{center}
    \begin{tcolorbox}[width=5cm]
      \begin{pchstack}[center]
        \procedure[headlinesep=1pt]{$\game{\Game~\algo{Preimage}}{\algo{H}}$}{%
          \kappa \gets \algo{H.Gen}(\secpar) \\
          x \sample \{0, 1\}^\lambda\\
          y \gets \algo{H.Eval}(\kappa, x) \\
          x' \gets \adv(\kappa, y) \\
          \pcreturn \algo{H.Eval}(\kappa, x') = y
        }
      \end{pchstack}
    \end{tcolorbox}
  \end{center}
  \caption{Game for finding the preimage of a given value under the hash function \label{fig:break-hash}}
\end{figure}

Similar to the algorithm in \autoref{fig:sha-preimg-adv}, there exists an algorithm that wins $\game{\Game~\algo{Preimage}}{\algo{H}}$ with probability 1 by just evaluating $\algo{H.Eval}$ on fresh inputs until $\algo{H.Eval}(x) = y$.
However, for adequate hash functions this algorithm runs in time exponential in the security parameter.
Hence, the definition of preimage-resistance only considers \ppt adversaries.

\begin{definition}[Preimage-resistance]\label{def:preimage-resistance}
  A hash function $\algo{H}$ is \emph{preimage-resistant} if for any \ppt algorithm $\adv$,
 \[
  \advantage{PreI}{\adv, \algo{H}} \defeq \pr{\game{\algo{Preimage}}{\algo{H}} = \pctrue} = \negl.
 \]
\end{definition}

% The key is necessary to make the security definitions work in a non-uniform model (where we have a different adversary for every security parameter).
% If there was no key, then, given the security parameter, there exists an algorithm where the state transitions encode a sorted map from all $2^\secpar$ possible hash function outputs y to a preimage x.
% Then the algorithm could look up the challenge y in $O(\log(2^\secpar)) = O(\secpar)$ time.

\subsection{Exercises}

\begin{exercise}\label{ex:negligible-functions}
  Determine which of the following functions are negligible in $\secpar$:
  \begin{enumerate}
    \item $f(\secpar) = 0$
    \item $f(\secpar) = \frac{1}{2}$
    \item $f(\secpar) = \frac{1}{\secpar^2}$
    \item $f(\secpar) = \secpar^{4096} \cdot 2^{-\secpar}$
    \item $f(\secpar) = \sqrt{2^{-\secpar}}$
  \end{enumerate}
\end{exercise}

\ifsolutions
\begin{mysolution}
  We analyze each function using the definition of negligible functions and Lemma~\ref{lem:negl}:
  \begin{enumerate}
    \item $f(\secpar) = 0$ is \textbf{negligible}.
    For any polynomial $p(\secpar)$, we have $0 \le \frac{1}{p(\secpar)}$ for all $\secpar > 0$.
    
    \item $f(\secpar) = \frac{1}{2}$ is \textbf{not negligible}.
    By Lemma~\ref{lem:negl}(3), since $0$ is negligible and $\frac{1}{2}$ is a positive constant, $0 + \frac{1}{2} = \frac{1}{2}$ is not negligible.
    
    \item $f(\secpar) = \frac{1}{\secpar^2}$ is \textbf{not negligible}.
    If it were negligible, then by Lemma~\ref{lem:negl}(1), $\frac{1}{\secpar^2} \cdot \secpar^2 = 1$ would be negligible.
    But $1$ is not negligible by Lemma~\ref{lem:negl}(3), giving us a contradiction.
    
    \item $f(\secpar) = \secpar^{4096} \cdot 2^{-\secpar}$ is \textbf{negligible}.
    By Lemma~\ref{lem:negl}(1), since $2^{-\secpar}$ is negligible and $\secpar^{4096}$ is a polynomial, their product is negligible.
    
    \item $f(\secpar) = \sqrt{2^{-\secpar}}$ is \textbf{negligible}.
    For any polynomial $p(\secpar)$, we need to show that $\sqrt{2^{-\secpar}} \le \frac{1}{p(\secpar)}$ for sufficiently large $\secpar$.
    Squaring both sides (valid since both are positive), we need $2^{-\secpar} \le \frac{1}{p(\secpar)^2}$.
    Since $p(\secpar)^2$ is also a polynomial and $2^{-\secpar}$ is negligible, this inequality holds for sufficiently large $\secpar$.
  \end{enumerate}
\end{mysolution}
\fi

\begin{exercise}\label{ex:guessing-game-equivalence}
  Consider the game $\game{\Game~\algo{Guess'}}{}$ which is identical to $\game{\Game~\algo{Guess}}{}$ except that $x$ is sampled \emph{after} $\adv$ is run.
  That is, the game first runs $x' \gets \adv()$, then samples $x \sample \ZZ_n$, and returns $\pctrue$ if $x = x'$.
  \begin{enumerate}
    \item Are these two games equivalent? That is, does $\advantage{Guess}{\adv} = \advantage{Guess'}{\adv}$ for all algorithms $\adv$?
    \item Does this change affect the proof of the proposition above?
  \end{enumerate}
  
  This exercise illustrates an important principle in game-based proofs, namely how the order of operations affects game outcomes.
\end{exercise}

\ifsolutions
\begin{mysolution}
  \begin{enumerate}
    \item Yes, the games are equivalent.
    Since $\adv$ receives no input and $x$ is sampled independently of $\adv$'s execution, the order doesn't matter.
    In both cases, $\adv$ outputs some fixed value $x'$ (determined by its internal randomness), and then we check whether $x = x'$ for a uniformly random $x$.
    The probability is $2^{-\lambda}$ in both cases.
    
    \item No, the proof remains the same.
    The key insight is that $\adv$'s output $x'$ is fixed before we consider the probability over the random choice of $x$.
    Whether $x$ is sampled before or after running $\adv$ doesn't affect this probability since $\adv$ has no information about $x$.
  \end{enumerate}
  
  This illustrates that when operations are independent (here, $\adv$'s execution and the sampling of $x$), their order doesn't affect the outcome—a fundamental principle in game-based proofs.
\end{mysolution}
\fi

\begin{exercise}\label{ex:asymptotic-security}
  In one sentence, when do we consider a cryptographic scheme secure in the asymptotic security framework?
\end{exercise}

\ifsolutions
\begin{mysolution}
  A scheme is secure if any \ppt adversary succeeds in breaking the scheme with at most negligible probability in $\secpar$.
\end{mysolution}
\fi

\begin{exercise}\label{ex:sha-preimage-problem}
  Why is the definition of $\advantage{SHAPreIZ}{\adv}$ in Proposition~\ref{prop:sha-preimg-z} problematic from a cryptographic perspective?
  What fundamental issue does it illustrate about concrete hash functions like SHA256?
\end{exercise}

\ifsolutions
\begin{mysolution}
  The definition is problematic because it doesn't depend on the security parameter $\secpar$ in a meaningful way.
  The adversary $\adv$ that returns a hardcoded preimage of 0 (if one exists) always succeeds with probability 1, regardless of $\secpar$.
  This illustrates that we cannot achieve asymptotic security for concrete, fixed hash functions like SHA256—their output size doesn't grow with the security parameter.
  This is why we define abstract hash function families where the output length grows with $\secpar$.
\end{mysolution}
\fi

\begin{exercise}\label{ex:sha-random-preimage-problem}
  Why is the definition of $\advantage{SHAPreI}{\adv}$ following Proposition~\ref{prop:sha-preimg-z} also problematic?
  Consider the adversary described in the proof.
\end{exercise}

\ifsolutions
\begin{mysolution}
  The adversary's running time is indeed polynomial in $\secpar$—it's a constant independent of $\secpar$, which is $O(1)$ and therefore polynomial.
  However, this constant is potentially $2^{256}$, which is astronomically large.
  The problem is that the output size (256 bits) doesn't grow with $\secpar$, so even exhaustive search becomes ``polynomial time'' in $\secpar$.
  This shows that asymptotic security only makes sense when the problem size grows with the security parameter.
\end{mysolution}
\fi

\begin{exercise}\label{ex:hash-function-definition}
  Why does the definition of preimage resistance for hash functions (Definition~\ref{def:preimage-resistance}) resolve the issues with the SHA256 examples?
\end{exercise}

\ifsolutions
\begin{mysolution}
  The hash function definition resolves the issues in two key ways:
  \begin{enumerate}
    \item The output size grows with $\secpar$ (specifically, $c\lambda$ bits for some constant $c$), so exhaustive search takes time exponential in $\secpar$, not constant time.
    \item The hash function is actually a family of functions indexed by $\kappa$, selected by $\algo{Gen}(\secparam)$, preventing hardcoding of specific preimages.
  \end{enumerate}
  Together, these ensure that the adversary's success probability can meaningfully depend on $\secpar$ and can be made negligible by increasing it.
\end{mysolution}
\fi

\begin{exercise}[Optional]\label{ex:negl-property-proof}
  Prove property (1) of Lemma~\ref{lem:negl}: If $f$ is a negligible function and $p$ is a polynomial, then $f(\secpar) \cdot p(\secpar)$ is negligible.
  
  \textbf{Hints:}
  \begin{itemize}
    \item Any polynomial $p(\lambda)$ can be bounded by $\lambda^{a}$ for some constant $a$.
    \item Use the negligibility of $f$ with the polynomial $\lambda^{a + a'}$ where $a'$ bounds an arbitrary polynomial $q$.
  \end{itemize}
\end{exercise}

\ifsolutions
\begin{mysolution}
  Let $a$ be an integer such that $p(\lambda) < \lambda^{a}$ for sufficiently large $\lambda$.
  Let $q(\lambda)$ be an arbitrary polynomial and let $a'$ be a positive integer such that $q(\lambda) \le \lambda^{a'}$ for sufficiently large $\lambda$.
  Since $f$ is negligible, for the polynomial $\lambda^{a + a'}$, there exists $N \in \NN$ such that for all $\secpar \ge N$, we have $f(\secpar) \le \frac{1}{\lambda^{a + a'}}$.
  Therefore, for all $\secpar \ge N$:
  \[
  f(\secpar) \cdot p(\secpar) \le \frac{1}{\lambda^{a + a'}} \cdot \lambda^{a} = \frac{1}{\lambda^{a'}} \le \frac{1}{q(\secpar)}
  \]
  This shows that $f(\secpar) \cdot p(\secpar)$ is negligible.
\end{mysolution}
\fi

\begin{exercise}[Optional]\label{ex:concrete-security}
  An alternative to the asymptotic approach is the \emph{concrete security} approach.
  In this approach, we define $(t,\epsilon)$-hardness as follows:
  The game $\game{\Game~\algo{Guess}}{}$ as defined in \autoref{fig:guessing-game} is $(t,\epsilon)$-hard if for any algorithm $\adv$ running in time $t$, we have $\pr{\game{\Game~\algo{Guess}}{} = \pctrue} \le \epsilon$.
  
  Give a proposition about the concrete hardness of the guessing game, analogous to Proposition~\ref{prop:guessing-game}.
\end{exercise}

\ifsolutions
\begin{mysolution}
  \textbf{Proposition:} For any $t$ and $\lambda$, the game $\game{\Game~\algo{Guess}}{}$ is $(t, 2^{-\lambda})$-hard.
  
  \textbf{Proof:} The proof is identical to Proposition~\ref{prop:guessing-game}, but we emphasize that the bound $2^{-\lambda}$ holds for \emph{all} $t$, not just polynomial time.
  
  Note: This shows that in the concrete setting, we explicitly state the security level ($2^{-\lambda}$) rather than saying it's ``negligible''.
  The concrete approach gives precise bounds rather than asymptotic statements.
\end{mysolution}
\fi

\begin{exercise}[Optional]\label{ex:foundations-hashing}
  This question explores problems when defining security for hash functions.
  Read Section 1 and Section 4 of ``Formalizing Human Ignorance''~\cite{VIETCRYPT:Rogaway06}.
  \begin{enumerate}
    \item What is the ``Foundations-of-Hashing dilemma''?
    \item To get around this, what alternative approach to defining security is suggested?
    What does ``explicitly given'' mean in this context?
    \item (Optional) See also the discussion in Appendix B.7 and the proposed approaches in Appendix B.5 of~\cite{AC:BerLan13}.
  \end{enumerate}
\end{exercise}

\ifsolutions
\begin{mysolution}
  \begin{enumerate}
    \item The Foundations-of-Hashing dilemma: For any fixed, unkeyed hash function (like SHA256), there always exists an adversary that simply prints a collision.
    This adversary exists mathematically even if we don't know how to construct it.
    This makes it impossible to prove collision resistance in the standard model.
    \item The alternative approach requires ``explicitly given'' reductions.
    This means the security proof must provide an explicit, constructive reduction that converts a collision-finding adversary into a solution to some other hard problem.
    The adversary that prints SHA256 collisions is not ``explicitly given'' because we cannot actually write down its code (we don't know any collisions).
  \end{enumerate}
\end{mysolution}
\fi

\section{Reductions}
We use \emph{reductions} to relate the hardness of winning various games.
To show that problem $Y$ is at least as hard as problem $X$, we construct a reduction: a \ppt algorithm that solves $X$ using any algorithm that solves $Y$ as a subroutine.
This reduces problem $X$ to $Y$: if there were an efficient algorithm that solves $Y$, we would also have an efficient algorithm that solves $X$.
By contrapositive, if $X$ is hard, then $Y$ must also be hard.

We illustrate the concept of reductions through a series of examples.

\begin{figure}[tbhp]
  \begin{center}
    \begin{tcolorbox}[width=7cm]
      \begin{pchstack}[center]
        \procedure[headlinesep=1pt]{$\game{\Game~\algo{PreimageEither}}{\algo{H}}$}{%
          \kappa \gets \algo{H.Gen}(\secpar) \\
          y_1, y_2 \sample \{0, 1\}^\lambda\\
          x \gets \adv(\kappa, y_1, y_2) \\
          \pcreturn \algo{H.Eval}(\kappa, x) = y_1 \vee \algo{H.Eval}(\kappa, x) = y_2
        }
      \end{pchstack}
    \end{tcolorbox}
  \end{center}
  \caption{Game for finding the preimage of either of two given values under the hash function \label{fig:break-hash-either}}
\end{figure}


\begin{definition}
  For $\game{\Game~\algo{PreimageEither}}{\algo{H}}$ as defined in \autoref{fig:break-hash-either} we define the advantage of $\adv$ as
 \[
  \advantage{PreIE}{\adv, \algo{H}} \defeq \pr{\game{\algo{PreimageEither}}{\algo{H}} = 1}.
 \]
\end{definition}

\begin{proposition}
  Let $\algo{H}$ be a hash function. If for all \ppt adversaries $\adv$, it holds that
  \[
  \advantage{PreIE}{\adv, \algo{H}} = \negl
  \]
  then $\algo{H}$ is preimage-resistant.
\end{proposition}
\begin{proof}
  We prove the contrapositive statement (which is equivalent to the statement in the proposition):
  If $\algo{H}$ is not preimage-resistant, then there exists a \ppt algorithm $\bdv$ that wins\\ $\game{\Game~\algo{PreimageEither}}{\algo{H}}[\bdv]$ with non-negligible probability.

  If $\algo{H}$ is not preimage-resistant, then there exists a \ppt algorithm $\adv$ that wins $\game{\Game~\algo{Preimage}}{\algo{H}}$ with non-negligible probability.
  Let $\bdv^\adv_\algo{H}$ be the algorithm defined in \autoref{fig:break-hash-either-bdv}.
  It gets both challenges $y_1, y_2$, runs $\adv$ on $y_1$ and returns its output.
  \begin{figure}[tbhp]
  \begin{center}
    \begin{tcolorbox}[width=5cm]
      \begin{pchstack}[center]
          \procedure[headlinesep=1pt]{$\bdv^\adv_\algo{H}(\kappa, y_1, y_2)$}{%
            x \gets \adv(\kappa, y_1) \\
            \pcreturn x
          }
      \end{pchstack}
    \end{tcolorbox}
  \end{center}
  \caption{Algorithm for finding the preimage of either of two given values under the hash function \label{fig:break-hash-either-bdv}}
  \end{figure}

  The success probability of $\bdv$ is at least that of $\adv$.
  When $\adv$ successfully finds a preimage of $y_1$, then $\bdv$ succeeds in $\Game~\algo{PreimageEither}$.
  Additionally, there is a non-zero probability that $\algo{H.Eval}(\kappa, x) = y_2$ even when $\algo{H.Eval}(\kappa, x) \neq y_1$.
  Therefore, $\advantage{PreIE}{\bdv^\adv_\algo{H}, \algo{H}} \ge \advantage{PreI}{\adv, \algo{H}}$.
  Since $\advantage{PreI}{\adv, \algo{H}}$ is non-negligible and $\bdv^\adv_\algo{H}$ is \ppt, we have found a \ppt algorithm with non-negligible advantage for $\Game~\algo{PreimageEither}$.
\end{proof}

\begin{proposition}\label{prop:preimage-either-reverse}
  Let $\algo{H}$ be a preimage-resistant hash function.
  Then $\advantage{PreIE}{\adv, \algo{H}}$ is negligible for all \ppt adversaries $\adv$.
  More precisely, for any \ppt adversary $\adv$ against $\Game~\algo{PreimageEither}$ of $\algo{H}$, there exists a \ppt adversary $\bdv$ against the preimage-resistance of $\algo{H}$ such that
    \[
    \advantage{PreIE}{\adv, \algo{H}} = 2\advantage{PreI}{\bdv^\adv_\algo{H}, \algo{H}}.
    \]
\end{proposition}

The proof is left as an exercise (see \autoref{ex:preimage-either-reverse}).


\subsection{Collision Resistance}

\begin{figure}[tbhp]
  \begin{center}
    \begin{tcolorbox}[width=8cm]
      \begin{pchstack}[center]
        \procedure[headlinesep=1pt]{$\game{\Game~\algo{Collision}}{\algo{H}}$}{%
          \kappa \gets \algo{H.Gen}(\secpar) \\
          (x,x') \gets \adv(\kappa) \\
          \pcreturn (x \neq x' \wedge \algo{H.Eval}(\kappa, x) = \algo{H.Eval}(\kappa, x'))
        }
      \end{pchstack}
    \end{tcolorbox}
  \end{center}
  \caption{Game for finding a collision under the hash function \label{fig:break-hash-collision}}
\end{figure}

\begin{definition}[Collision-resistance]
  Hash function $\algo{H}$ is collision-resistant if for any \ppt algorithm $\adv$,
 \[
  \advantage{Coll}{\adv, \algo{H}} \defeq \pr{\game{\algo{Collision}}{\algo{H}} = 1} = \negl.
 \]
\end{definition}

\begin{theorem}[Collision-resistance implies preimage-resistance]\label{thm:collision-implies-preimage}
  Let $\algo{H}$ be a collision-resistant hash function. Then $\algo{H}$ is preimage-resistant.
  More precisely, for any \ppt adversary $\adv$ against $\Game~\algo{Preimage}$ of $\algo{H}$, there exists a \ppt adversary $\bdv$ against $\Game~\algo{Collision}$ of $\algo{H}$ such that
    \[
    \advantage{PreI}{\adv, \algo{H}} \le \advantage{Coll}{\bdv^\adv_\algo{H}, \algo{H}} + 2^{-\secpar}.
    \]

\end{theorem}

The proof is left as an exercise (see \autoref{ex:collision-implies-preimage}).

\subsection{Exercises}

\begin{exercise}\label{ex:preimage-either-reverse}
  Prove \autoref{prop:preimage-either-reverse}.
  
  \textbf{Hint:} Construct a reduction that randomly places the challenge in either the first or second position.
\end{exercise}

\ifsolutions
\begin{mysolution}
  We prove the contrapositive statement:
  If there exists a \ppt algorithm $\adv$ that wins $\game{\Game~\algo{PreimageEither}}{\algo{H}}$ with non-negligible probability, then $\algo{H}$ is not preimage-resistant.
  Let $\bdv^\adv_\algo{H}$ be an algorithm that runs $\adv$ and returns its output:

  \begin{center}
    \begin{tcolorbox}[width=6cm]
      \begin{pchstack}[center]
          \procedure[headlinesep=1pt]{$\bdv^\adv_\algo{H}(\kappa, y)$}{%
            y' \sample \{0, 1\}^{\lambda} \\
            b \sample \{0, 1\} \\
            \pcif b = 0 \pcthen \\
            \t x \gets \adv(\kappa, y, y') \\
            \pcelse \\
            \t x \gets \adv(\kappa, y', y) \\
            \pcreturn x
          }
      \end{pchstack}
    \end{tcolorbox}
  \end{center}
  
  Let us analyze the success probability of $\bdv$.
  Define the following events:
  \begin{itemize}
    \item Let $E_1$ be the event that $\algo{H.Eval}(\kappa, x) = y_1$ where $x$ is the output of $\adv(\kappa, y_1, y_2)$
    \item Let $E_2$ be the event that $\algo{H.Eval}(\kappa, x) = y_2$ where $x$ is the output of $\adv(\kappa, y_1, y_2)$
    \item Let $E$ be the event that $\algo{H.Eval}(\kappa, x) \in \{y_1, y_2\}$, i.e., $E = E_1 \vee E_2$
  \end{itemize}
  
  By definition, $\pr{E} = \advantage{PreIE}{\adv, \algo{H}}$ since this is exactly the success probability of $\adv$ in the PreimageEither game.
  
  Note that $E_1$ and $E_2$ are disjoint events (since $x$ cannot simultaneously be a preimage of two different values).
  Therefore, $\pr{E} = \pr{E_1 \vee E_2} = \pr{E_1} + \pr{E_2}$.
  
  In the PreimageEither game, both $y_1$ and $y_2$ are chosen uniformly at random from $\{0,1\}^{\lambda}$ and independently, so by symmetry we have $\pr{E_1} = \pr{E_2}$.
  
  Thus: $\advantage{PreIE}{\adv, \algo{H}} = \pr{E} = \pr{E_1} + \pr{E_2} = 2 \cdot \pr{E_1}$
  
  Therefore: $\pr{E_1} = \frac{1}{2} \cdot \advantage{PreIE}{\adv, \algo{H}}$
  
  In $\bdv$'s execution, when $b = 0$, it calls $\adv(\kappa, y, y')$ and succeeds if the output is a preimage of $y$ (the first argument).
  When $b = 1$, it calls $\adv(\kappa, y', y)$ and succeeds if the output is a preimage of $y$ (now the second argument).
  
  Since $\bdv$ succeeds when either ($b = 0$ and $x$ is a preimage of the first argument) or ($b = 1$ and $x$ is a preimage of the second argument), we have:
  $\advantage{PreI}{\bdv^\adv_\algo{H}, \algo{H}} = \frac{1}{2} \cdot \pr{E_1} + \frac{1}{2} \cdot \pr{E_2} = \frac{1}{2} \cdot \advantage{PreIE}{\adv, \algo{H}}$
  
  Since $\advantage{PreIE}{\adv, \algo{H}}$ is non-negligible and $\bdv^\adv_\algo{H}$ is \ppt, $\algo{H}$ is not preimage-resistant.
\end{mysolution}
\fi

\begin{exercise}\label{ex:collision-implies-preimage}
  Prove \autoref{thm:collision-implies-preimage} (collision-resistance implies preimage-resistance).
  
  \textbf{Hint:} Construct a reduction that samples a random $x$, computes $y = \algo{H.Eval}(\kappa, x)$, and uses the preimage-finding adversary to find $x'$ such that $\algo{H.Eval}(\kappa, x') = y$. What is the probability that $x = x'$?
\end{exercise}

\ifsolutions
\begin{mysolution}
  We prove the contrapositive statement:
  If $\algo{H}$ is not preimage-resistant, then there exists a \ppt algorithm $\adv$ that wins $\game{\Game~\algo{Preimage}}{\algo{H}}$ with non-negligible probability.
  Let $\bdv^\adv_\algo{H}$ be an algorithm that runs $\adv$ and returns its output:
  
  \begin{center}
    \begin{tcolorbox}[width=4cm]
      \begin{pchstack}[center]
        \procedure[linenumbering, headlinesep=1pt]{$\bdv^\adv_\algo{H}(\kappa)$}{%
          x \sample \{0, 1\}^{(c+1)\lambda} \\
          y \defeq \algo{H.Eval}(\kappa, x) \\
          x' \gets \adv(\kappa, y) \\
          \pcassert x \neq x' \label{line:break-hash-preimage-bdv-assert-sol} \\
          \pcreturn (x, x')
        }
      \end{pchstack}
    \end{tcolorbox}
  \end{center}
  
  If $\adv$ succeeds and $\bdv$ does not abort in line 4, then $\bdv$ wins game $\game{Collision}{\algo{H}}$, or more precisely:

  \begin{align*}
  \pr{\game{Collision}{\algo{H}} = 0} &= \pr{\game{Preimage}{\algo{H}} = 0 \vee \bdv \text{ aborts at line 4} } \\
  &\le \pr{\game{Preimage}{\algo{H}} = 0} + \pr{\bdv \text{ aborts at line 4} } \\
  \end{align*}
  using union bound.
  Then, by the definition of $\advantage{Coll}{\bdv, \algo{H}}$ and $\advantage{PreI}{\adv, \algo{H}}$ we have
  \begin{align*}
    1 - \advantage{Coll}{\bdv, \algo{H}} &\le 1 - \advantage{PreI}{\adv, \algo{H}} + \pr{\bdv \text{ aborts at line 4} }\\
  \end{align*}
  and
  \begin{align*}
    \advantage{Coll}{\bdv, \algo{H}} &\ge \advantage{PreI}{\adv, \algo{H}} - \pr{\bdv \text{ aborts at line 4} }.\\
  \end{align*}
  
  We now show that $\pr{\bdv \text{ aborts at line 4} } = \negl$.
  Let us denote this event by $A$.
  Let $B_y$ denote the event that $\algo{H.Eval}(\kappa, x) = y$ and $\Ima \algo{H.Eval}(\kappa, \cdot)$ be the image of $\algo{H.Eval}$ for a fixed $\kappa$.
  Then, by the law of total probability and the definition of conditional probability we have
  \begin{align*}
    \pr{A} &= \sum_{y \in \Ima \algo{H.Eval}(\kappa, \cdot)} \pr{A \wedge B_y} \\
           &= \sum_{y \in \Ima \algo{H.Eval}(\kappa, \cdot)} \pr{B_y} \pr{A \mid B_y}
  \end{align*}
  
  Let $H^{-1}(y) = \{x \in \{0, 1\}^{(c+1)\lambda} : H(x) = y\}$, i.e., the preimage of $y$.
  Since $x$ is uniformly random from a set of size $2^{(c+1)\lambda}$, the probability $\pr{B_y}$ that $\algo{H.Eval}(\kappa, x) = y$ is $\frac{|\algo{H}^{-1}(y)|} {2^{(c+1)\lambda}}$.
  Also, the probability $\pr{A \mid B_y}$ that the sampled value $x$ matches the adversary's answer $x'$ given that $\algo{H.Eval}(\kappa, x) = y$ is $\frac{1}{|\algo{H}^{-1}(y)|}$.
  Therefore, we have
  \begin{align*}
    \pr{A} &= \sum_{y \in \Ima \algo{H.Eval}(\kappa, \cdot)} \frac{|\algo{H}^{-1}(y)|} {2^{(c+1)\lambda}} \frac{1}{|\algo{H}^{-1}(y)|}  \\
           &= \sum_{y \in \Ima \algo{H.Eval}(\kappa, \cdot)} \frac{1} {2^{(c+1)\lambda}} \\
           &= \frac{|\Ima \algo{H.Eval}(\kappa, \cdot)|} {2^{(c+1)\lambda}} \\
           &\le 2^{-\lambda} \\
  \end{align*}
  which is negligible.
  
  Since $\advantage{PreI}{\adv, \algo{H}}$ is not negligible and $\pr{\bdv \text{ aborts}} = \negl$, by \cref{lem:negl}, $\advantage{Coll}{\bdv, \algo{H}}$ is not negligible.
  Since $\bdv$ is \ppt, $\algo{H}$ is not collision-resistant.
\end{mysolution}
\fi

\begin{exercise}[Optional]
  Describe target collision resistance, extended target collision resistance, and multi-target collision resistance~\cite{PKC:HulRijSon16} in your own words and discuss their security against classical and quantum attacks (see Table 1 in the referenced paper).
\end{exercise}

\begin{exercise}[Optional]
  Drijvers et al.~\cite{SP:DEFKLN19} give a \emph{metareduction} showing that MuSig(1)~\cite{DCC:MPSW19} without the first nonce-commitment round cannot be proven secure under the one-more discrete logarithm (OMDL) assumption.
  Interpret Theorem 1 and Figure 2 in their paper.
  Complete the following informal statement: If there exists an algorithm $\bdv$ that reduces $n$-OMDL to the EUF-CMA security of the MuSig(1) variant, then there exists a reduction $\mdv$ and a forger $\mathcal{F}$ that solve the \underline{\hspace{3cm}} problem.
\end{exercise}

\ifsolutions
\begin{mysolution}
  $(n + k)$-OMDL
\end{mysolution}
\fi

\section{Commitments}\label{sec:commitments}


A commitment scheme consists of algorithm $\mathsf{Com}_\mathsf{Commit}$:
\begin{itemize}
    \item $\mathsf{Com}_\mathsf{Commit}(m;r)\rightarrow C$ outputs a commitment $C$ to message $m$ with randomness $r$.
\end{itemize}

\subsection{Syntax}
A commitment scheme $\mathsf{Com}$ is a tuple of algorithms:

\begin{itemize}
  \item $\mathsf{Com}_\mathsf{Setup}(1^\lambda) \rightarrow \mathsf{pp}$
        On input the security parameter $\lambda$ (in unary) output public
        parameters~$\mathsf{pp}$.

  \item $\mathsf{Com}_\mathsf{Commit}(\mathsf{pp},m;r)\rightarrow C$
        On message $m$ and randomness $r$ output a commitment $C$ to $m$.
\end{itemize}

% Why does this need a setup algorithm
% What are the sets, the types?

\paragraph{Example} An example is the $\mathsf{Trivial}$ commitment scheme:
\begin{itemize}
  \item $\mathsf{Trivial}.\mathsf{Com}_\mathsf{Setup}(1^\lambda) = ()$
  \item $\mathsf{Trivial}.\mathsf{Com}_\mathsf{Commit}(\mathsf{pp},m;r) = m$
\end{itemize}

\subsection{Security}
We define binding security through a game between a challenger and an adversary.
A commitment scheme is binding if no p.p.t adversary can win the following game with non-negligible probability:

\begin{figure}[tbhp]
  \begin{center}
    \begin{tcolorbox}[width=8cm]
      \begin{pchstack}[center]
        \procedure[headlinesep=1pt]{$\game{\Game~\algo{ComBind}}{\algo{Com}}$}{%
            \params \sample \mathsf{Com}_\mathsf{Setup}(1^\lambda) \\
            (m_0, m_1, r_0, r_1) \gets \adv(\params) \\
            C_0 \gets \mathsf{Com}_\mathsf{Commit}(\params, m_0; r_0) \\
            C_1 \gets \mathsf{Com}_\mathsf{Commit}(\params, m_1; r_1) \\
            \pcreturn (C_0 = C_1) \wedge (m_0 \neq m_1)
        }
      \end{pchstack}
    \end{tcolorbox}
  \end{center}
  \caption{Game for finding a binding attack on a commitment scheme \label{fig:break-com-bind}}
\end{figure}

% Why does the adversary get \params? Let say we have a Pedersen commitment scheme and the group is hardcoded in the commitment scheme.
% Then there exists an adversary that just has the collision hardcoded.

% TODO begin definition
More formally, a commitment scheme $\mathsf{Com}$ is binding if for all p.p.t. adversaries $\adv$, there exists a negligible function $\negl$ such that:
\[ \pr{\game{\algo{ComBind}}{\algo{Com}} = \pctrue} \leq \negl \]
If the advantage is 0, then the commitment scheme is perfectly binding.

% collision resistant hash function makes for a binding commitment

% What does this mean? We can make the probability as close to 0 as we want by increasing the security parameter.
% Note that we choose the pp and therefore the group in an pedersen commitment randomly.
% Hence the adversery can't just hardcode the collision. But how many groups are there of a certain order? Or do we mean with n-bit group that the prime has at least n bits, but we can also select a choice of larger groups?



%Informal
%\paragraph{Security} A commitment scheme is \emph{binding} if no PPT adversary produces two distinct messages $m_0, m_1$ and randomness $r_0, r_1$ such that $\com(m_0;r_0) = \com(m_1;r_1)$  (except with probability negligible in $\secpar$).
%A commitment scheme is \emph{hiding} if no PPT adversary obtains any information about the message from the commitment.

% define hiding
% show that hash function is not hiding in the standard model and motivate the ROM in that way??

\begin{figure}[tbhp]
  \begin{center}
    \begin{tcolorbox}[width=8cm]
      \begin{pchstack}[center]
        \procedure[headlinesep=1pt]{$\game{\Game~\algo{ComHid}}{\algo{Com}, b}$}{%
            \params \sample \mathsf{Com}_\mathsf{Setup}(1^\lambda) \\
            (m_0, m_1) \gets \adv(\params) \\
            r \sample \params.\mathcal{R} \\
            b \sample \{0, 1\} \\
            C \gets \mathsf{Com}_\mathsf{Commit}(\params, m_b; r) \\
            b' \gets \adv(\params, C) \\
            \pcreturn (b = b')
        }
      \end{pchstack}
    \end{tcolorbox}
  \end{center}
  \caption{Game for finding a hiding attack on a commitment scheme \label{fig:break-com-hid}}
\end{figure}

A commitment scheme $\mathsf{Com}$ is hiding if for all p.p.t. adversaries $\adv$, there exists a negligible function $\negl$ such that:
\[ \left| \pr{\game{\algo{ComHid}}{\algo{Com}, b} = \pctrue} - \frac{1}{2} \right| \leq \negl \]
If the advantage is 0, then the commitment scheme is perfectly hiding.

\section{Accumulators}\label{sec:accumulators}

\subsection{Syntax}

An accumulator is a compact representation of a set of elements and allows certain operations on the set.

\begin{definition}[Accumulator]
An accumulator that supports insertions of elements and membership proofs is also called an \emph{additive positive accumulator}~\cite{RSA:BalCanYak20}.
It is a tuple of \ppt algorithms $(\algo{Gen}, \algo{Init}, \algo{Value}, \algo{Insert}, \algo{ProveMembership}, \algo{VerifyMembership})$, where

% TODO: mention something about the set of elements

\begin{itemize}
    \item $\algo{Gen}(1^\secpar)\rightarrow \params$ is a \ppt algorithm that generates the public parameters $\params$ for the accumulator.
    \item $\algo{Init}(\params)\rightarrow \state$ is a deterministic algorithm that outputs the initial state $\state$ of the accumulator.
    \item $\algo{Value}(\params, \state)\rightarrow v$ outputs the accumulator value $v$ given the accumulator state $\state$.
    \item $\algo{Insert}(\params, \state, x) \rightarrow \state'$ outputs a new state $\state'$ after inserting the element $x \in \mathcal{X}_\mathsf{Acc}$ into the accumulator with state $\state$.
    \item $\algo{ProveMembership}(\params, \state, x) \rightarrow \pi$ outputs a proof $\pi$ that attests to the membership of $x$ in the accumulator with state $\state$.
    \item $\algo{VerifyMembership}(\params, v, x, \pi) \rightarrow b$ outputs $b = 1$ if the proof $\pi$ proves that $x \in \mathcal{X}_\mathsf{Acc}$ is a member of the set represented by accumulator value $v$. Otherwise, it outputs $b = 0$.
\end{itemize}
\end{definition}

Typically, an accumulator involves two parties: the manager, who uses $\algo{Init}$, $\algo{Value}$, $\algo{Insert}$, $\algo{ProveMembership}$ and the verifier, who uses $\algo{VerifyMembership}$.

\begin{example}
  An example is the trivial accumulator\label{sec:trivial-acc}, where the accumulator value is just the accumulator state and therefore of size linear in the size of the set. In contrast, the accumulator value is usually a constant size value:
  \begin{itemize}
    \item $\algo{Gen}(1^\secpar)\rightarrow \bot$.
    \item $\algo{Init}(\params)\rightarrow ()$
    \item $\algo{Value}(\params, \state)\rightarrow \state$
    \item $\algo{Insert}(\params, \state, x) \rightarrow \state \| x$
    \item $\algo{ProveMembership}(\params, \state, x) \rightarrow \bot$
    \item $\algo{VerifyMembership}(\params, v, x, \pi) \rightarrow b$ outputs $b = 1$ if $v$ contains $x$. Otherwise, it outputs $b = 0$.
  \end{itemize}
\end{example}

\paragraph{Correctness}

\begin{definition}[Correctness]
An additive positive accumulator is considered \emph{correct} for a domain $\mathcal{X}_\mathsf{Acc}$ if, for any element $x$ that was inserted into the accumulator, the membership proof $\pi$ generated honestly for $x$ is successfully verified.
More formally, for all security parameters $\secpar$, all public parameters $\params \gets \algo{Gen}(1^\secpar)$, all values $x \in \mathcal{X}_\mathsf{Acc}$, all positive integers $t$ polynomial in $\secpar$, all integers $t_x \in [1, t]$, all tuples $\vec{y} \in \mathcal{X}_\mathsf{Acc}^{t- 1}$, the following holds:

\[
    \Pr\left[
    \begin{array}{l}
        \state_0 \gets \algo{Init}(\params); \\
        \state_i \gets \algo{Insert}(\params, \state_{i - 1}, y_i) \text{ for } i \in [1, t_x - 1]; \\
        \state_{t_x} \gets \algo{Insert}(\params, \state_{i - 1}, x); \\
        \state_i \gets \algo{Insert}(\params, \state_{i - 1}, y_{i - 1}) \text{ for } i \in [t_x + 1, t]; \\
        \pi \gets \algo{ProveMembership}(\params, \state_t, x); \\
        v \defeq \algo{Value}(\params, \state_t); \\
        \algo{VerifyMembership}(\params, v, x, \pi) = 1
    \end{array}
    \right] = 1 - \negl.
\]
\end{definition}

\paragraph{Security}

In $\Game~\algo{AccColl}$ (defined in \autoref{fig:break-acc-coll}), the adversary $\adv$ is given access to \emph{oracles}, which he can call, but not look inside.
In other words, the adversary can insert elements into the accumulator and obtain membership proofs, but does not have access to the state directly.

\begin{figure}[tbhp]
  \begin{center}
    \begin{tcolorbox}[width=13cm]
      \begin{pcvstack}[center]
        \procedure[headlinesep=1pt]{$\game{\Game~\algo{AccColl}}{\algo{Acc}}$}{%
            \params \sample \algo{Acc}.\algo{Gen}(1^\secpar) \\
            \mathcal{Q} \gets \emptyset \\
            \state \gets \algo{Acc.Init}(\params) \\
            v \gets \algo{Acc.Value}(\params, \state) \\
            (x^*, \pi^*) \gets \adv^{\algo{Insert}, \algo{ProveMembership}}(\params, v) \\
            \pcreturn x^* \notin \mathcal{Q} \wedge \algo{Acc.VerifyMembership}(\params, v, x^*, \pi^*) = 1
        }
        \pcvspace
        \begin{pchstack}[center]
        \procedure[headlinesep=1pt]{Oracle $\pcoracle{Insert}(x)$}{%
            \mathcal{Q} \gets \mathcal{Q} \cup \{x\} \\
            \state \gets \algo{Acc.Insert}(\params, \state, x) \\
            \pcreturn \algo{Acc.Value}(\params, \state)
        }
        \pchspace
        \procedure[headlinesep=1pt]{Oracle $\pcoracle{ProveMembership}(x)$}{%
            \pcif x \in \mathcal{Q} \pcthen\\
                \t \pcreturn \algo{Acc.ProveMembership}(\params, \state, x) \\
            \pcelse \\
                \t \pcreturn \bot
        }
        \end{pchstack}
      \end{pcvstack}
    \end{tcolorbox}
  \end{center}
  \caption{Collision game \label{fig:break-acc-coll}}
\end{figure}

\begin{definition}[Collision-Freeness]
    An additive positive accumulator is \emph{collision-free} for a given domain $\mathcal{X}_\mathsf{Acc}$ if it is hard to fabricate a membership proof for a value that is not in the set.
    More formally, an accumulator $\algo{Acc}$ is collision-free if for any \ppt algorithm $\adv$
    \[
      \advantage{Collision}{\adv, \algo{Acc}} \defeq \pr{\game{\algo{AccColl}}{\algo{Acc}} = \pctrue} = \negl.
    \]
\end{definition}

\subsection{Exercises}

\begin{exercise}
  Informally state the correctness definition in your own words and argue whether the trivial example is correct.
\end{exercise}

\begin{exercise}
  Informally state the security definition in your own words and argue whether the trivial example is collision-free.
\end{exercise}

\begin{exercise}
  Using a collision-resistant hash function, give an example of an accumulator with a constant size accumulator value and argue that it's correct and collision-free.
\end{exercise}

% TODO: mention that the state is public, so a simpler security game would suffice for merkle trees
\begin{exercise}[Optional]
  Using a collision-resistant hash function, give an example of an accumulator with a constant size accumulator value and logarithmic membership proofs and argue that it's correct and collision-free.
\end{exercise}

% todo: add hint?
\begin{exercise}
  Define syntax, correctness and collision-freeness of an additive universal accumulator, i.e., an additive positive accumulator that supports non-membership proofs.
\end{exercise}

% todo: add hint
\begin{exercise}[Optional]
  Using a collision-resistant hash function, give an example of an additive universal accumulator with a constant size accumulator value and logarithmic membership proofs and argue that it's correct and collision-free.
\end{exercise}

\begin{exercise}[Optional]
  Read about the various types of accumulators in Section 2 of Baldimtsi, Canetti, and Yakoubov~\cite{RSA:BalCanYak20} and describe the notion of Strength in your own words.
\end{exercise}

\begin{exercise}[Optional]
  How do the accumulators required for Shielded CSV~\cite{nick2025shielded} (Sections 3.5 and 3.6) differ from the accumulator discussed in this section?
  How would you approach proving the \algo{ToSA\textnormal{-}SEC} security property of the accumulator described in Appendix A.2 (an open problem at the time of writing)?
\end{exercise}

\begin{exercise}[Optional]
  Learn about various formalizations of security properties by reading Section 5 of Cremer, Loss, and Wagner~\cite{EC:CreLosWag24}.
\end{exercise}



\section{One-Time Signatures}\label{sec:one-time-sigs}

\subsection{Syntax \& Correctness}

\begin{definition}[One-Time Signature]
    A one-time signature scheme $\algo{OTS}$ is a tuple of \ppt algorithms $(\algo{Setup}, \algo{KeyGen}, \algo{Sign}, \algo{Verify})$ with message space $\mathcal{M}$, where
    \begin{itemize}
        \item $\algo{Setup}(1^\secpar)\rightarrow \params$ is a \ppt algorithm that generates the public parameters $\params$ for the one-time signature scheme.
        \item $\algo{KeyGen}(\params)\rightarrow (\sk, \pk)$ is a \ppt algorithm that generates the key pair $(\sk, \pk)$ of the one-time signature scheme.
        \item $\algo{Sign}(\params, \sk, m) \rightarrow \sigma$ is a \ppt algorithm that outputs a signature $\sigma$ for the message $m$ under the one-time signature scheme with secret key $\sk$.
        \item $\algo{Verify}(\params, \pk, m, \sigma) \rightarrow b$ is a \ppt algorithm that outputs $b = 1$ if the signature $\sigma$ is valid for the message $m$ under the one-time signature scheme with public key $\pk$. Otherwise, it outputs $b = 0$.
    \end{itemize}
\end{definition}

\begin{definition}[Correctness]\label{def:ots-correctness}
  A one-time signature scheme $\algo{OTS} = (\algo{Setup}, \algo{KeyGen}, \algo{Sign}, \algo{Verify})$ is \emph{correct} if for all $\secpar \in \NN$, all $\params \in \algo{Setup}(1^\secpar)$, all $(\sk, \pk) \in \algo{KeyGen}(\params)$, and all messages $m \in \mathcal{M}$:
  \[
    \Pr[\algo{Verify}(\params, \pk, m, \algo{Sign}(\params, \sk, m)) = 1] = 1
  \]
  where the probability is taken over the randomness of $\algo{Sign}$.
\end{definition}

\subsection{Security}

\begin{definition}[EUF-CMA Security]\label{def:euf-cma-ot}
    A one-time signature scheme $\algo{OTS}$ is \emph{EUF-CMA secure} (Existential Unforgeability under Chosen Message Attack) if for all \ppt adversaries $\adv$, it holds that
    \[
    \advantage{EUF-CMA}{\adv, \algo{OTS}} \defeq \pr{\game{\Game~\algo{EUF-CMA}}{\algo{OTS}} = 1} = \negl.
    \]
\end{definition}

\begin{figure}[tbh]
 \begin{tcolorbox}%[left=0mm,right=0mm]
  \begin{pchstack}[center]
   \procedure{$\game{\Game~\algo{EUF-CMA}}{\algo{OTS}}$}{%
     \params \gets \algo{Setup}(1^\secpar) \\
     (\sk,\pk) \gets \algo{KeyGen}(\params) \\
     \mathcal{Q} \defeq \emptyset\\
     (m^*, \sigma) \gets \adv^{\pcoracle{Sign}}(\params, \pk) \\
     \pcreturn m^* \notin \mathcal{Q} \wedge \algo{Verify}(\params, \pk, m^*, \sigma) = 1
   }
   \pchspace
   \procedure{Oracle $\pcoracle{Sign}(m)$}{%
     \pcassert \mathcal{Q} = \emptyset \\
     \mathcal{Q} \defeq \{m\} \\
     \pcreturn \algo{Sign}(\params, \sk, m)
   }
  \end{pchstack}
 \end{tcolorbox}
 \caption{The EUF-CMA security game for a one-time signature scheme.}
 \label{fig:euf-cma-ots}
\end{figure}

\begin{figure}[t!]
 \begin{tcolorbox}
  \begin{pchstack}[center]
  \begin{pcvstack}
    \procedure{$\algo{Setup}(1^\secpar)$}{%
      \params \defeq \algo{H.Gen}(\secparam) \\
      \pcreturn \params
    }
    \pcvspace
    \procedure{$\algo{KeyGen}(\params)$}{%
      x_{i, j} \sample \{0, 1\}^\secpar \text{ for } i \in [1, k], j \in \{0, 1\} \\
      \sk \defeq (x_{1, 0}, x_{1, 1}, \dots, x_{k, 0}, x_{k, 1}) \\
      \pk \defeq (\algo{H.Eval}(\params, x_{1, 0}), \algo{H.Eval}(\params, x_{1, 1}), \dots, \\
      \t \t \algo{H.Eval}(\params, x_{k, 0}), \algo{H.Eval}(\params, x_{k, 1})) \\
      \pcreturn (\sk, \pk)
    }
  \end{pcvstack}
  \pchspace
  \pchspace
  \begin{pcvstack}
    \procedure{$\algo{Sign}(\params, \sk, m)$}{%
        \pccomment{$\algo{Sign}$ must be called at most once per $\sk$.}\\
        (m_1, \ldots, m_k) \defeq m \\
        (\sk_{1, 0}, \sk_{1, 1}), \ldots, (\sk_{k, 0}, \sk_{k, 1}) \defeq \sk \\
        \sigma \defeq (\sk_{1, m_1}, \ldots, \sk_{k, m_k}) \\
        \pcreturn \sigma
    }
    \pcvspace
    \procedure{$\algo{Verify}(\params, \pk, m, \sigma)$}{%
      (m_1, \ldots, m_k) \defeq m \\
      (\pk_{1, 0}, \pk_{1, 1}), \ldots, (\pk_{k, 0}, \pk_{k, 1}) \defeq \pk \\
      \pcreturn \bigwedge_{i=1}^k \algo{H.Eval}(\params, \sigma_i) = \pk_{i, m_i}
    }
  \end{pcvstack}
  \end{pchstack}
 \end{tcolorbox}
 \caption{The one-time Lamport signature scheme $\algo{LS}[\algo{H}, k]$.}
 \label{fig:ots-lamport}
\end{figure}

\subsection{Lamport Signatures}

\begin{definition}[Lamport Signature Scheme]\label{def:lamport}
  Let $\algo{H}$ be a hash function and $k$ be a positive integer.
  The Lamport one-time signature scheme $\algo{LS}[\algo{H}, k]$ with message space $\mathcal{M} = \{0,1\}^k$ is defined in \autoref{fig:ots-lamport}.
\end{definition}

\begin{proposition}[Correctness of Lamport Signatures]\label{prop:lamport-correctness}
  The Lamport signature scheme $\algo{LS}[\algo{H}, k]$ is correct.
\end{proposition}

The proof of \autoref{prop:lamport-correctness} is left as an exercise (see \autoref{ex:lamport-correctness}).

\begin{theorem}[Security of Lamport Signatures]\label{thm:ots-lamport-euf-cma}
    Let $\algo{H}$ be a preimage-resistant hash function.
    Then the Lamport one-time signature scheme $\algo{LS}[\algo{H}, k]$ is EUF-CMA secure.
    More precisely, for any \ppt adversary $\adv$ against the EUF-CMA security of $\algo{LS}[\algo{H, k}]$,
    there exists a \ppt adversary $\bdv$ against the preimage resistance of $\algo{H}$ such that
    \[
    \advantage{EUF-CMA}{\adv, \algo{LS}[\algo{H}, k]} \leq 2k\advantage{Preimage}{\bdv, \algo{H}}.
    \]
\end{theorem}

\begin{figure}[tbh]
 \begin{tcolorbox}%[left=0mm,right=0mm]
  \begin{pchstack}[center]
  \begin{pcvstack}
   \procedure{$\game{\Game~\algo{EUF-CMA}}{\algo{LS}[\algo{H, k}]}$}{%
     \params \gets \algo{Setup}(1^\secpar) \\
     (\sk,\pk) \gets \algo{KeyGen}(\params) \\
     \mathcal{Q} \defeq \emptyset\\
     (m^*, \sigma) \gets \adv^{\pcoracle{Sign}}(\params, \pk) \\
     \pcreturn m^* \notin \mathcal{Q} \wedge \\
      \t \algo{Verify}(\params, \pk, m^*, \sigma) = 1
   }
   \pcvspace
   \procedure{Oracle $\pcoracle{Sign}(m)$}{%
     \pcassert \mathcal{Q} = \emptyset \\
     \mathcal{Q} \defeq \{m\} \\
     \pcreturn \algo{Sign}(\params, \sk, m)
   }
  \end{pcvstack}
  \pchspace
   \begin{pcvstack}
    \procedure{$\game{\Game~\algo{G_1}}{}$}{%
      \params \gets \algo{Setup}(1^\secpar) \\
      (\sk,\pk) \gets \algo{KeyGen}(\params) \\
      \gamechange{$i \sample [1, k] \pcsc j \sample \{0, 1\}$} \\
      \mathcal{Q} \defeq \emptyset\\
      (m^*, \sigma) \gets \adv^{\pcoracle{Sign}}(\params, \pk) \\
      \gamechange{$\pcif \mathcal{Q} = \emptyset \pcthen$} \\
      \gamechange{$\t m \sample \{0, 1\}^k$} \\
      \gamechange{$\t \pcoracle{Sign}(m)$} \\
      \pcreturn m^* \notin \mathcal{Q} \wedge \\
      \t \algo{Verify}(\params, \pk, m^*, \sigma) = 1
    }
    \pcvspace
    \procedure{Oracle $\pcoracle{Sign}(m)$}{%
      \pcassert \mathcal{Q} = \emptyset \\
      \gamechange{$\pcassert m_i \neq j$} \\
      \mathcal{Q} \defeq \{m\} \\
      \pcreturn \algo{Sign}(\params, \sk, m)
    }
   \end{pcvstack}
  \pchspace
   \begin{pcvstack}
    \procedure{$\game{\Game~\algo{G_2}}{\algo{OTS}}$}{%
      \params \gets \algo{Setup}(1^\secpar) \\
      (\sk,\pk) \gets \algo{KeyGen}(\params) \\
      i \sample [1, k]  \\
      j \sample [0, 1]  \\
      \mathcal{Q} \defeq \emptyset\\
      (m^*, \sigma) \gets \adv^{\pcoracle{Sign}}(\params, \pk) \\
      \pcif \mathcal{Q} = \emptyset \pcthen \\
      \t m \sample \{0, 1\}^k \\
      \t \pcoracle{Sign}(m) \\
      \gamechange{$\pcassert m^*_i = j$} \\
      \pcreturn m^* \notin \mathcal{Q} \wedge \\
      \t \algo{Verify}(\params, \pk, m^*, \sigma) = 1
    }
    \pcvspace
    \procedure{Oracle $\pcoracle{Sign}(m)$}{%
      \pcassert \mathcal{Q} = \emptyset \\
      \pcassert m_i \neq j \\
      \mathcal{Q} \defeq \{m\} \\
      \pcreturn \algo{Sign}(\params, \sk, m)
    }
   \end{pcvstack}
  \end{pchstack}
 \end{tcolorbox}
 \caption{Sequence of games for the proof that Lamport signatures are EUF-CMA secure. Differences are \gamechange{highlighted}.}
 \label{fig:euf-cma-ots-lamport}
\end{figure}


\begin{proof}
    \autoref{fig:euf-cma-ots-lamport} shows a sequence of games for the proof that Lamport signatures are EUF-CMA secure.
    The first game is identical to the $\Game~\algo{EUF\text{-}CMA}$ game.
    Each subsequent game introduces a small change compared to the previous game.
    Then we can construct an algorithm that runs a $\Game~\algo{EUF\text{-}CMA}$ adversary $\adv$ internally and wins $\game{\Game~\algo{Preimage}}{\algo{H}}$ with the same probability as $\adv$ winning the last  $\Game~\algo{G_2}$.
    The rest of the proof is left as an exercise (see \autoref{ex:ots-lamport-euf-cma}).
\end{proof}

\subsection{Exercises}

\begin{exercise}
  What is ``existential unforgeability'' in ``EUF-CMA'' (\autoref{def:euf-cma-ot}) and what is ``chosen message attack''?
\end{exercise}

\ifsolutions
\begin{mysolution}
  \textbf{Existential unforgeability:} The adversary wins if it can forge a signature for \emph{any} message of its choice.
  \textbf{Chosen message attack:} The adversary has access to a signing oracle and can obtain a signature for a message of its choice before attempting the forgery.
\end{mysolution}
\fi

\begin{exercise}
  To see why the security definition for signatures includes a signing oracle, consider $\game{\Game~\algo{G}}{\algo{OTS}}$ which is the same as $\game{\Game~\algo{EUF-CMA}}{\algo{OTS}}$ except that the adversary $\adv$ is not given access to the signing oracle.
  Give an example of a one-time signature scheme that is correct and achieves $\pr{\game{\Game~\algo{G}}{\algo{OTS}} = 1} = \negl$ for all \ppt adversaries $\adv$ but is not EUF-CMA secure.
\end{exercise}

\ifsolutions
\begin{mysolution}
  Consider the following scheme:
  \begin{itemize}
      \item $\algo{KeyGen}(\params) = (\sk = x, \pk = \algo{H.Eval}(\params, x))$ where $x \sample \{0,1\}^\secpar$.
      \item $\algo{Sign}(\params, \sk, m) = \sk$ (simply outputs the secret key).
      \item $\algo{Verify}(\params, \pk, m, \sigma) = 1$ if and only if $\algo{H.Eval}(\params, \sigma) = \pk$.
  \end{itemize}
  This scheme is correct and would be secure in $\Game~\algo{G}$ (without signing oracle) since the adversary cannot forge without knowing $\sk$, which requires solving the preimage problem.
  However, this scheme is obviously insecure in practice: anyone who sees one signature learns the secret key and can forge signatures for any message!
  The $\Game~\algo{EUF\text{-}CMA}$ game with signing oracle correctly captures this insecurity.
\end{mysolution}
\fi

\begin{exercise}
  Is $(\sk_{1,0}, \sk_{2,0}, \sk_{3,1})$ a valid signature for the message $m = (0, 0, 1)$ under the Lamport signature scheme $\algo{LS}[\algo{H}, 3]$?
\end{exercise}

\ifsolutions
\begin{mysolution}
  Yes, it is a valid signature.
  For message $m = (0, 0, 1)$ with bits $m_1 = 0$, $m_2 = 0$, $m_3 = 1$, the Lamport signing algorithm outputs $(\sk_{1,m_1}, \sk_{2,m_2}, \sk_{3,m_3}) = (\sk_{1,0}, \sk_{2,0}, \sk_{3,1})$.
  The verification will check that $\algo{H.Eval}(\params, \sk_{i,m_i}) = \pk_{i,m_i}$ for each $i \in \{1,2,3\}$, which holds by construction of the public key during key generation.
\end{mysolution}
\fi

\begin{exercise}\label{ex:lamport-correctness}
  Prove \autoref{prop:lamport-correctness}.
\end{exercise}

\ifsolutions
\begin{mysolution}
  We prove that the Lamport signature scheme satisfies the correctness definition (\autoref{def:ots-correctness}).
  
  Let $\secpar \in \NN$, $\params \in \algo{Setup}(1^\secpar)$, $(\sk, \pk) \in \algo{KeyGen}(\params)$, and $m = (m_1, \ldots, m_k) \in \{0,1\}^k$.
  We need to show that $\Pr[\algo{Verify}(\params, \pk, m, \algo{Sign}(\params, \sk, m)) = 1] = 1$.
  
  By the definition of the Lamport scheme:
  \begin{itemize}
    \item $\algo{KeyGen}$ sets $\sk = (x_{1,0}, x_{1,1}, \ldots, x_{k,0}, x_{k,1})$ and $\pk = (\pk_{1,0}, \pk_{1,1}, \ldots, \pk_{k,0}, \pk_{k,1})$ where $\pk_{i,b} = \algo{H.Eval}(\params, x_{i,b})$.
    \item $\algo{Sign}(\params, \sk, m)$ outputs $\sigma = (\sk_{1,m_1}, \ldots, \sk_{k,m_k})$.
    \item $\algo{Verify}(\params, \pk, m, \sigma)$ outputs 1 if and only if $\bigwedge_{i=1}^k \algo{H.Eval}(\params, \sigma_i) = \pk_{i,m_i}$.
  \end{itemize}
  
  For the signature $\sigma$ produced by $\algo{Sign}$, we have $\sigma_i = \sk_{i,m_i} = x_{i,m_i}$ for each $i \in [1,k]$.
  Therefore:
  \[
    \algo{H.Eval}(\params, \sigma_i) = \algo{H.Eval}(\params, x_{i,m_i}) = \pk_{i,m_i}
  \]
  holds for all $i \in [1,k]$.
  Thus $\algo{Verify}(\params, \pk, m, \sigma) = 1$ with probability 1 (the Lamport signing algorithm is deterministic).
\end{mysolution}
\fi

\begin{exercise}
  \label{ex:ots-lamport-euf-cma}
  Complete the proof of \autoref{thm:ots-lamport-euf-cma}:
  \begin{enumerate}
      \item Express $\pr{\game{\Game~\algo{G_1}}{} = 1}$ in terms of $\pr{\game{\Game~\algo{EUF-CMA}}{\algo{LS[\algo{H}, k]}} = 1}$.
      \item Express $\pr{\game{\Game~\algo{G_2}}{} = 1}$ in terms of $\pr{\game{\Game~\algo{G_1}}{} = 1}$.
      \item Give a \ppt algorithm $\bdv^\adv$ that wins $\game{\Game~\algo{Preimage}}{\algo{H}}$ with probability $\pr{\game{\Game~\algo{G_2}}{} = 1}$.
      \item Put it all together to prove \autoref{thm:ots-lamport-euf-cma}.
  \end{enumerate}
\end{exercise}

\ifsolutions
\begin{mysolution}
   We have
  \begin{align*}
      \pr{\game{\Game~\algo{G_1}}{} = 1} &= \pr{\game{(\Game~\algo{EUF-CMA}}{\algo{LS[\algo{H}, k]}}) = 1) \wedge (m_i \neq j)} \\
      &= \pr{\game{\Game~\algo{EUF-CMA}}{\algo{LS[\algo{H}, k]}} = 1 | m_i \neq j} \pr{m_i \neq j} \\
      &= \frac{1}{2}\pr{\game{\Game~\algo{EUF-CMA}}{\algo{LS[\algo{H}, k]}} = 1} \\
  \end{align*}
  $\game{\Game~\algo{G_2}}{} = 1$ only when $m^* \neq m$ and $m^*_i = j$.
  Therefore,
  \begin{align*}
      \pr{\game{\Game~\algo{G_2}}{} = 1} &= \pr{\game{\Game~\algo{G_1}}{} = 1 \wedge m^*_i = j} \\
      &= \pr{m^*_i = j | \game{\Game~\algo{G_1}}{} = 1} \pr{\game{\Game~\algo{G_1}}{} = 1} \\
      &\geq \frac{1}{k}\pr{\game{\Game~\algo{G_1}}{} = 1} \\
  \end{align*}
  The last inequality holds because when $\Game~\algo{G_1}$ succeeds, we have $m \neq m^*$ (at least one position differs), $m_i \neq j$, and $i$ is uniformly random. The probability that position $i$ is one where $m$ and $m^*$ differ is at least $1/k$. When they differ at position $i$ and $m_i \neq j$, then $m^*_i = j$ (since the message space is binary).
  $\bdv^\adv$ is the same as $\Game~\algo{G_2}$ except that
  \begin{itemize}
      \item it gets a preimage challenge $y$ from a challenger and
      \item replaces $\pk_{i, j}$ with $y$
      \item it returns $\sigma_i$.
  \end{itemize}
  Since $m_i \neq j$, the inputs to the adversary (including the signing oracle call) has the same distribution as in $\Game~\algo{G_2}$.
  Moreover, if $\game{\Game~\algo{G_2}}{}$ returns 1, then $\algo{H.Eval}(\params, \sigma_i) = \pk_{i, j}$
  Therefore,
  \[
  \advantage{Preimage}{\bdv^\adv, \algo{H}} = \pr{\game{\Game~\algo{G_2}}{} = 1}
  \]
  and then combining the inequalities:
  \begin{align*}
    \advantage{EUF-CMA}{\adv, \algo{LS}[\algo{H, k}]} &= \pr{\game{\Game~\algo{EUF-CMA}}{\algo{LS[\algo{H}, k]}} = 1} \\
    &= 2 \cdot \pr{\game{\Game~\algo{G_1}}{} = 1} \\
    &\leq 2k \cdot \pr{\game{\Game~\algo{G_2}}{} = 1} \\
    &= 2k \cdot \advantage{Preimage}{\bdv^\adv, \algo{H}}
  \end{align*}
\end{mysolution}
\fi

\begin{exercise}
  Let $\Game~\algo{G'}$ be the same as $\Game~\algo{EUF\text{-}CMA}$ except that the signing oracle can be called twice.
  Where does the proof of \autoref{ex:ots-lamport-euf-cma} break?
\end{exercise}

\ifsolutions
\begin{mysolution}
  The proof breaks at $\Game~\algo{G_1}$, which asserts that $m_i \neq j$.
  With two signing queries, an adversary can query messages that cover both possible values at every position.
  For example, the adversary can query $m^{(1)} = (0, 0, \ldots, 0)$ and $m^{(2)} = (1, 1, \ldots, 1)$.
  For any choice of $i \in [1,k]$ and $j \in \{0,1\}$, either $m^{(1)}_i = j$ or $m^{(2)}_i = j$, causing $\Game~\algo{G_1}$ to abort.
  Therefore, $\Pr[\Game~\algo{G_1} = 1] = 0$ for this adversary, breaking the reduction.
  This shows why Lamport signatures are only secure for one-time use.
\end{mysolution}
\fi

\begin{exercise}[Optional]
  Why are Lamport signatures considered post-quantum secure?
\end{exercise}

\ifsolutions
\begin{mysolution}
  Lamport signatures are considered post-quantum secure because their security relies solely on the preimage resistance of the underlying hash function.
  Unlike RSA or ECDSA signatures that rely on the hardness of factoring or discrete logarithm problems (which can be solved efficiently by quantum computers using Shor's algorithm), hash function preimage resistance is believed to remain hard even for quantum computers.
  Grover's algorithm provides at most a quadratic speedup for finding preimages, which can be compensated by doubling the hash output length.
  Therefore, with appropriately chosen parameters, Lamport signatures remain secure against quantum adversaries.
\end{mysolution}
\fi

\begin{exercise}[Optional]
  Consider a variant of Lamport signatures that uses an accumulator (\autoref{def:accumulator}).
  Instead of storing $\pk = (\pk_{1,0}, \pk_{1,1}, \ldots, \pk_{k,0}, \pk_{k,1})$, we insert all $2k$ values $\{\pk_{i,b} : i \in [1,k], b \in \{0,1\}\}$ into an accumulator and set the public key to be the accumulator value.
  \begin{enumerate}
    \item Describe how signing and verification would work.
    \item What are the trade-offs compared to standard Lamport signatures?
    \item For security, what property of the accumulator is needed?
    Sketch how you would modify the EUF-CMA proof.
  \end{enumerate}
\end{exercise}

\ifsolutions
\begin{mysolution}
  \begin{enumerate}
    \item \textbf{Signing and Verification:}
    \begin{itemize}
      \item \textbf{KeyGen:} Generate Lamport keys as usual: $x_{i,b} \sample \{0,1\}^\secpar$ and $\pk_{i,b} = \algo{H.Eval}(\params, x_{i,b})$ for $i \in [1,k], b \in \{0,1\}$.
      Insert all $\pk_{i,b}$ values into an accumulator and output $\pk = v$ (the accumulator value) and keep $\sk = ((x_{1,0}, x_{1,1}), \ldots, (x_{k,0}, x_{k,1}), \state_{acc})$ where $\state_{acc}$ is the accumulator state.
    \item \textbf{Sign:} For message $m = (m_1, \ldots, m_k)$, compute membership proofs
      \[
        \pi_i \gets \algo{Acc.ProveMembership}(\params_{acc}, \state_{acc}, \pk_{i,m_i})
      \]
      for each $i \in [1,k]$.
      Output $\sigma = ((x_{1,m_1}, \pi_1), \ldots, (x_{k,m_k}, \pi_k))$.
    \item \textbf{Verify:} For each $i \in [1,k]$, compute $\pk'_{i,m_i} = \algo{H.Eval}(\params, x_{i,m_i})$ and check that
      \[
      \algo{Acc.VerifyMembership}(\params_{acc}, v, \pk'_{i,m_i}, \pi_i) = 1.
      \]
      Output 1 if all checks pass.
    \end{itemize}
    
    \item \textbf{Trade-offs:}
    \begin{itemize}
      \item \textbf{Advantage:} Smaller public key (just the accumulator value) instead of $2k$ hash values.
      \item \textbf{Disadvantage:} Larger signatures (must include $k$ membership proofs).
    \end{itemize}
    
    \item \textbf{Security:}
    We need the accumulator to be collision-free.
    
    \textbf{Theorem:} Let $\algo{H}$ be a preimage-resistant hash function and $\algo{Acc}$ be a collision-free accumulator.
    Then the Lamport signature scheme with accumulated public key is EUF-CMA secure.
    More precisely, for any \ppt adversary $\adv$, there exist \ppt adversaries $\bdv_1$ and $\bdv_2$ such that:
    \[
      \advantage{EUF-CMA}{\adv, \algo{LS\text{-}Acc}} \leq 2k \cdot (\advantage{Preimage}{\bdv_1, \algo{H}} + \advantage{Coll}{\bdv_2, \algo{Acc}})
    \]
    
    \textbf{Proof sketch:}
    We follow the game sequence from the original Lamport proof.
    Let $\Game~\algo{G_2}$ be the game where we guess $(i^*, j^*)$ and abort unless the adversary queries $m$ with $m_{i^*} \neq j^*$ and forges on $m^*$ with $m^*_{i^*} = j^*$.
    As in the original proof, $\Pr[\Game~\algo{G_2} = 1] \geq \frac{1}{2k} \cdot \advantage{EUF-CMA}{\adv, \algo{LS\text{-}Acc}}$.
    
    Now we build $\bdv_1$ that simulates $\Game~\algo{G_2}$:
    \begin{itemize}
      \item $\bdv_1$ receives preimage challenge $y$, embeds it as $\pk_{i^*,j^*} = y$, and runs $\Game~\algo{G_2}$ with the adversary.
      \item When the adversary succeeds in $\Game~\algo{G_2}$, they provide $(x^*_{i^*,j^*}, \pi^*_{i^*})$ where the membership proof verifies for some value $\pk' = \algo{H.Eval}(\params, x^*_{i^*,j^*})$.
      \item If $\pk' = y$, then $\bdv_1$ outputs $x^*_{i^*,j^*}$ as the preimage of $y$.
    \end{itemize}
    
    However, the adversary might succeed with $\pk' \neq y$.
    This means the adversary found a valid membership proof for a value different from what we inserted at position $(i^*, j^*)$.
    To handle this, we build $\bdv_2$:
    \begin{itemize}
      \item $\bdv_2$ also simulates $\Game~\algo{G_2}$ but generates all keys honestly (no preimage challenge).
      \item When the adversary succeeds with some $(x^*_{i^*,j^*}, \pi^*_{i^*})$ where $\pk' = \algo{H.Eval}(\params, x^*_{i^*,j^*}) \neq \pk_{i^*,j^*}$, this constitutes a collision for the accumulator.
      \item $\bdv_2$ outputs this as evidence of breaking collision-freeness.
    \end{itemize}
    
    We have:
    \begin{align}
      \Pr[\Game~\algo{G_2} = 1] &= \Pr[\Game~\algo{G_2} = 1 \wedge \pk' = \pk_{i^*,j^*}] + \Pr[\Game~\algo{G_2} = 1 \wedge \pk' \neq \pk_{i^*,j^*}]\\
      &\leq \advantage{Preimage}{\bdv_1, \algo{H}} + \advantage{Coll}{\bdv_2, \algo{Acc}}
    \end{align}
    
    Since $\advantage{EUF-CMA}{\adv, \algo{LS\text{-}Acc}} \leq 2k \cdot \Pr[\Game~\algo{G_2} = 1]$ (from the original proof), we get:
    \[
      \advantage{EUF-CMA}{\adv, \algo{LS\text{-}Acc}} \leq 2k \cdot (\advantage{Preimage}{\bdv_1, \algo{H}} + \advantage{Coll}{\bdv_2, \algo{Acc}})
    \]
  \end{enumerate}
\end{mysolution}
\fi

\begin{exercise}[Optional]
  Describe how to build a stateful many-time signature scheme from a one-time signature scheme and an accumulator.
  Your construction should support signing up to $n$ messages (where $n$ is fixed at setup).
\end{exercise}

\ifsolutions
\begin{mysolution}
  Here is a construction for a stateful many-time signature scheme:
  \begin{itemize}
    \item \textbf{Setup/KeyGen:} 
      \begin{enumerate}
        \item Generate $n$ one-time signature key pairs: $(\sk_i, \pk_i) \gets \algo{OTS.KeyGen}(\params)$ for $i \in [1,n]$.
        \item Insert all $n$ public keys $\{\pk_1, \ldots, \pk_n\}$ into an accumulator.
        \item The master public key is $\pk = v$ (the accumulator value).
        \item The master secret key is $\sk = ((\sk_1, \pk_1), \ldots, (\sk_n, \pk_n), \state_{acc})$ where $\state_{acc}$ is the accumulator state.
        \item Initialize a counter $\mathit{ctr} = 0$.
      \end{enumerate}
    \item \textbf{Sign (stateful):}
      \begin{enumerate}
        \item Increment $\mathit{ctr} \gets \mathit{ctr} + 1$.
        \item If $\mathit{ctr} > n$, abort (all keys have been used).
        \item Sign the message using the $\mathit{ctr}$-th OTS key: $\sigma_{ots} \gets \algo{OTS.Sign}(\params, \sk_{\mathit{ctr}}, m)$.
        \item Generate a membership proof: $\pi \gets \algo{Acc.ProveMembership}(\params_{acc}, \state_{acc}, \pk_{\mathit{ctr}})$.
        \item Output $\sigma = (\mathit{ctr}, \pk_{\mathit{ctr}}, \sigma_{ots}, \pi)$.
      \end{enumerate}
    \item \textbf{Verify:}
      \begin{enumerate}
        \item Parse $\sigma = (i, \pk_i, \sigma_{ots}, \pi)$.
        \item Verify the OTS signature: check that $\algo{OTS.Verify}(\params, \pk_i, m, \sigma_{ots}) = 1$.
        \item Verify accumulator membership: check that $\algo{Acc.VerifyMembership}(\params_{acc}, v, \pk_i, \pi) = 1$.
        \item Output 1 if both checks pass, 0 otherwise.
      \end{enumerate}
  \end{itemize}
  The scheme is stateful because the signer must track which keys have been used (via $\mathit{ctr}$) to ensure no OTS key is reused.
  Security relies on the one-time unforgeability of the OTS scheme and the collision-freeness of the accumulator.
\end{mysolution}
\fi

\begin{exercise}[Optional]
  Read Sections 2.2, Theorem 1, and 3.2 of Hülsing~\cite{AFRICACRYPT:Hulsing13} about the Winternitz One-Time Signature scheme.
  Explain what challenges the security reduction embeds and how they are embedded.
\end{exercise}


\section{Discrete Logarithm}\label{sec:discrete-log}

\subsection{The Discrete Logarithm Problem}

\begin{definition}[Group Generation Algorithm]
  A \emph{group generation algorithm} $\grgen$ is a \ppt algorithm that takes the security parameter $1^\secpar$ as input and outputs a group description $(\GG, p, g)$, where $\GG$ is a cyclic group of prime order $p$ and $g$ is a generator of $\GG$.
\end{definition}

\begin{remark}
  The group operation is denoted multiplicatively ($g\cdot g = g^2$).
\end{remark}


\begin{figure}[tbhp]
  \begin{center}
    \begin{tcolorbox}[width=5cm]
      \begin{pchstack}[center]
        \procedure[headlinesep=1pt]{$\game{\Game~\algo{DL}}{\grgen}$}{%
          \gparam \gets \grgen(1^\secpar) \\
          x \sample \ZZ_p \\
          X \defeq g^x \\
          x' \gets \adv(\gparam, X) \\
          \pcreturn x' = x
        }
      \end{pchstack}
    \end{tcolorbox}
  \end{center}
  \caption{The discrete logarithm game\label{fig:dl-game}}
\end{figure}

\begin{definition}[Discrete Logarithm Assumption]
  Let $\grgen$ be a group generation algorithm, and let $\game{\Game~\algo{DL}}{\grgen}$ be as defined in \autoref{fig:dl-game}.
  The discrete logarithm (DL) problem is hard for $\grgen$ if for any \ppt algorithm $\adv$,
  \[
  \advantage{DL}{\adv, \grgen} \defeq \pr{\game{\Game~\algo{DL}}{\grgen} = \pctrue} = \negl.
  \]
\end{definition}

\subsection{Pedersen Commitments}

\begin{definition}[Pedersen Commitment Scheme]
  The Pedersen commitment scheme $\algo{PedCom}[\grgen]$ is parameterized by a group generation algorithm $\grgen$ and defined as follows:
  \begin{itemize}
  \item $\algo{Setup}(1^\secpar) \rightarrow \params$: Run $\gparam \gets \grgen(1^\secpar)$, sample $h \sample \GG \setminus \{1_\GG\}$, and output $\params = (\gparam, h, \mathcal{M} \defeq \ZZ_p, \mathcal{R} \defeq \ZZ_p)$.
  \item $\algo{Commit}(\params, m; r) \rightarrow C$: Parse $\params = ((\GG, p, g), h, \mathcal{M}, \mathcal{R})$. For message $m \in \mathcal{M}$ and randomness $r \in \mathcal{R}$, output $C = g^m h^r$.
  \end{itemize}
\end{definition}

\begin{remark}
  Pedersen commitments are \emph{homomorphic}: for commitments $C_1 = \algo{Commit}(\params, m_1; r_1)$ and $C_2 = \algo{Commit}(\params, m_2; r_2)$, we have
  \[
  C_1 \cdot C_2 = g^{m_1 + m_2} h^{r_1 + r_2} = \algo{Commit}(\params, m_1 + m_2; r_1 + r_2).
  \]
\end{remark}

\begin{theorem}[Pedersen Commitments are Binding]\label{thm:pedersen-binding}
  Let $\grgen$ be a group generation algorithm for which the discrete logarithm problem is hard, then the Pedersen commitment scheme $\algo{PedCom}[\grgen]$ is binding.
  More precisely, for any \ppt adversary $\adv$ against the binding property of $\algo{PedCom}[\grgen]$, there exists a \ppt adversary $\bdv$ against the discrete logarithm problem such that
  \[
  \advantage{Bind}{\adv, \algo{PedCom}} = \advantage{DL}{\bdv, \grgen}.
  \]
\end{theorem}

The proof is left as an exercise (see \autoref{ex:pedersen-binding}).

\begin{theorem}[Pedersen Commitments are Hiding]\label{thm:pedersen-hiding}
  The Pedersen commitment scheme $\algo{PedCom}[\grgen]$ is perfectly hiding.
  More precisely, for any (possibly unbounded) adversary $\adv$,
  \[
  \advantage{Hid}{\adv, \algo{PedCom}} = 0.
  \]
\end{theorem}

The proof is left as an exercise (see \autoref{ex:pedersen-hiding}).

\subsection{The DDH Assumption}

\begin{figure}[tbhp]
  \begin{center}
    \begin{tcolorbox}[width=6cm]
      \begin{pchstack}[center]
        \procedure[headlinesep=1pt]{$\game{\Game~\algo{DDH}}{\grgen}$}{%
          \gparam \gets \grgen(1^\secpar) \\
          a, b \sample \ZZ_p \\
          A \defeq g^a, B \defeq g^b \\
          d \sample \{0, 1\} \\
          \pcif d = 0 \pcthen \\
          \t c \sample \ZZ_p \\
          \t C \defeq g^c \\
          \pcelse \\
          \t C \defeq g^{ab} \\
          d' \gets \adv(\gparam, A, B, C) \\
          \pcreturn d' = d
        }
      \end{pchstack}
    \end{tcolorbox}
  \end{center}
  \caption{The decisional Diffie-Hellman game\label{fig:ddh-game}}
\end{figure}

\begin{definition}[Decisional Diffie-Hellman Assumption]
  Let $\grgen$ be a group generation algorithm, and let $\game{\Game~\algo{DDH}}{\grgen}$ be as defined in \autoref{fig:ddh-game}.
  The decisional Diffie-Hellman (DDH) problem is hard for $\grgen$ if for any \ppt algorithm $\adv$,
  \[
  \advantage{DDH}{\adv, \grgen} \defeq \left|\pr{\game{\Game~\algo{DDH}}{\grgen} = \pctrue} - \frac{1}{2}\right| = \negl.
  \]
\end{definition}

\subsection{ElGamal Commitments}

\begin{definition}[ElGamal Commitment Scheme]
  The ElGamal commitment scheme $\algo{ElGamalCom}[\grgen]$ is parameterized by a group generation algorithm $\grgen$ and defined as follows:
  \begin{itemize}
  \item $\algo{Setup}(1^\secpar) \rightarrow \params$: Run $\gparam \gets \grgen(1^\secpar)$, sample $h \sample \GG \setminus \{1_\GG\}$, and output $\params = (\gparam, h, \mathcal{M} \defeq \ZZ_p, \mathcal{R} \defeq \ZZ_p)$.
  \item $\algo{Commit}(\params, m; r) \rightarrow C$: Parse $\params = ((\GG, p, g), h, \mathcal{M}, \mathcal{R})$. For message $m \in \mathcal{M}$ and randomness $r \in \mathcal{R}$, output $C = (g^r, h^{m+r})$.
  \end{itemize}
\end{definition}

\begin{theorem}[ElGamal Commitments are Hiding]\label{thm:elgamal-hiding}
  Let $\grgen$ be a group generation algorithm for which the DDH problem is hard, then the ElGamal commitment scheme $\algo{ElGamalCom}[\grgen]$ is hiding.
  More precisely, for any \ppt adversary $\adv$ against the hiding property of $\algo{ElGamalCom}[\grgen]$, there exists a \ppt adversary $\bdv$ against DDH such that
  \[
  \advantage{Hide}{\adv, \algo{ElGamalCom}} = 2 \cdot \advantage{DDH}{\bdv, \grgen}.
  \]
\end{theorem}

The proof is left as an exercise (see \autoref{ex:elgamal-hiding}).

\subsection{The OMDL and AOMDL Problems}

\begin{figure}[tbhp]
  \begin{center}
    \begin{tcolorbox}[width=\textwidth]
      \begin{pchstack}[center]
        \begin{pcvstack}
          \procedure[headlinesep=1pt]{$\game{\Game~\algo{OMDL}}{\grgen}$}{%
            \gparam \gets \grgen(1^\secpar) \\
            \ell \defeq 0 \pcsc q \defeq 0 \\
            \vect{y} \gets \adv^{\pcoracle{Chal}, \pcoracle{DL}}(\gparam) \\
            \vect{x} \defeq (x_1, \ldots, x_\ell) \\
            \pcreturn (\vect{y} = \vect{x} \wedge q < \ell)
          }
          \pcvspace
          \procedure[headlinesep=1pt]{Oracle $\pcoracle{DL}(X)$}{%
            q \defeq q+1 \\
            \pcreturn \log_g(X)
          }
        \end{pcvstack}
        \pchspace
        \begin{pcvstack}
          \procedure[headlinesep=1pt]{Oracle $\pcoracle{Chal}()$}{%
            \ell \defeq \ell + 1 \\
            x_\ell \sample \ZZ_p \\
            X_\ell \defeq g^{x_\ell} \\
            \pcreturn X_\ell
          }
        \end{pcvstack}
        \pchspace
        \begin{pcvstack}
          \procedure[headlinesep=1pt]{$\game{\Game~\algo{AOMDL}}{\grgen}$}{%
            \gparam \gets \grgen(1^\secpar) \\
            \ell \defeq 0 \pcsc q \defeq 0 \\
            \vect{y} \gets \adv^{\pcoracle{Chal}, \gamechange{$\oracle{ADL}$}}(\gparam) \\
            \vect{x} \defeq (x_1, \ldots, x_\ell) \\
            \pcreturn (\vect{y} = \vect{x} \wedge q < \ell)
          }
          \pcvspace
          \procedure[headlinesep=1pt]{Oracle $\oracle{ADL}((\alpha, \beta_1, \dots, \beta_{\ell}))$}{%
            q \defeq q+1 \\
            \pcreturn \alpha + \tsum_{i=1}^{\ell} {\beta_i x_i} \\
            \pclinecomment{${} = \log_g\left(g^\alpha \tprod_{i=1}^{\ell} {X_i^{\beta_i}}\right)$}
          }
        \end{pcvstack}
      \end{pchstack}
    \end{tcolorbox}
  \end{center}
  \caption{The one-more discrete logarithm (OMDL) and algebraic one-more discrete logarithm (AOMDL) games. The difference is \gamechange{highlighted}.\label{fig:omdl-aomdl-games}}
\end{figure}

\begin{definition}[One-More Discrete Logarithm Assumption]
  Let $\grgen$ be a group generation algorithm, and let $\game{\Game~\algo{OMDL}}{\grgen}$ be as defined in \autoref{fig:omdl-aomdl-games}.
  The one-more discrete logarithm (OMDL) problem is hard for $\grgen$ if for any \ppt algorithm $\adv$,
  \[
  \advantage{OMDL}{\adv, \grgen} \defeq \pr{\game{\Game~\algo{OMDL}}{\grgen} = \pctrue} = \negl.
  \]
\end{definition}

\begin{definition}[Algebraic One-More Discrete Logarithm Assumption]
  Let $\grgen$ be a group generation algorithm, and let $\game{\Game~\algo{AOMDL}}{\grgen}$ be as defined in \autoref{fig:omdl-aomdl-games}.
  The algebraic one-more discrete logarithm (AOMDL) problem is hard for $\grgen$ if for any \ppt algorithm $\adv$,
  \[
  \advantage{AOMDL}{\adv, \grgen} \defeq \pr{\game{\Game~\algo{AOMDL}}{\grgen} = \pctrue} = \negl.
  \]
\end{definition}

\begin{remark}
  When proving the security of a scheme using the AOMDL assumption, the $\pcoracle{ADL}$ oracle is (typically) used by the reduction and not by the adversary against the scheme.
  Therefore, the security proof is not restricted to adversaries that know the algebraic representations of all group elements that they output.
\end{remark}

\begin{remark}[Falsifiable Assumptions]
  In order to be able to evaluate whether an assumption is true or not, it must be falsifiable, i.e., there must be a (constructive) way to demonstrate that it is false, if this is the case.
  More precisely, a cryptographic assumption is \emph{falsifiable} if there exists a \ppt challenger algorithm that interacts with an adversary and decides whether the adversary breaks the assumption.
\end{remark}

\subsection{Exercises}

\begin{exercise}
  In the framework of asymptotic security, why is the DL problem on secp256k1 not hard? What does $\grgen$ do?
\end{exercise}

\ifsolutions
\begin{mysolution}
  The DL problem on secp256k1 is not hard in the asymptotic security framework because secp256k1 has a fixed 256-bit group order, independent of the security parameter $\secpar$.
  For asymptotic hardness, the group size must grow with $\secpar$.
  The group generation algorithm $\grgen$ generates groups of size polynomial in $\secpar$ (typically $p \approx 2^\secpar$), ensuring that the problem scales appropriately with the security parameter.
\end{mysolution}
\fi

\begin{exercise}\label{ex:pedersen-binding}
  Argue that Pedersen commitments are binding under the DL assumption (\autoref{thm:pedersen-binding}).
\end{exercise}

\ifsolutions
\begin{mysolution}
  Let $\adv$ be an adversary that breaks the binding property of Pedersen commitments.
  That is, $\adv$ outputs $(m_0, r_0, m_1, r_1)$ such that $m_0 \neq m_1$ and $g^{m_0}h^{r_0} = g^{m_1}h^{r_1}$.
  We construct $\bdv$ that breaks DL as follows:
  \begin{itemize}
    \item $\bdv$ receives $(\gparam, h)$ where $h = g^z$ for unknown $z$.
    \item $\bdv$ runs $\adv$ with parameters $\params = (\gparam, h)$.
    \item When $\adv$ outputs $(m_0, r_0, m_1, r_1)$, we have $g^{m_0}h^{r_0} = g^{m_1}h^{r_1}$.
    \item This implies $g^{m_0 - m_1} = h^{r_1 - r_0}$, so $g^{m_0 - m_1} = g^{z(r_1 - r_0)}$.
    \item If $r_0 = r_1$, then $g^{m_0 - m_1} = 1$, which contradicts $m_0 \neq m_1$ (since $g$ generates $\GG$).
    \item Therefore $r_0 \neq r_1$, and we can compute $z = \frac{m_0 - m_1}{r_1 - r_0} \bmod p$.
    \item $\bdv$ outputs $z$. Also $\bdv$ is \ppt if $\adv$ is.
  \end{itemize}
  The reduction is perfect: whenever $\adv$ succeeds in breaking binding, $\bdv$ succeeds in computing the discrete log.
  Therefore, $\advantage{DL}{\bdv, \grgen} = \advantage{Bind}{\adv, \algo{PedCom}}$.
\end{mysolution}
\fi

\begin{exercise}\label{ex:pedersen-hiding}
  Prove that Pedersen commitments are perfectly hiding (\autoref{thm:pedersen-hiding}).
\end{exercise}

\ifsolutions
\begin{mysolution}
  For any message $m \in \ZZ_p$, the commitment $C = g^m h^r$ is uniformly distributed over $\GG$ when $r$ is chosen uniformly from $\ZZ_p$.
  Since $h \in \GG \setminus \{1_\GG\}$ and $\GG$ has prime order $p$, $h$ generates $\GG$ (every non-identity element generates a prime-order group).
  Thus the map $r \mapsto h^r$ is a bijection from $\ZZ_p$ to $\GG$.
  For fixed $m$, the map $r \mapsto g^m h^r$ is also a bijection from $\ZZ_p$ to $\GG$.
  Therefore, for any two messages $m_0, m_1$, the distributions of $\algo{Commit}(\params, m_0; r)$ and $\algo{Commit}(\params, m_1; r)$ for uniform $r$ are both uniform over $\GG$.
\end{mysolution}
\fi

\begin{exercise}\label{ex:elgamal-hiding}
  Prove that ElGamal commitments are hiding under the DDH assumption (\autoref{thm:elgamal-hiding}).
\end{exercise}

\ifsolutions
\begin{mysolution}
   Let $\adv$ be an adversary against the hiding property of ElGamal.
  We construct $\bdv$ against DDH:
  \begin{itemize}
    \item $\bdv$ receives $(\gparam, A, B, C)$ where $A = g^a$, $B = g^b$, and either $C = g^{ab}$ or $C = g^c$ for random $c$.
    \item $\bdv$ sets $h = B$ and gives $\params = (\gparam, h)$ to $\adv$.
    \item When $\adv$ outputs $(m_0, m_1)$, $\bdv$ chooses $e \sample \{0, 1\}$ and computes:
          If $C = g^{ab}$, set implicitly $r = a$, so $(g^r, h^{m_e + r}) = (g^a, g^{b(m_e + a)}) = (A, g^{bm_e + ab}) = (A, B^{m_e} \cdot C)$.
          If $C = g^c$ for random $c$, then $(A, B^{m_e} \cdot C) = (g^a, g^{bm_e + c})$.
    \item $\bdv$ gives $(A, B^{m_e} \cdot C)$ to $\adv$.
    \item When $\adv$ outputs $e'$, $\bdv$ outputs $1$ if $e' = e$ and $0$ otherwise.
  \end{itemize}

  Analysis: We have:
  \begin{itemize}
      \item When $C = g^{ab}$ (real tuple): $\bdv$ perfectly simulates the hiding game, so $|\Pr[e = e' | C = g^{ab}] - \frac{1}{2}| = \advantage{Hide}{\adv, \algo{ElGamalCom}}$.
      \item When $C = g^c$ (random tuple): The commitment reveals no information about $e$, so $\Pr[e = e' | C = g^c] = \frac{1}{2}$.
  \end{itemize}
  For the DDH advantage:
  \begin{align}
  \advantage{DDH}{\bdv, \grgen} &= |\Pr[d = d'] - \frac{1}{2}|\\
  \Pr[d = d'] &= \Pr[d' = 1 \wedge d = 1] + \Pr[d' = 0 \wedge d = 0]\\
  &= \Pr[d' = 1 \wedge C = g^{ab}] + \Pr[d' = 0 \wedge C = g^c]\\
  &= \Pr[d' = 1 | C = g^{ab}] \cdot \Pr[C = g^{ab}] + \Pr[d' = 0 | C = g^c] \cdot \Pr[C = g^c]\\
  &= \frac{1}{2} \Pr[d' = 1 | C = g^{ab}] + \frac{1}{2} \Pr[d' = 0 | C = g^c]\\
  &= \frac{1}{2} \Pr[d' = 1 | C = g^{ab}] + \frac{1}{2}(1 - \Pr[d' = 1 | C = g^c])\\
  &= \frac{1}{2} \Pr[d' = 1 | C = g^{ab}] + \frac{1}{2} - \frac{1}{2}\Pr[d' = 1 | C = g^c]
  \end{align}
  Therefore:
  \begin{align}
  \advantage{DDH}{\bdv, \grgen} &= \left|\frac{1}{2} \Pr[d' = 1 | C = g^{ab}] - \frac{1}{2}\Pr[d' = 1 | C = g^c]\right|\\
  &= \frac{1}{2} |\Pr[d' = 1 | C = g^{ab}] - \Pr[d' = 1 | C = g^c]|\\
  &= \frac{1}{2} |\Pr[e = e' | C = g^{ab}] - \Pr[e = e' | C = g^c]|\\
  &= \frac{1}{2} \left|\Pr[e = e' | C = g^{ab}] - \frac{1}{2}\right|\\
  &= \frac{1}{2} \cdot \advantage{Hide}{\adv, \algo{ElGamalCom}}
  \end{align}

  This gives us: $\advantage{Hide}{\adv, \algo{ElGamalCom}} = 2 \cdot \advantage{DDH}{\bdv, \grgen}$. $\bdv$ is \ppt\ if $\adv$ is.
\end{mysolution}
\fi

\begin{exercise}
  Argue that if OMDL is hard for a group generation algorithm then DL is hard.
\end{exercise}

\ifsolutions
\begin{mysolution}
  If OMDL is hard, then DL must be hard.
  Given a DL adversary $\adv$, we construct an OMDL adversary $\bdv$:
  \begin{itemize}
    \item $\bdv$ calls $\pcoracle{Chal}()$ once to get $X_1$.
    \item $\bdv$ runs $\adv(\gparam, X_1)$ to get $x_1'$.
    \item $\bdv$ outputs $(x_1')$.
  \end{itemize}
  If $\adv$ succeeds, then $X_1 = g^{x_1'}$, so $\bdv$ solves one discrete log with zero challenge queries, winning the OMDL game.
\end{mysolution}
\fi

\begin{exercise}
  Consider an algorithm $\adv$ playing the AOMDL game. It queries $\pcoracle{Chal}$ twice to obtain $X_1$ and $X_2$, then samples $\alpha, \beta_1, \beta_2 \sample \ZZ_p$. 
  \begin{enumerate}
    \item How should $\adv$ call the $\oracle{ADL}$ oracle to obtain $\log_g(P)$ where $P = g^\alpha X_1^{\beta_1} X_2^{\beta_2}$?
    \item Assuming $\adv$ makes no more $\pcoracle{Chal}$ queries, how many more $\oracle{ADL}$ queries can it make?
    \item What must $\adv$ return to win the game?
  \end{enumerate}
\end{exercise}

\ifsolutions
\begin{mysolution}
  \begin{enumerate}
    \item $\adv$ should call $\oracle{ADL}((\alpha, \beta_1, \beta_2))$. The oracle will return $\alpha + \beta_1 x_1 + \beta_2 x_2 = \log_g(g^\alpha X_1^{\beta_1} X_2^{\beta_2}) = \log_g(P)$.
    \item Since $\ell = 2$ (two $\pcoracle{Chal}$ queries) and $q = 1$ (one $\oracle{ADL}$ query), and the winning condition requires $q < \ell$, the adversary can make at most 0 more queries (as $1 < 2$ but $2 \not< 2$).
    \item $\adv$ must return $(x_1, x_2)$, the discrete logarithms of $X_1$ and $X_2$.
  \end{enumerate}
\end{mysolution}
\fi

\begin{exercise}
  Is the OMDL assumption falsifiable? Is the AOMDL assumption falsifiable? Do we prefer the security of a scheme to be based on OMDL or AOMDL?
\end{exercise}

\ifsolutions
\begin{mysolution}
  The OMDL assumption is \emph{non-falsifiable} because verifying the adversary's output requires computing discrete logarithms to check that $g^{x_i} = X_i$, which cannot be done in polynomial time classically.

  The AOMDL assumption is \emph{falsifiable} because:
  \begin{itemize}
    \item The ADL oracle can be implemented efficiently: given $(\alpha, \beta_1, \ldots, \beta_\ell)$, it simply returns $\alpha + \sum_{i=1}^\ell \beta_i x_i$
    \item The winning condition can be verified by checking that the returned values equal the stored $x_i$ values
    \item All operations (additions, multiplications) are polynomial time
  \end{itemize}

  We prefer basing security on falsifiable assumptions like AOMDL. Note that AOMDL is a weaker assumption than OMDL---if AOMDL doesn't hold, then OMDL doesn't hold either (since any OMDL adversary is also an AOMDL adversary).
\end{mysolution}
\fi

\begin{exercise}[Optional]
  Why is the DL problem not hard in general?
\end{exercise}

\ifsolutions
\begin{mysolution}
  The DL problem is not hard in general because quantum computers can solve DL in polynomial time using Shor's algorithm.
\end{mysolution}
\fi

\begin{exercise}[Optional]
  Read about the Generic Group Model (GGM) in the introduction of~\cite{EC:Shoup97} and~\cite{IMA:Maurer05}, and argue why the DLP is hard for classical computational models.
\end{exercise}

\begin{exercise}[Optional]
  How does Silent Payments~\cite{add:bip-silentpayments} rely on the DDH assumption?
\end{exercise}

\begin{exercise}[Optional]
  Study the concept of falsifiability of cryptographic assumptions by reading Naor~\cite{C:Naor03}.
  Analyze the following assumptions and explain the extent to which each one is falsifiable:
  \begin{itemize}
    \item Factoring
    \item The RSA assumption
    \item The Decisional Diffie-Hellman assumption
    \item The Knowledge-of-Exponent assumption
  \end{itemize}
\end{exercise}

\section{Random Oracle Model}\label{sec:rom}

\subsection{Random Oracles}

A \emph{random oracle} (RO) is an oracle that can be provided to an algorithm that takes inputs in $\{0, 1\}^*$ and returns elements from some set $S$ where $|S| \geq 2^\secpar$.
The key property is that the oracle responds consistently: identical queries always yield identical responses, while distinct queries yield independent uniformly random values from $S$.

Formally, a random oracle can be described as follows.
Consider the random oracle $\pcoracle{H}$ in \autoref{fig:ro-example}.
The oracle maintains a table $T$ (initially empty) to ensure consistency.
When queried with input $x$, it checks whether $T[x]$ exists. If not, it samples a uniformly random element from $S$ and stores it as $T[x]$.
The oracle returns $T[x]$.


\begin{figure}[tbhp]
  \begin{center}
    \begin{tcolorbox}[width=8cm]
      \begin{pchstack}
        \procedure[headlinesep=1pt]{$\game{\Game~\algo{ROExample}}{}$}{%
          T \defeq \emptyset \\
          r \sample \{0,1\}^\secpar \\
          h \defeq \pcoracle{H}(r) \\
          \pcreturn \adv^{\pcoracle{H}}(h)
        }
        \pchspace
        \procedure[headlinesep=1pt]{Oracle $\pcoracle{H}(x)$}{%
          \pcif T[x] = \bot \pcthen \\
          \t T[x] \sample S \\
          \pcreturn T[x]
        }
      \end{pchstack}
    \end{tcolorbox}
  \end{center}
  \caption{Random oracle example game\label{fig:ro-example}}
\end{figure}

Before examining the consequences of random oracles, we present a simple lemma that will be useful for analyzing games that differ only when certain events occur.

\begin{lemma}[Difference Lemma]\label{lem:difference}
  Let $A$, $B$ and $E$ be events in a probability space. If $\Pr[A \wedge \neg E] = \Pr[B \wedge \neg E]$, then
  \[
  |\Pr[A] - \Pr[B]| \leq \Pr[E].
  \]
\end{lemma}

\begin{proof}
  \begin{align*}
    |\Pr[A] - \Pr[B]| &= |\Pr[A \wedge E] + \Pr[A \wedge \neg E] - (\Pr[B \wedge E] + \Pr[B \wedge \neg E])| \\
    &= |\Pr[A \wedge E] - \Pr[B \wedge E]|
  \end{align*}
  Since $\Pr[A \wedge E] \leq \Pr[E]$ and $\Pr[B \wedge E] \leq \Pr[E]$, both terms are in $[0, \Pr[E]]$, so their difference is at most $\Pr[E]$.
\end{proof}


\begin{lemma}\label{lem:ro-indistinguishable}
  No adversary against $\algo{ROExample}$ can distinguish between being given $h = \pcoracle{H}(r)$ for a random $r$ and being given a uniformly random value from $S$, except with probability negligible in the number of oracle queries.
  More precisely, let $\game{\Game~\algo{ROExample}}{}$ be as defined in \autoref{fig:ro-example}.
  For any adversary $\adv$ making $q$ queries to the random oracle $\pcoracle{H}$, we have
  \[
  \left|\Pr[\game{\algo{ROExample}}{} = 1] - \Pr[\game{\algo{ROExample}_1}{} = 1]\right| \leq \frac{q}{2^\secpar}
  \]
  where $\game{\algo{ROExample}_1}{}$ is defined in \autoref{fig:ro-game-hop}.
\end{lemma}

\begin{figure}[tbhp]
  \begin{center}
    \begin{tcolorbox}[width=8cm]
      \begin{pchstack}
        \procedure[headlinesep=1pt]{$\game{\Game~\algo{ROExample}_1}{}$}{%
          T \defeq \emptyset \\
          r \sample \{0,1\}^\secpar \\
          h \defeq \pcoracle{H}(r) \\
          \gamechange{$h' \sample S$} \\
          \pcreturn \adv^{\pcoracle{H}}(\gamechange{$h'$})
        }
        \pchspace
        \procedure[headlinesep=1pt]{Oracle $\pcoracle{H}(x)$}{%
          \gamechange{$\pcassert x \neq r$} \\
          \pcif T[x] = \bot \pcthen \\
          \t T[x] \sample S \\
          \pcreturn T[x]
        }
      \end{pchstack}
    \end{tcolorbox}
  \end{center}
  \caption{Game hop for random oracle indistinguishability\label{fig:ro-game-hop}}
\end{figure}

\begin{proof}
  By inspection of the code, we observe that $\game{\algo{ROExample}_1}{}$ is identical to $\game{\algo{ROExample}}{}$ except that:
  \begin{enumerate}
  \item The input to $\adv$ is changed from $h = \pcoracle{H}(r)$ to a uniformly random $h' \sample S$.
  \item The oracle $\pcoracle{H}$ asserts that no query equals $r$.
  \end{enumerate}
  
  Since $h'$ is uniformly random in $S$ and has the same distribution as $\pcoracle{H}(r)$ (when $r$ is not queried), the games are identical unless the adversary queries $r$ to the oracle.
  
  We apply the Difference Lemma (\autoref{lem:difference}).
  Let $E$ denote the event that $\adv$ queries $r$ to the oracle.
  
  When $\neg E$ occurs (i.e., $r$ is not queried), the two games behave identically, so $\Pr[\game{\algo{ROExample}}{} = 1 \wedge \neg E] = \Pr[\game{\algo{ROExample}_1}{} = 1 \wedge \neg E]$. 
  
  By Lemma~\ref{lem:difference}:
  \begin{align*}
    \left|\Pr[\game{\algo{ROExample}}{} = 1] - \Pr[\game{\algo{ROExample}_1}{} = 1]\right| &\leq \Pr[E] \\
    &= \Pr[\exists i \in [q] : x_i = r] \\
    &\leq \sum_{i=1}^q \Pr[x_i = r] \tag{union bound} \\
    &= \sum_{i=1}^q \frac{1}{2^\secpar} = \frac{q}{2^\secpar}
  \end{align*}
  where $x_1, \ldots, x_q$ are the queries made by $\adv$ to $\pcoracle{H}$.
\end{proof}

\subsection{ROM}

  For a large class of cryptographic schemes, the properties of hash functions seen so far are insufficient to prove security.
  Instead, we prove security of a modified scheme where hash function evaluations are replaced with random oracle calls.
  This is the \emph{random oracle model} (ROM)~\cite{CCS:BelRog93}.
  Random oracles cannot be instantiated in the real world because their output must be distributed uniformly at random unless explicitly queried.
  On the other hand, the output of a hash function is deterministic once the key is known.
  Moreover, there's no way to formally define an ``explicit query'' for public hash functions.

  There exist contrived schemes provably secure in the ROM that become insecure when instantiated with any concrete hash function (see optional exercise).
  Despite these theoretical concerns, the ROM has proven very useful in practice for analyzing real-world cryptographic protocols~\cite{DCC:KobMen15}.

\subsection{Hash Commitments}

\begin{definition}[Hash Commitment Scheme]
  Let $\algo{H}$ be a hash function.
  The hash commitment scheme $\algo{HashCom}[\algo{H}]$ is defined as follows:
  \begin{itemize}
    \item $\algo{Setup}(1^\secpar) \rightarrow \params$: Output $\params = (\mathcal{M} \defeq \{0, 1\}^*, \mathcal{R} \defeq \{0, 1\}^\secpar)$.
    \item $\algo{Commit}(\params, m, r) \rightarrow C$: Output $C = \algo{H.Eval}(m \| r)$.
  \end{itemize}
\end{definition}

\begin{lemma}\label{lem:hash-com-hiding}
  The hash commitment scheme $\algo{HashCom}[\algo{H}]$ is hiding in the random oracle model for $\algo{H}$.
  More precisely, for any algorithm $\adv$ making at most $q$ queries to $\algo{H}$,
  \[
  \advantage{Hide}{\adv, \algo{HashCom}} \leq \frac{2q}{2^\secpar}.
  \]
\end{lemma}

The proof is left as an exercise (see \autoref{ex:hash-com-hiding}).

\subsection{Taproot Commitments}

\begin{definition}[Taproot Commitment Scheme~\cite{add:bip-taproot}]
  Let $\grgen$ be a group generation algorithm and $\algo{H}$ be a hash function with output in $\ZZ_p$ where $p$ is the group order. 
  The Taproot commitment scheme $\algo{TapCom}[\grgen, \algo{H}]$ is defined as follows:
  \begin{itemize}
    \item $\algo{Setup}(1^\secpar) \rightarrow \params$: Run $\gparam \gets \grgen(1^\secpar)$ and $\kappa \gets \algo{H.Gen}(1^\secpar)$. Parse $\gparam = (\GG, p, g)$. Output $\params = (\gparam, \kappa, \mathcal{M} \defeq \{0,1\}^*, \mathcal{R} \defeq \GG)$.
    \item $\algo{Commit}(\params, m, R) \rightarrow C$: Parse $\params$ to obtain $g$ and $\kappa$. Output $C = R \cdot g^{\algo{H.Eval}(\kappa, (R, m))}$.
  \end{itemize}
\end{definition}

\begin{lemma}\label{lem:taproot-binding}
  Let $\grgen$ be a group generation algorithm and $\algo{H}$ be a hash function.
  The Taproot commitment scheme $\algo{TapCom}[\grgen, \algo{H}]$ is binding in the random oracle model for $\algo{H}$.
  More precisely, for any algorithm $\adv$ making at most $q$ queries to $\algo{H}$,
  \[
  \advantage{Bind}{\adv, \algo{TapCom}} \leq \frac{(q+2)^2}{2^\secpar}.
  \]
\end{lemma}

The proof is left as an exercise (see \autoref{ex:taproot-binding}).

\subsection{Exercises}

\begin{exercise}\label{ex:rom-contrived-hashcom}
  Consider a modified hash commitment scheme $\algo{HashCom}_y[\algo{H}]$ defined as follows:
  \begin{itemize}
    \item $\algo{Setup}(1^\secpar) \rightarrow \params$: Same as $\algo{HashCom}[\algo{H}]$.
    \item $\algo{Commit}(\params, m, r) \rightarrow C$: 
    \begin{align*}
      C = \begin{cases}
        \algo{H.Eval}(m) & \text{if } \algo{H.Eval}(0) = y \\
        \algo{H.Eval}(m \| r) & \text{otherwise}
      \end{cases}
    \end{align*}
  \end{itemize}
  
  \begin{enumerate}
    \item Using Lemma~\ref{lem:hash-com-hiding}, argue that for all $y \in \{0,1\}^\secpar$, $\algo{HashCom}_y[\algo{H}]$ is hiding in the ROM.
    \item Show that there exists $y$ such that $\algo{HashCom}_y$ instantiated with SHA256 is not hiding.
  \end{enumerate}
\end{exercise}

\ifsolutions
\begin{mysolution}
  \begin{enumerate}
    \item In the ROM, $\algo{H}$ behaves as a random oracle. For any fixed $y$, we have $\Pr[\algo{H.Eval}(0) = y] = \frac{1}{2^\secpar}$, which is negligible. With overwhelming probability, the scheme behaves identically to the standard $\algo{HashCom}[\algo{H}]$, which is hiding by Lemma~\ref{lem:hash-com-hiding}.
    
    More formally, the hiding advantage is bounded by:
    \[
    \advantage{Hide}{\adv, \algo{HashCom}_y} \leq \Pr[\algo{H.Eval}(0) = y] + \advantage{Hide}{\adv, \algo{HashCom}} \leq \frac{1}{2^\secpar} + \frac{2q}{2^\secpar}
    \]
    
    \item Let $y = \text{SHA256}(0)$. When instantiated with SHA256, we always have $\algo{H.Eval}(0) = y$, so the commitment becomes $C = \text{SHA256}(m)$ without any randomness. This is deterministic and therefore not hiding: given two messages $m_0, m_1$, the adversary can compute $\text{SHA256}(m_0)$ and $\text{SHA256}(m_1)$ and compare with $C$ to determine which message was committed to.
  \end{enumerate}
\end{mysolution}
\fi

\begin{exercise}\label{ex:hash-com-hiding}
  Prove Lemma~\ref{lem:hash-com-hiding} using a similar technique as in Lemma~\ref{lem:ro-indistinguishable}.
\end{exercise}

\ifsolutions
\begin{mysolution}
\autoref{fig:hashcom-hiding-rom} shows the hiding game for $\algo{HashCom}[\algo{H}]$ where $\algo{H}$ is modeled as a random oracle.

\begin{figure}[h]
  \begin{center}
    \begin{tcolorbox}[width=8cm]
      \begin{pchstack}
        \procedure[headlinesep=1pt]{$\game{\Game~\algo{ComHide}}{\algo{HashCom}}$}{%
          \params \gets \algo{Setup}(1^\secpar) \\
          (m_0, m_1) \gets \adv(\params) \\
          T \defeq \emptyset \\
          b \sample \{0,1\} \\
          r \sample \{0,1\}^\secpar \\
          c \defeq \pcoracle{H}(m_b \| r) \\
          b' \gets \adv^{\pcoracle{H}}(c) \\
          \pcreturn b = b'
        }
        \pchspace
        \procedure[headlinesep=1pt]{Oracle $\pcoracle{H}(x)$}{%
          \pcif T[x] = \bot \pcthen \\
          \t T[x] \sample \{0,1\}^\secpar \\
          \pcreturn T[x]
        }
      \end{pchstack}
    \end{tcolorbox}
  \end{center}
  \caption{The hiding game for $\algo{HashCom}[\algo{H}]$ in the random oracle model.}
  \label{fig:hashcom-hiding-rom}
\end{figure}

Now consider the modified game shown in \autoref{fig:hashcom-hiding-rom-modified}:

\begin{figure}[h]
  \begin{center}
    \begin{tcolorbox}[width=8cm]
      \begin{pchstack}
        \procedure[headlinesep=1pt]{$\game{\Game~\algo{ComHide}_1}{\algo{HashCom}}$}{%
          \params \gets \algo{Setup}(1^\secpar) \\
          (m_0, m_1) \gets \adv(\params) \\
          T \defeq \emptyset \\
          b \sample \{0,1\} \\
          r \sample \{0,1\}^\secpar \\
          c \defeq \pcoracle{H}(m_b \| r) \\
          \gamechange{$c' \sample \{0,1\}^\secpar$} \\
          b' \gets \adv^{\pcoracle{H}}(\gamechange{$c'$}) \\
          \pcreturn b = b'
        }
        \pchspace
        \procedure[headlinesep=1pt]{Oracle $\pcoracle{H}(x)$}{%
          \gamechange{$\pcassert x \neq m_0 \| r$} \\
          \gamechange{$\pcassert x \neq m_1 \| r$} \\
          \pcif T[x] = \bot \pcthen \\
          \t T[x] \sample \{0,1\}^\secpar \\
          \pcreturn T[x]
        }
      \end{pchstack}
    \end{tcolorbox}
  \end{center}
  \caption{Modified hiding game for $\algo{HashCom}[\algo{H}]$ with changes \gamechange{highlighted}.}
  \label{fig:hashcom-hiding-rom-modified}
\end{figure}

Note that in $\game{\algo{ComHide}_1}{\algo{HashCom}}$, the value $b$ is independent of the adversary's view, so:
\[
\Pr[\game{\algo{ComHide}_1}{\algo{HashCom}} = 1] = \frac{1}{2}
\]

The games differ only when $\adv$ queries either $m_0 \| r$ or $m_1 \| r$ to the oracle.
By the Difference Lemma (\autoref{lem:difference}), we can bound the difference between the games by the probability of this event:

\begin{align*}
  \advantage{Hide}{\adv, \algo{HashCom}} &= \left|\Pr[\game{\algo{ComHide}}{\algo{HashCom}} = 1] - \frac{1}{2}\right| \\
  &= \left|\Pr[\game{\algo{ComHide}}{\algo{HashCom}} = 1] - \Pr[\game{\algo{ComHide}_1}{\algo{HashCom}} = 1]\right| \\
  &\leq \Pr[\exists i \in [q] : x_i \in \{m_0 \| r, m_1 \| r\}]
\end{align*}

Since $r$ is uniformly random from $\{0,1\}^\secpar$ and unknown to $\adv$, and the adversary knows both $m_0$ and $m_1$, we apply the union bound:

\begin{align*}
  \Pr[\exists i \in [q] : x_i \in \{m_0 \| r, m_1 \| r\}] &\leq \sum_{i=1}^q \Pr[x_i = m_0 \| r \text{ or } x_i = m_1 \| r] \\
  &\leq \sum_{i=1}^q \left(\Pr[x_i = m_0 \| r] + \Pr[x_i = m_1 \| r]\right) \\
  &= \sum_{i=1}^q \left(\frac{1}{2^\secpar} + \frac{1}{2^\secpar}\right) \\
  &= \sum_{i=1}^q \frac{2}{2^\secpar} = \frac{2q}{2^\secpar}
\end{align*}
\end{mysolution}
\fi

\begin{exercise}\label{ex:taproot-binding}
  Prove Lemma~\ref{lem:taproot-binding}.
\end{exercise}

\ifsolutions
\begin{mysolution}
  Let $\adv$ be an adversary against the binding property of $\algo{TapCom}[\grgen, \algo{H}]$. The adversary wins if it outputs $(m_0, m_1, R_0, R_1)$ with $m_0 \neq m_1$ such that:
  \[
  R_0 \cdot g^{\algo{H}(R_0, m_0)} = R_1 \cdot g^{\algo{H}(R_1, m_1)}
  \]

  Rearranging, this equality holds if and only if:
  \[
  R_0 \cdot g^{\algo{H}(R_0, m_0)} \cdot R_1^{-1} = g^{\algo{H}(R_1, m_1)}
  \]

  Our strategy is to check for binding collisions as queries are made to the random oracle.
  Consider the distinct queries to the random oracle of the form $(R_i, m_i)$ and define the collision predicate:
  \[
  c(i, j) = \text{true} \iff R_i \cdot g^{\algo{H}(R_i, m_i)} \cdot R_j^{-1} = g^{\algo{H}(R_j, m_j)}
  \]
  
  The probability that the adversary breaks binding is:
  \[
  \advantage{Bind}{\adv, \algo{TapCom}} = \Pr\left[\bigvee_{i=1}^{q'} \bigvee_{j=i+1}^{q'} c(i,j) = \text{true}\right]
  \]
  where $q'$ is the number of distinct queries.
  
  For all $j > i$, we have $\Pr[c(i,j) = \text{true}] = \frac{1}{p}$, since $\algo{H}(R_j, m_j)$ is a fresh uniformly random value in $\ZZ_p$, independent of $R_i$, $\algo{H}(R_i, m_i)$, and $R_j$.
  
  By the union bound:
  \begin{align*}
    \advantage{Bind}{\adv, \algo{TapCom}} &= \Pr\left[\bigvee_{i=1}^{q'} \bigvee_{j=i+1}^{q'} c(i,j) = \text{true}\right] \\
    &\leq \sum_{i=1}^{q'} \sum_{j=i+1}^{q'} \Pr[c(i,j) = \text{true}] \\
    &= \sum_{i=1}^{q'} \sum_{j=i+1}^{q'} \frac{1}{p} \\
    &= \binom{q'}{2} \cdot \frac{1}{p} < \frac{(q')^2}{2p}
  \end{align*}
  
  The total number of queries includes the adversary's queries (at most $q$) plus the 2 queries made by the binding game when checking $C_0 = C_1$. Thus $q' \leq q + 2$.
  
  Since $p \geq 2^{\secpar-1}$, we have:
  \[
  \advantage{Bind}{\adv, \algo{TapCom}} < \frac{(q+2)^2}{2p} \leq \frac{(q+2)^2}{2 \cdot 2^{\secpar-1}} = \frac{(q+2)^2}{2^\secpar}
  \]
\end{mysolution}
\fi

\begin{exercise}[Optional]
  \begin{enumerate}
    \item Let $\algo{H}$ be a hash function with output in $\{0,1\}^\secpar$ and let $\kappa$ be chosen uniformly at random from a set of size $2^\secpar$. Define $A = \algo{H.Eval}(\kappa, 0) \| \algo{H.Eval}(\kappa, 1)$. What is the size of the set over which $A$ is uniformly distributed?
    \item Let $\algo{H}$ be a random oracle with output in $\{0,1\}^\secpar$. Define $B = \algo{H}(0) \| \algo{H}(1)$. What is the size of the set over which $B$ is uniformly distributed?
  \end{enumerate}
\end{exercise}

\ifsolutions
\begin{mysolution}
  \begin{enumerate}
    \item There are $2^\secpar$ possibilities for $\kappa$, and each determines a unique value of $A$. Therefore, $A$ is uniformly distributed over a set of size at most $2^\secpar$.
    \item Since $\algo{H}(0)$ and $\algo{H}(1)$ are independent uniformly random values in $\{0,1\}^\secpar$, $B$ is uniformly distributed over a set of size $2^{2\secpar}$.
  \end{enumerate}
\end{mysolution}
\fi

\begin{exercise}[Optional]
  Let $\algo{H}$ be a random oracle with output in $\{0,1\}^\secpar$. Define the function $G: (\{0,1\}^*)^{\secpar+1} \rightarrow \{0,1\}^\secpar$ as:
  \[
  G(x_0, x_1, \ldots, x_\secpar) = \algo{H}(x_0) \oplus \algo{H}(x_1) \oplus \cdots \oplus \algo{H}(x_\secpar)
  \]
  where $\oplus$ denotes bitwise XOR. Is $G$ preimage resistant? That is, given $y \in \{0,1\}^\secpar$, is it hard for a \ppt adversary with oracle access to $\algo{H}$ to find $(x_0, \ldots, x_\secpar)$ such that $G(x_0, \ldots, x_\secpar) = y$?
\end{exercise}

\ifsolutions
\begin{mysolution}
  No, $G$ is not preimage resistant. Wagner's algorithm for the generalized birthday problem~\cite{C:Wagner02} can find a preimage efficiently.
\end{mysolution}
\fi

\begin{exercise}[Optional]
  Why is the ROM not considered adequate to model quantum computation?
\end{exercise}

\begin{exercise}[Optional]
  Study the diagonalization argument in Section 4 of Canetti, Goldreich, and Halevi~\cite{STOC:CanGolHal98}.
  Explain how this argument demonstrates the existence of a contrived signature scheme that is provably secure in the random oracle model but becomes insecure when the random oracle is instantiated with any concrete hash function (not just a specific fixed hash function as considered in \autoref{ex:rom-contrived-hashcom}).
\end{exercise}

\section{Programmable ROM and Forking Lemma}\label{sec:prog-rom}

\subsection{Programmable Random Oracles}

In the random oracle model, we can program the oracle to return specific values for certain inputs, as long as we maintain consistency and the correct distribution.
This technique is particularly useful in security reductions.

For example, consider the hash commitment scheme $\algo{HashCom}[\algo{H}]$ where commitments are of the form $\algo{H}(m \| r)$.
A reduction can program the oracle such that certain algebraic relations hold between committed messages, while maintaining the adversary's view.

The following proposition demonstrates that careful programming preserves the adversary's advantage. Notably, $\Game_1$ forces the XOR of committed messages to be equal to a message $m^*$ that is determined before even running the adversary.
The game first draws $m^*$, runs the adversary with a random hash $h$ and obtains a hash  $h'$ from the adversary.
If the adversary created $h'$ as $\algo{Commit}(m', r')$, then when the adversary is invoked a second time we have
\begin{align*}
  \algo{Commit}(m, r) &= h\\
  \algo{Commit}(m', r') &= h'\\
  m \oplus m' &= m^*.
\end{align*}
Note that the challenger's commitment $h$ was given to the adversary before the adversary returned commitment $h'$ and the adversary has never revealed the opening $(m', r')$ to the challenger.

A similar technique is used in MuSig multi-signature scheme~\cite{DCC:MPSW19} to simulate signing queries (i.e., to answer signing queries without knowledge of the secret key, as explained in the next section).

\begin{figure}[tbhp]
  \begin{center}
    \begin{tcolorbox}[width=\textwidth]
      \begin{pchstack}
        \procedure[headlinesep=1pt]{$\game{\Game_0}{}$}{%
          \params \gets \algo{HashCom.Setup}(1^\secpar) \\
          T \defeq \emptyset \\
          m \sample \{0,1\} \\
          r \sample \{0,1\}^\secpar \\
          h \defeq \pcoracle{H}(m \| r) \\
          (\state, h') \gets \adv^{\pcoracle{H}}(\params, h) \\
          \pcreturn \adv^{\pcoracle{H}}(\state, m, r) \\
        }
        \pchspace
        \procedure[headlinesep=1pt]{Oracle $\pcoracle{H}(x)$}{%
          \pcif T[x] = \bot \pcthen \\
          \t T[x] \sample \{0,1\}^\secpar \\
          \pcreturn T[x]
        }
        \pchspace
        \procedure[headlinesep=1pt]{$\game{\Game_1}{}$}{%
          \params \gets \algo{HashCom.Setup}(1^\secpar) \\
          T \defeq \emptyset \\
          m^* \sample \{0,1\} \\
          h \sample \{0,1\}^\secpar \\
          (\state, h') \gets \adv^{\pcoracle{H}}(\params, h) \\
          \pcif \exists (m', r') : T[m' \| r'] = h' \pcthen \\
          \t \text{Choose any such } (m', r') \\
          \t m \defeq m^* \oplus m' \\
          \pcelse \\
          \t m \sample \{0, 1\} \\
          r \sample \{0,1\}^\secpar \\
          \pcassert T[m \| r] = \bot \\
          T[m \| r] \defeq h \\
          \pcreturn \adv^{\pcoracle{H}}(\state, m, r) \\
        }
      \end{pchstack}
    \end{tcolorbox}
  \end{center}
  \caption{Games $\Game_0$ and $\Game_1$ \label{fig:prog-rom-games}}
\end{figure}

\begin{proposition}\label{prop:prog-rom}
  Let $\game{\Game_0}{}$ and $\game{\Game_1}{}$ be as defined in \autoref{fig:prog-rom-games}.
  For any adversary $\adv$ making at most $q$ queries to the oracle $\pcoracle{H}$, we have
  \[
  \left|\Pr[\game{\Game_0}{} = 1] - \Pr[\game{\Game_1}{} = 1]\right| \leq \frac{q}{2^\secpar}
  \]
\end{proposition}

The proof is left as an exercise (see \autoref{ex:prog-rom}).

\subsection{The Forking Lemma}

The forking lemma is a fundamental tool for analyzing algorithms that interact with random oracles. It considers an adversary $\adv$ with oracle access to $\pcoracle{H}$, which is executed twice on different but related instances of the oracle:

\begin{itemize}
  \item In the first execution, all random oracle responses are sampled uniformly as usual.
  \item In the second execution, the responses are identical to those from the first execution up to a specific query called the \emph{forking point}, whose response is \emph{refreshed} (resampled with fresh randomness).
\end{itemize}

As a result, the two executions of $\adv$ proceed identically up to the forking point but diverge afterwards. The forking lemma quantifies how often an adversary that succeeds in the first execution also succeeds in the second execution, where the oracle gives a different response at the forking point.

\begin{figure}[tbhp]
  \begin{center}
    \begin{tcolorbox}[width=\textwidth]
      \begin{pchstack}[center]
        \begin{pcvstack}
          \procedure[headlinesep=1pt]{$\game{\Game~\algo{Single}}{\algo{InpGen}}$}{%
            \mathit{inp} \gets \algo{InpGen}(1^\secpar) \\
            \rho \sample \mathcal{R} \\
            T \defeq \emptyset \\
            \alpha \gets \adv^{\pcoracle{H}}(\mathit{inp}; \rho) \\
            \pcassert \alpha \neq \bot \\
            \pcreturn 1
          }
          \pcvspace
          \procedure[headlinesep=1pt]{Oracle $\pcoracle{H}(x)$}{%
            \pcif T[x] = \bot \pcthen \\
            \t T[x] \sample S \\
            \pcreturn T[x]
          }
        \end{pcvstack}
        \pchspace
        \begin{pcvstack}
          \procedure[headlinesep=1pt]{$\game{\Game~\algo{Fork}}{\algo{InpGen}}$}{%
            \mathit{inp} \gets \algo{InpGen}(1^\secpar) \\
            \rho \sample \mathcal{R} \\
            T \defeq \emptyset \\
            \alpha \gets \adv^{\pcoracle{H}}(\mathit{inp}; \rho) \\
            \pcassert \alpha \neq \bot \\
            (z, \mathit{aux}) \defeq \alpha \\
            T' \defeq T \\
            T'[z] \sample S \\
            \alpha' \gets \adv^{\pcoracle{H'}}(\mathit{inp}; \rho) \\
            \pcassert \alpha' \neq \bot \\
            (z', \mathit{aux}') \defeq \alpha' \\
            \pcassert z = z' \wedge T[z] \neq T'[z] \\
            \pcreturn (z, \mathit{aux}, z', \mathit{aux}')
          }
          \pcvspace
          \procedure[headlinesep=1pt]{Oracle $\pcoracle{H'}(x)$}{%
            \pcif T'[x] = \bot \pcthen \\
            \t T'[x] \sample S \\
            \pcreturn T'[x]
          }
        \end{pcvstack}
      \end{pchstack}
    \end{tcolorbox}
  \end{center}
  \caption{The Single and Fork games\label{fig:fork-games}}
\end{figure}

\begin{lemma}[Local Forking Lemma~\cite{AC:BelDaiLi19}]\label{lem:fork}
  Let $q \geq 1$ be an integer and $\algo{InpGen}$ be a randomized algorithm that outputs some input $\mathit{inp}$.
  Let $\adv$ be a randomized algorithm that takes input $\mathit{inp}$ and random coins $\rho \in \mathcal{R}$, has oracle access to $\pcoracle{H}: \{0,1\}^* \rightarrow S$ where $|S| \geq 2^\secpar$, makes at most $q$ queries to $\pcoracle{H}$, and returns either $\bot$ or a pair $(z, \mathit{aux})$ where $z \in \{0,1\}^*$ and $\mathit{aux}$ is auxiliary output.
  % TODO: The requirement $|S| \geq 2^\secpar$ is too strict when $S = \ZZ_p$ with prime $p \geq 2^{\secpar-1}$. Need to fix this.
  
  For games $\game{\algo{Single}}{\algo{InpGen}}$ and $\game{\algo{Fork}}{\algo{InpGen}}$ defined in \autoref{fig:fork-games}, define:
  \begin{align*}
    \advantage{Single}{\adv, \algo{InpGen}} &\defeq \Pr[\game{\algo{Single}}{\algo{InpGen}} = 1] \\
    \advantage{Fork}{\adv, \algo{InpGen}} &\defeq \Pr[\game{\algo{Fork}}{\algo{InpGen}} \neq \bot]
  \end{align*}
  
  Then:
  \[
  \advantage{Fork}{\adv, \algo{InpGen}} \geq \advantage{Single}{\adv, \algo{InpGen}} \left( \frac{\advantage{Single}{\adv, \algo{InpGen}}}{q} - \frac{1}{2^\secpar} \right)
  \]
\end{lemma}

\subsection{Exercises}

\begin{exercise}
  Describe the similarities and differences between games $\Game_0$ and $\Game_1$ in \autoref{fig:prog-rom-games}.
\end{exercise}

\ifsolutions
\begin{mysolution}
  In both games:
  \begin{itemize}
    \item The oracle $\pcoracle{H}$ behaves identically as a random oracle
    \item The adversary receives a commitment $h$ and later the opening $(m, r)$
    \item The games return the same bit $b$ output by the adversary
  \end{itemize}
  
  The key differences are:
  \begin{itemize}
    \item In $\Game_0$: $m$ and $r$ are chosen first, then $h = \pcoracle{H}(m \| r)$
    \item In $\Game_1$: $h$ is chosen uniformly at random, then we program the oracle so that $\pcoracle{H}(m \| r) = h$
    \item In $\Game_1$: If $h'$ is a valid hash commitment (i.e., $\exists (m', r')$ such that $\pcoracle{H}(m' \| r') = h'$), then $m \oplus m' = m^*$, where $m^*$ was determined before running $\adv$.
  \end{itemize}
\end{mysolution}
\fi

\begin{exercise}
  Consider $\Game_F$ which is exactly like $\Game_1$ except that $m^* \defeq 0$ (instead of sampling uniformly). Give a \ppt adversary $\adv$ that makes one random oracle query and achieves
  \[
    \left|\Pr[\game{\Game_0}{} = 1] - \Pr[\game{\Game_F}{} = 1]\right| = \frac{1}{2}.
  \]
\end{exercise}

\ifsolutions
\begin{mysolution}
  Consider the following adversary $\adv$:
  
  \textbf{First invocation:}
  \begin{itemize}
    \item Query $h' \gets \pcoracle{H}(0 \| 0^\secpar)$
    \item Output $(\bot, h')$
  \end{itemize}
  
  \textbf{Second invocation:}
  \begin{itemize}
    \item Output $m$
  \end{itemize}
  
  Analysis:
  \begin{itemize}
    \item In $\Game_0$: $m$ is uniform, so $\Pr[\game{\Game_0}{} = 1] = \frac{1}{2}$.
    \item In $\Game_F$: Since $m^* = 0$ (fixed in $\Game_F$) and $m' = 0$, we have $m = m^* \oplus m' = 0 \oplus 0 = 0$. Thus $\Pr[\game{\Game_F}{} = 1] = 0$.
  \end{itemize}
  
  Therefore: $\left|\Pr[\game{\Game_0}{} = 1] - \Pr[\game{\Game_F}{} = 1]\right| = \left|\frac{1}{2} - 0\right| = \frac{1}{2}$.
\end{mysolution}
\fi

\begin{exercise}\label{ex:prog-rom}
  Sketch a proof of Proposition~\ref{prop:prog-rom}.
  \textbf{Hints:}
  \begin{enumerate}
    \item Let $E$ be the event that the adversary queries $m \| r$ to the oracle during its first invocation (i.e., the event that triggers the assertion).
    Argue that $\Pr[\game{\Game_0}{} = 1 \wedge \neg E] = \Pr[\game{\Game_1}{} = 1 \wedge \neg E]$.
    \item Bound the probability of $E$.
  \end{enumerate}
\end{exercise}

\ifsolutions
\begin{mysolution}
    Let $E$ be the event that the adversary queries $m \| r$ to the oracle during its first invocation (when it outputs $(\state, h')$).
  
  We claim that $\Pr[\game{\Game_0}{} = 1 \wedge \neg E] = \Pr[\game{\Game_1}{} = 1 \wedge \neg E]$. To see this, we analyze the distribution of all values conditioned on $\neg E$:
  
  \begin{itemize}
    \item \textbf{The value $h$:} In $\Game_0$, $h = \pcoracle{H}(m \| r)$ where the oracle uses lazy sampling. Since $\neg E$ occurs, the adversary never queried $m \| r$ during the first invocation, so $h$ is a fresh uniform sample. In $\Game_1$, $h$ is explicitly drawn uniformly at random. Thus $h$ is uniform in $\{0,1\}^\secpar$ in both games.
    
    \item \textbf{The value $r$:} In both games, $r$ is drawn uniformly from $\{0,1\}^\secpar$.
    
    \item \textbf{The value $m$:} In $\Game_0$, $m$ is drawn uniformly from $\{0,1\}$. In $\Game_1$:
      \begin{itemize}
        \item If $h'$ is a valid commitment: $m = m^* \oplus m'$ where $m^*$ is uniform in $\{0,1\}$ and independent of the adversary's view
        \item If $h'$ is not a valid commitment: $m$ is drawn uniformly from $\{0,1\}$
      \end{itemize}
      In both cases, $m$ is uniform in $\{0,1\}$.
    
    \item \textbf{The oracle programming:} In $\Game_1$, we set $T[m \| r] = h$. Since $\neg E$ occurs, the input $m \| r$ is fresh (never queried before), so this programming is invisible to the adversary and maintains consistency with $h = \pcoracle{H}(m \| r)$.
  \end{itemize}
  
  Since all values have identical distributions and the oracle behaves identically in both games (returning the same values for the same queries), the adversary's view is bit-by-bit identical conditioned on $\neg E$. Therefore $\Pr[\game{\Game_0}{} = 1 \wedge \neg E] = \Pr[\game{\Game_1}{} = 1 \wedge \neg E]$.
  
  By Lemma~\ref{lem:difference}, we have:
  \[
  \left|\Pr[\game{\Game_0}{} = 1] - \Pr[\game{\Game_1}{} = 1]\right| \leq \Pr[E]
  \]
  
  To bound $\Pr[E]$, note that $E$ occurs if the adversary queries $m \| r$ during the first invocation. The key observation is that $r \in \{0,1\}^\secpar$ is chosen uniformly at random and independently of the adversary's first invocation:
  \begin{itemize}
    \item In $\Game_0$: $r$ is chosen at the beginning but is unknown to the adversary
    \item In $\Game_1$: $r$ is chosen after the first invocation
  \end{itemize}
  
  In both cases, for any specific query $x_i$ made by the adversary during the first invocation, the probability that $x_i = m \| r$ is at most $\frac{1}{2^\secpar}$ (the adversary must guess the $\secpar$-bit value $r$ correctly).
  
  Since the adversary makes at most $q$ queries during the first invocation, by the union bound:
  \[
  \Pr[E] \leq \sum_{i=1}^q \Pr[x_i = m \| r] \leq \sum_{i=1}^q \frac{1}{2^\secpar} = \frac{q}{2^\secpar}
  \]
  
  Therefore:
  \[
  \left|\Pr[\game{\Game_0}{} = 1] - \Pr[\game{\Game_1}{} = 1]\right| \leq \frac{q}{2^\secpar}
  \]
\end{mysolution}
\fi

\begin{exercise}
  Consider an algorithm $\adv$ that never queries $\pcoracle{H}$. What is $\advantage{Fork}{\adv, \algo{InpGen}}$?
\end{exercise}

\ifsolutions
\begin{mysolution}
  If $\adv$ never queries $\pcoracle{H}$, then the table $T$ remains empty throughout the first execution. In the Fork game:
  
  \begin{itemize}
    \item First execution: $\adv$ outputs $\alpha = (z, \mathit{aux})$ with probability $\advantage{Single}{\adv, \algo{InpGen}}$
    \item Since $z$ was never queried, $T[z] = \bot$ (undefined)
    \item We copy $T$ to $T'$, so $T'[z] = \bot$ as well
    \item We then set $T'[z] \sample S$ (a fresh random value)
    \item Second execution: Since $\adv$ receives the same $\mathit{inp}$ and $\rho$, and never queries the oracle, it outputs the same $(z, \mathit{aux})$ deterministically
    \item The condition $z = z'$ is satisfied
    \item The condition $T[z] \neq T'[z]$ is satisfied since $T[z] = \bot$ while $T'[z] \in S$
  \end{itemize}
  
  Therefore:
  \[
  \advantage{Fork}{\adv, \algo{InpGen}} = \advantage{Single}{\adv, \algo{InpGen}}
  \]
\end{mysolution}
\fi

\begin{exercise}
  Consider an algorithm $\adv$ that makes exactly one query $z \gets \pcoracle{H}(\rho)$ and outputs $(z, \bot)$. What is $\advantage{Fork}{\adv, \algo{InpGen}}$?
\end{exercise}

\ifsolutions
\begin{mysolution}
  In this case, $\adv$ always succeeds in outputting a non-$\bot$ value, so $\advantage{Single}{\adv, \algo{InpGen}} = 1$. In the Fork game:
  
  \begin{itemize}
    \item First execution: $\adv$ queries $z \gets \pcoracle{H}(\rho)$ and outputs $(z, \bot)$. Since $\rho$ is fixed, $z$ is determined by the oracle's response.
    \item We copy $T$ to $T'$, so $T'[\rho] = T[\rho] = z$
    \item We then set $T'[z] \sample S$ (refreshing the value at the forking point)
    \item Second execution: $\adv$ queries $z' \gets \pcoracle{H'}(\rho) = T'[\rho] = z$ (unchanged) and outputs $(z', \bot) = (z, \bot)$
    \item The condition $z = z'$ is satisfied
    \item The condition $T[z] \neq T'[z]$ is satisfied with probability $1 - \frac{1}{|S|} \geq 1 - \frac{1}{2^\secpar}$ (since $T'[z]$ was refreshed to a random value)
  \end{itemize}
  
  Therefore:
  \[
  \advantage{Fork}{\adv, \algo{InpGen}} = 1 \cdot \left(1 - \frac{1}{2^\secpar}\right) = 1 - \frac{1}{2^\secpar}
  \]
  
  This matches the forking lemma bound when $q = 1$ and $\advantage{Single}{\adv, \algo{InpGen}} = 1$:
  \[
  1 \cdot \left(\frac{1}{1} - \frac{1}{2^\secpar}\right) = 1 - \frac{1}{2^\secpar}
  \]
\end{mysolution}
\fi


\begin{exercise}
  Let $\algo{InpGen}$ output $\mathit{inp} = (\GG,p,g)$ where $g$ is a generator of a cyclic group $\GG$ of prime order $p \geq 2^{\secpar-1}$.
  Let $\adv$ be an algorithm that makes at most $q$ queries to a random oracle $\pcoracle{H}: \{0,1\}^* \rightarrow \ZZ_p$ and always succeeds in outputting $(z, s)$ such that $s = z + \pcoracle{H}(z) \cdot x \bmod p$ for some fixed $x \in \ZZ_p$.
  Using the forking lemma, give a \ppt algorithm $\bdv$ that outputs $x$ and bound its success probability.
\end{exercise}

\ifsolutions
\begin{mysolution}
  We construct a \ppt algorithm $\bdv$ that runs the forking game and extracts $x$ from the outputs.
  If $\game{\algo{Fork}}{\algo{InpGen}}$ returns $(z, s, z', s') \neq \bot$, then we have:
  \begin{itemize}
    \item From the first execution: $(z, s) = \alpha$ where $s = z + T[z] \cdot x \bmod p$
    \item From the second execution: $(z', s') = \alpha'$ where $s' = z' + T'[z'] \cdot x \bmod p$
    \item The conditions $z = z'$ and $T[z] \neq T'[z]$ hold
  \end{itemize}
  
  Therefore: $s = z + T[z] \cdot x$ and $s' = z + T'[z] \cdot x \pmod{p}$.
  
  Subtracting yields $s - s' = (T[z] - T'[z]) \cdot x \pmod{p}$. Since $T[z] \neq T'[z]$ and $p$ is prime, we can compute:
  \[
  x = (s - s') \cdot (T[z] - T'[z])^{-1} \bmod p
  \]
  
  Since $\adv$ always succeeds, $\advantage{Single}{\adv, \algo{InpGen}} = 1$. By the forking lemma:
  \[
  \Pr[\bdv \text{ outputs } x] = \advantage{Fork}{\adv, \algo{InpGen}} \geq 1 \cdot \left(\frac{1}{q} - \frac{1}{2^\secpar}\right) = \frac{1}{q} - \frac{1}{2^\secpar}
  \]
\end{mysolution}
\fi

\begin{exercise}
  Consider an algorithm $\adv$ that queries $h_0 \gets \pcoracle{H}(0)$ and outputs $(0, \bot)$ if $h_0 \bmod 2 = 0$, otherwise outputs $(h_0, \bot)$. What is $\advantage{Fork}{\adv, \algo{InpGen}}$?
\end{exercise}

\ifsolutions
\begin{mysolution}
  First, note that $\advantage{Single}{\adv, \algo{InpGen}} = 1$ since $\adv$ always outputs a non-$\bot$ value. 
  
  The adversary only queries $\pcoracle{H}(0)$, so after the first execution, $T[0] = h_0$ and all other entries remain $\bot$.
  
  Case 1: $h_0 \bmod 2 = 0$ (happens with probability $\frac{1}{2}$)
  \begin{itemize}
    \item First execution: $\adv$ outputs $(0, \bot)$
    \item We set $T'[0] \sample S$ (refreshing the value at key 0)
    \item Second execution: $\adv$ queries $h_0' = T'[0]$ (the refreshed value)
    \item If $h_0' \bmod 2 = 0$: outputs $(0, \bot)$, so $z = z' = 0$ and $T[0] = h_0 \neq h_0' = T'[0]$ with probability $1 - \frac{1}{2^\secpar}$
    \item If $h_0' \bmod 2 = 1$: outputs $(h_0', \bot)$, so $z = 0 \neq h_0' = z'$ (fork fails)
    \item Fork succeeds with probability $\frac{1}{2} \cdot (1 - \frac{1}{2^\secpar})$
  \end{itemize}
  
  Case 2: $h_0 \bmod 2 = 1$ (happens with probability $\frac{1}{2}$)
  \begin{itemize}
    \item First execution: $\adv$ outputs $(h_0, \bot)$
    \item Since $h_0$ was never queried, $T[h_0] = \bot$
    \item We set $T'[h_0] \sample S$
    \item Second execution: $\adv$ queries $h_0' = T'[0] = T[0] = h_0$ (unchanged)
    \item Since $h_0 \bmod 2 = 1$, outputs $(h_0, \bot)$
    \item So $z = z' = h_0$ and $T[h_0] = \bot \neq T'[h_0]$
    \item Fork succeeds with probability 1
  \end{itemize}
  
  Therefore:
  \[
  \advantage{Fork}{\adv, \algo{InpGen}} = \frac{1}{2} \cdot \frac{1}{2} \cdot \left(1 - \frac{1}{2^\secpar}\right) + \frac{1}{2} \cdot 1 = \frac{1}{4} - \frac{1}{2^{\secpar+2}} + \frac{1}{2} \approx \frac{3}{4}
  \]
\end{mysolution}
\fi

\begin{exercise}[Optional]
  Find an algorithm $\adv$ making exactly $q$ queries for which $\advantage{Fork}{\adv, \algo{InpGen}}$ is as close to $\advantage{Single}{\adv, \algo{InpGen}} \left( \frac{\advantage{Single}{\adv, \algo{InpGen}}}{q} - \frac{1}{2^\secpar} \right)$ as possible.
\end{exercise}

\begin{exercise}[Optional]
  Compare the Generalized Forking Lemma of Bellare and Neven~\cite{CCS:BelNev06} with the Local Forking Lemma.
  What are the key differences in their formulations and applications?
\end{exercise}


\section{Signatures}\label{sec:signatures}

\subsection{Syntax \& Correctness}

\begin{definition}[Signature Scheme]
  A signature scheme $\algo{Sig}$ is a tuple of polynomial-time algorithms $(\algo{Setup}, \algo{KeyGen}, \algo{Sign}, \algo{Verify})$ where:
  \begin{itemize}
    \item $\algo{Setup}(1^\secpar) \rightarrow \params$ is a probabilistic algorithm that takes the security parameter as input and outputs public parameters $\params$ including the message space $\mathcal{M}$ and the signature space $\mathcal{S}$.
    \item $\algo{KeyGen}(\params) \rightarrow (\sk, \pk)$ is a probabilistic algorithm that takes public parameters $\params$ as input and outputs a secret key $\sk$ and a public key $\pk$.
    \item $\algo{Sign}(\params, \sk, m) \rightarrow \sigma$ is a probabilistic algorithm that takes public parameters $\params$, a secret key $\sk$, and a message $m \in \mathcal{M}$ as input and outputs a signature $\sigma \in \mathcal{S}$.
    \item $\algo{Verify}(\params, \pk, m, \sigma) \rightarrow \{0, 1\}$ is a deterministic algorithm that takes public parameters $\params$, a public key $\pk$, a message $m \in \mathcal{M}$, and a signature $\sigma \in \mathcal{S}$ as input and outputs $1$ (accept) if the signature is valid, and outputs $0$ (reject) otherwise.
  \end{itemize}
\end{definition}

\begin{definition}[Correctness]\label{def:sig-correctness}
  A signature scheme $\algo{Sig} = (\algo{Setup}, \algo{KeyGen}, \algo{Sign}, \algo{Verify})$ is \emph{correct} if for all $\secpar \in \NN$, all $\params \in \algo{Setup}(1^\secpar)$, all $(\sk, \pk) \in \algo{KeyGen}(\params)$, and all messages $m \in \mathcal{M}$:
  \[
    \Pr[\algo{Verify}(\params, \pk, m, \algo{Sign}(\params, \sk, m)) = 1] = 1
  \]
  where the probability is taken over the randomness of $\algo{Sign}$.
\end{definition}

In other words, a signature scheme is correct if honestly generated signatures always verify correctly.

\subsection{Security}

The standard security notion for digital signatures is existential unforgeability under chosen message attack (EUF-CMA).
Unlike one-time signatures, the adversary can request signatures on multiple messages.

\begin{definition}[EUF-CMA Security]\label{def:euf-cma}
  A signature scheme $\algo{Sig}$ is \emph{existentially unforgeable under chosen message attack (EUF-CMA)} if for all \ppt adversaries $\adv$:
  \[
    \advantage{EUF-CMA}{\adv, \algo{Sig}} \defeq \pr{\game{\Game~\algo{EUF-CMA}}{\algo{Sig}} = 1} = \negl
  \]
  where $\game{\Game~\algo{EUF-CMA}}{\algo{Sig}}$ is defined in \autoref{fig:euf-cma-sig}.
\end{definition}

\begin{figure}[tbh]
  \begin{tcolorbox}
    \begin{pchstack}[center]
      \procedure{$\game{\Game~\algo{EUF-CMA}}{\algo{Sig}}$}{%
        \params \gets \algo{Setup}(1^\secpar) \\
        (\sk,\pk) \gets \algo{KeyGen}(\params) \\
        \mathcal{Q} \defeq \emptyset\\
        (m^*, \sigma^*) \gets \adv^{\pcoracle{Sign}}(\params, \pk) \\
        \pcreturn m^* \notin \mathcal{Q} \wedge \\
        \t \algo{Verify}(\params, \pk, m^*, \sigma^*) = 1
      }
      \pchspace
      \procedure{Oracle $\pcoracle{Sign}(m)$}{%
        \sigma \gets \algo{Sign}(\params, \sk, m) \\
        \mathcal{Q} \defeq \mathcal{Q} \cup \{m\} \\
        \pcreturn \sigma
      }
    \end{pchstack}
  \end{tcolorbox}
  \caption{The EUF-CMA security game for digital signatures}
  \label{fig:euf-cma-sig}
\end{figure}

\subsection{Schnorr Signatures}

\begin{definition}[Schnorr Signature Scheme]\label{def:schnorr}
  Let $\grgen$ be a group generation algorithm and $\algo{H}$ be a hash function with output space $\ZZ_p$. The Schnorr signature scheme $\algo{SchnorrSig}[\grgen, \algo{H}]$ is defined as follows:
  \begin{itemize}
    \item $\algo{Setup}(1^\secpar) \rightarrow \params$: Run $\gparam \gets \grgen(1^\secpar)$ and $\kappa \gets \algo{H.Gen}(1^\secpar)$. The message space is $\mathcal{M} = \{0,1\}^*$ and the signature space is $\mathcal{S} = \GG \times \ZZ_p$. Output $\params = (\gparam, \kappa, \mathcal{M}, \mathcal{S})$.
    
    \item $\algo{KeyGen}(\params) \rightarrow (\sk, \pk)$: Sample $x \sample \ZZ_p$. Compute $X = g^x$. Output $\sk = x$ and $\pk = X$.
    
    \item $\algo{Sign}(\params, \sk, m) \rightarrow \sigma$: Parse $\params = (\gparam, \kappa, \mathcal{M}, \mathcal{S})$ and $\sk = x$. Sample $r \sample \ZZ_p$. Compute:
    \begin{align*}
      R &= g^r \\
      c &= \algo{H.Eval}(\kappa, (R, m)) \\
      s &= r + c \cdot x \bmod p
    \end{align*}
    Output $\sigma = (R, s)$.
    
    \item $\algo{Verify}(\params, \pk, m, \sigma) \rightarrow \{0, 1\}$: Parse $\params = (\gparam, \kappa, \mathcal{M}, \mathcal{S})$, $\pk = X$ and $\sigma = (R, s)$. Compute $c = \algo{H.Eval}(\kappa, (R, m))$. Accept if and only if:
    \[
      g^s = R \cdot X^c
    \]
  \end{itemize}
\end{definition}

\begin{lemma}[Correctness of Schnorr Signatures]\label{lem:schnorr-correctness}
  The Schnorr signature scheme\\ $\algo{SchnorrSig}[\grgen, \algo{H}]$ is correct.
\end{lemma}

\begin{proof}
  Let $\sigma = (R, s)$ be a signature generated by $\algo{Sign}(\params, \sk, m)$ where $\sk = x$ and $\pk = X = g^x$. By the signing algorithm, we have $R = g^r$ for some $r \sample \ZZ_p$, $c = \algo{H.Eval}(\kappa, (R, m))$, and $s = r + c \cdot x \bmod p$. The verification equation holds because:
  \[
    g^s = g^{r + cx} = g^r \cdot g^{cx} = R \cdot (g^x)^c = R \cdot X^c
  \]
  Therefore, $\algo{Verify}(\params, \pk, m, \sigma) = 1$ for all honestly generated signatures.
\end{proof}

\begin{remark}
  In the security analysis that follows, we work in the Random Oracle Model where the hash function $\algo{H.Eval}(\kappa, \cdot)$ is modeled as a truly random function.
  In the ROM analysis, we write $\algo{H}(\cdot)$ instead of $\algo{H.Eval}(\kappa, \cdot)$, with the understanding that $\algo{H}$ represents a random oracle.
\end{remark}

\begin{figure}[tbhp]
  \begin{center}
    \begin{tcolorbox}[width=\textwidth]
      \begin{pchstack}[center]
        \begin{pcvstack}
          \procedure[headlinesep=1pt]{$\game{\Game~\algo{EUF-CMA}}{\algo{SchnorrSig}}$}{%
            \gparam \gets \grgen(1^\secpar) \\
            x \sample \ZZ_p \\
            X \defeq g^x \\
            \params \defeq (\gparam, \{0,1\}^*, \GG \times \ZZ_p) \\
            T \defeq \emptyset \\
            \mathcal{Q} \defeq \emptyset \\
            (m^*, \sigma^*) \gets \adv^{\pcoracle{Sign}, \pcoracle{H}}(\params, X) \\
            (R^*, s^*) \defeq \sigma^* \\
            c^* \defeq \pcoracle{H}(R^*, m^*) \\
            \pcreturn m^* \notin \mathcal{Q} \wedge g^{s^*} = R^* \cdot X^{c^*}
          }
          \pcvspace
          \procedure[headlinesep=1pt]{Oracle $\pcoracle{Sign}(m)$}{%
            r \sample \ZZ_p \\
            R \defeq g^r \\
            c \defeq \pcoracle{H}(R, m) \\
            s \defeq r + c \cdot x \bmod p \\
            \mathcal{Q} \defeq \mathcal{Q} \cup \{m\} \\
            \pcreturn (R, s)
          }
          \pcvspace
          \procedure[headlinesep=1pt]{Oracle $\pcoracle{H}(y)$}{%
            \pcif T[y] = \bot \pcthen \\
            \t T[y] \sample \ZZ_p \\
            \pcreturn T[y]
          }
        \end{pcvstack}
        \pchspace
        \begin{pcvstack}
          \procedure[headlinesep=1pt]{$\bdv(\gamechange{$\gparam, X;$} \rho)$}{%
            \gamechange{$\params \defeq (\gparam, \{0,1\}^*, \GG \times \ZZ_p)$} \\
            T \defeq \emptyset \\
            \mathcal{Q} \defeq \emptyset \\
            (m^*, \sigma^*) \gets \adv^{\pcoracle{Sign}_{\bdv}, \pcoracle{H}_{\bdv}}(\params, X; \rho) \\
            (R^*, s^*) \defeq \sigma^* \\
            c^* \defeq \pcoracle{H}_{\bdv}((R^*, m^*)) \\
            \gamechange{$\pcassert m^* \notin \mathcal{Q}$} \\
            \gamechange{$\pcassert g^{s^*} = R^* \cdot X^{c^*}$} \\
            \gamechange{$\pcreturn ((R^*, m^*), s^*)$}
          }
          \pcvspace
          \procedure[headlinesep=1pt]{Oracle $\pcoracle{Sign}_{\bdv}(m)$}{%
            \gamechange{$s \sample \ZZ_p$} \\
            \gamechange{$c \sample \ZZ_p$} \\
            \gamechange{$R \defeq g^s \cdot X^{-c}$} \\
            \gamechange{$\pcassert T[R, m] = \bot$} \\
            \gamechange{$T[R, m] \defeq c$} \\
            \mathcal{Q} \defeq \mathcal{Q} \cup \{m\} \\
            \pcreturn (R, s)
          }
          \pcvspace
          \procedure[headlinesep=1pt]{Oracle $\pcoracle{H}_{\bdv}(y)$}{%
            \pcif T[y] = \bot \pcthen \\
            \t T[y] \sample \ZZ_p \\
            \pcreturn T[y]
          }
        \end{pcvstack}
        \pchspace
        \begin{pcvstack}
          \procedure[headlinesep=1pt]{$\cdv^{\gamechange{$\pcoracle{H}$}}(\gparam, X; \rho)$}{%
            \params \defeq (\gparam, \{0,1\}^*, \GG \times \ZZ_p) \\
            T \defeq \emptyset \\
            \mathcal{Q} \defeq \emptyset \\
            (m^*, \sigma^*) \gets \adv^{\pcoracle{Sign}_{\cdv}, \pcoracle{H}_{\cdv}}(\params, X; \rho) \\
            (R^*, s^*) \defeq \sigma^* \\
            c^* \defeq \pcoracle{H}_{\cdv}((R^*, m^*)) \\
            \pcassert m^* \notin \mathcal{Q} \\
            \pcassert g^{s^*} = R^* \cdot X^{c^*} \\
            \pcreturn ((R^*, m^*), s^*)
          }
          \pcvspace
          \procedure[headlinesep=1pt]{Oracle $\pcoracle{Sign}_{\cdv}(m)$}{%
            s \sample \ZZ_p \\
            c \sample \ZZ_p \\
            R \defeq g^s \cdot X^{-c} \\
            \pcassert T[R, m] = \bot \\
            T[R, m] \defeq c \\
            \mathcal{Q} \defeq \mathcal{Q} \cup \{m\} \\
            \pcreturn (R, s)
          }
          \pcvspace
          \procedure[headlinesep=1pt]{Oracle $\pcoracle{H}_{\cdv}(y)$}{%
            \pcif T[y] = \bot \pcthen \\
            \gamechange{$\t T[y] \gets \pcoracle{H}(y)$} \\
            \pcreturn T[y]
          }
        \end{pcvstack}
      \end{pchstack}
    \end{tcolorbox}
  \end{center}
  \caption{Security reduction for Schnorr signatures: EUF-CMA game (left), algorithm $\bdv$ (middle), and algorithm $\cdv$ (right). Changes from the previous game are marked with $\gamechange{$\cdot$}$.}
  \label{fig:schnorr-reduction}
\end{figure}

\begin{lemma}\label{lem:schnorr-to-single}
  Let $\adv$ be an adversary against the EUF-CMA security of $\algo{SchnorrSig}[\grgen, \algo{H}]$ in the random oracle model for $\algo{H}$ making at most $q_s$ queries to $\pcoracle{Sign}$ and at most $q_h$ queries to $\algo{H}$.

  Let $\algo{InpGen}$ be the algorithm which on input $1^\secpar$ runs $(\GG, p, g) \gets \grgen(1^\secpar)$, draws $X \sample \GG$, and returns $((\GG, p, g), X)$.

  Consider algorithm $\cdv$ defined in \autoref{fig:schnorr-reduction} that takes as input $(\gparam, X) \gets \algo{InpGen}(1^\secpar)$ and has access to a random oracle $\algo{H}$. Then $\cdv$ makes at most $q_h + 1$ queries to $\algo{H}$, and satisfies
  \[
    \left|\advantage{Single}{\cdv, \algo{InpGen}} - \advantage{EUF-CMA}{\adv, \algo{SchnorrSig}[\grgen, \algo{H}]}\right| \leq \frac{2 q_s(q_s + q_h)}{2^\secpar}
  \]
  with $\advantage{Single}{\cdv, \algo{InpGen}}$ as defined in the forking lemma (\autoref{lem:fork}).

  Moreover, when $\cdv$ returns a non-$\bot$ output $(z, \mathit{aux})$, then $z = (R^*, m^*)$, $\mathit{aux} = s^*$, and
  \[
    g^{s^*} = R^* \cdot X^{\algo{H}(R^*, m^*)}
  \]
\end{lemma}

The proof is left as an exercise (see \autoref{ex:schnorr-to-single}).

\begin{theorem}[Schnorr signatures are EUF-CMA secure]\label{thm:schnorr-euf-cma}
  Let $\grgen$ be a group generation algorithm for which the discrete logarithm problem is hard and let $\algo{H}$ be a hash function with output space $\ZZ_p$.
  Then the Schnorr signature scheme $\algo{SchnorrSig}[\grgen, \algo{H}]$ is EUF-CMA secure in the random oracle model for $\algo{H}$.
  More precisely, for any \ppt adversary $\adv$ against the EUF-CMA security of $\algo{SchnorrSig}[\grgen, \algo{H}]$ making at most $q_s$ queries to the signing oracle and at most $q_h$ queries to the random oracle $\algo{H}$,
  there exists a \ppt adversary $\ddv$ against the discrete logarithm problem such that
  \[
    \advantage{EUF-CMA}{\adv, \algo{SchnorrSig}[\grgen, \algo{H}]} \leq \frac{2 q_s(q_s + q_h)}{2^\secpar} + \sqrt{(q_h + 1) \left(\advantage{DLog}{\ddv, \grgen} + \frac{1}{2^\secpar}\right)}.
  \]
\end{theorem}

The proof is left as an exercise (see \autoref{ex:schnorr-euf-cma}).

\subsection{Exercises}

\begin{exercise}
  The EUF-CMA security definition in \autoref{def:euf-cma} only requires the forged message to be new. An alternative notion called \emph{strong EUF-CMA} would prevent the adversary from winning even if they produce a different signature for a message they previously queried. Define a strong EUF-CMA security game that captures this intuition.
\end{exercise}

\ifsolutions
\begin{mysolution}
  For strong EUF-CMA security, we modify the signing oracle to maintain a set of message-signature pairs:
  \begin{itemize}
    \item Initialize $\mathcal{Q} \defeq \emptyset$
    \item In oracle $\pcoracle{Sign}(m)$: compute $\sigma \gets \algo{Sign}(\params, \sk, m)$, set $\mathcal{Q} \defeq \mathcal{Q} \cup \{(m, \sigma)\}$, and return $\sigma$
    \item Winning condition: $(m^*, \sigma^*) \notin \mathcal{Q} \wedge \algo{Verify}(\params, \pk, m^*, \sigma^*) = 1$
  \end{itemize}

  The key difference from standard EUF-CMA is that $\mathcal{Q}$ stores pairs $(m, \sigma)$ rather than just messages, and the adversary must output a pair that was never returned by the signing oracle, even if the message component was queried before.
\end{mysolution}
\fi


\begin{exercise}
  In the Generic Group Model (GGM)~\cite{EC:Shoup97}, the advantage of solving the discrete logarithm problem with $q$ group operations is bounded by $\advantage{DLog}{\adv, \grgen} \leq \frac{q^2}{2p}$ where $p$ is the prime group order.
  
  Consider Schnorr signatures over a 256-bit elliptic curve group (e.g., secp256k1 where $p \approx 2^{256}$). Suppose an adversary can make:
  \begin{itemize}
    \item $q_s = 2^{10}$ signing queries
    \item $q_h = 2^{85}$ hash queries
    \item $q = 2^{85}$ group operations
  \end{itemize}
  
  Calculate the concrete EUF-CMA advantage using the bound from \autoref{thm:schnorr-euf-cma} with $\secpar = 256$, and compare it to the discrete logarithm advantage $\advantage{DLog}{\adv, \grgen}$ in the GGM.
\end{exercise}

\ifsolutions
\begin{mysolution}
  From \autoref{thm:schnorr-euf-cma}:
  \[
    \advantage{EUF-CMA}{\adv, \algo{SchnorrSig}} \leq \frac{q_s(q_s + q_h)}{2^\secpar} + \sqrt{(q_h + 1) \left(\advantage{DLog}{\ddv, \grgen} + \frac{1}{2^\secpar}\right)}
  \]
  
  First, let's calculate the discrete logarithm advantage in the GGM:
  \[
    \advantage{DLog}{\adv, \grgen} \leq \frac{q^2}{2p} = \frac{(2^{85})^2}{2^{257}} = \frac{2^{170}}{2^{257}} = 2^{-87}
  \]
  
  Now for the EUF-CMA advantage. First term:
  \[
    \frac{q_s(q_s + q_h)}{2^\secpar} \approx \frac{2^{10} \cdot 2^{85}}{2^{256}} = \frac{2^{95}}{2^{256}} = 2^{-161}
  \]
  
  Second term:
  \[
    \sqrt{(q_h + 1) \left(\advantage{DLog}{\ddv, \grgen} + \frac{1}{2^\secpar}\right)} \approx \sqrt{2^{85} \cdot 2^{-87}} = \sqrt{2^{-2}} = 2^{-1} = 0.5
  \]
  
  Therefore:
  \[
    \advantage{EUF-CMA}{\adv, \algo{SchnorrSig}} \leq 2^{-161} + 0.5 \approx 0.5
  \]
  
  \textbf{Comparison:} The discrete logarithm advantage is $2^{-87}$, while the EUF-CMA advantage for Schnorr signatures is approximately $2^{-1}$.
  
  This dramatic difference—losing over 86 bits of security—is due to the square root loss in the forking lemma. 
  
  \textbf{Remark:} In the Algebraic Group Model (AGM)~\cite{C:FucKilLos18}, Schnorr signatures can be proven secure with a much tighter reduction. The AGM assumes that whenever an adversary outputs a group element, it must also provide an algebraic representation in terms of previously seen group elements. Because the AGM proof doesn't require rewinding (unlike the forking lemma), it avoids the square root loss entirely, achieving security that nearly matches the hardness of discrete logarithm.
\end{mysolution}
\fi

\begin{exercise}\label{ex:schnorr-to-single}
  Prove \autoref{lem:schnorr-to-single} using the games in \autoref{fig:schnorr-reduction}.
\end{exercise}

\ifsolutions
\begin{mysolution}
  We prove the lemma by a sequence of game hops between the three games in \autoref{fig:schnorr-reduction}.

  \paragraph{From $\Game~\algo{EUF\text{-}CMA}$ to $\game{\algo{Single}}{\algo{InpGen}}[\bdv]$}
  The main differences are:
  \begin{itemize}
    \item $\bdv$ receives $((\GG, p, g), X)$ from $\algo{InpGen}(1^\secpar)$ instead of generating the key pair $(x, X)$.
    \item The signing oracle $\pcoracle{Sign}_{\bdv}$ simulates signing without knowing $x$ by programming the random oracle.
    \item $\bdv$ explicitly asserts the winning conditions of $\Game~\algo{EUF\text{-}CMA}$, and outputs $((R^*, m^*), s^*)$.
  \end{itemize}

  Let $E$ be the event that $\bdv$ aborts during a signing oracle query(i.e., $T_{\bdv}[R  m] \neq \bot$).
  We now prove that that $\pr{\game{\Game~\algo{EUF-CMA}}{\algo{SchnorrSig}} = 1}$ is equal to $\pr{\game{\algo{Single}}{\algo{InpGen}}[\bdv] = 1}$ if $E$ does not occur.
  \begin{itemize}
    \item $\algo{InpGen}(1^\secpar)$ outputs $((\GG, p, g), X)$ where $X \sample \GG$, which has the same distribution as the key generation in $\game{\Game~\algo{EUF-CMA}}{\algo{SchnorrSig}}$ since $X = g^x$ for uniformly random $x \sample \ZZ_p$.
    \item The signatures output by the signing oracle have the same distribution in both games when event $E$ does not occur. To see why, let's examine how signatures are generated:
    \begin{itemize}
      \item In $\game{\Game~\algo{EUF-CMA}}{\algo{SchnorrSig}}$: Sample $r \sample \ZZ_p$, compute $R = g^r$, set $c = \algo{H}(R, m)$, and output $(R,
  s)$ where $s = r + x \cdot c$
      \item In $\bdv$: Sample $(c, s) \sample \ZZ_p \times \ZZ_p$, compute $R = g^s \cdot X^{-c}$, program $T_{\bdv}[(R, m)] = c$, and output $(R, s)$
    \end{itemize}

    Both methods produce signatures satisfying the verification equation $g^s = R \cdot X^c$. The key insight is that:
    \begin{itemize}
      \item In the EUF-CMA game: Starting from uniform $r$, we get uniform $R \in \GG$ and compute $s$ deterministically
      \item In $\bdv$: Starting from uniform $(c, s)$, we compute $R$ deterministically to satisfy the same equation
    \end{itemize}

    Since there's a bijection between the randomness and valid signatures in both cases, and the oracle programming ensures $c = \algo{H}(R, m)$ when
  queried, both games produce identical signature distributions when $E$ does not occur.
  \item The programmed oracle $\pcoracle{H}_{\bdv}$ is indistinguishable from the real random oracle $\pcoracle{H}$ when $E$ does not occur. The oracle
  $\pcoracle{H}_{\bdv}$ returns either programmed values (for queries $(R, m)$ where $(R, s)$ was output by $\pcoracle{Sign}_{\bdv}(m)$) or fresh random
  values from $\pcoracle{H}$. Since programmed values are chosen uniformly at random when signatures are created, all oracle responses are uniformly
  distributed in $\ZZ_p$. The adversary can only distinguish between $\pcoracle{H}_{\bdv}$ and $\pcoracle{H}$ if it queries $\pcoracle{H}_{\bdv}((R, m))$
  before $\pcoracle{Sign}_{\bdv}(m)$ happens to generate the same $R$ – but this is precisely event $E$.
  \end{itemize}

  Thus we conclude that
  \[
    \pr{\game{\Game~\algo{EUF-CMA}}{\algo{SchnorrSig}} = 1 \wedge \neg E} =  \pr{\game{\algo{Single}}{\algo{InpGen}}[\bdv] = 1 \wedge \neg E}
  \] and by the Difference Lemma (\autoref{lem:difference}) we have
  \[
    |\pr{\game{\Game~\algo{EUF-CMA}}{\algo{SchnorrSig}} = 1} - \pr{\game{\algo{Single}}{\algo{InpGen}}[\bdv] = 1}| \leq \Pr[E]
  \].

  To bound the probability of event $E$, we first observe that when event $E$ occurs there exists a key $(R', m')$ in $T_{\bdv}$ such that $R'$ is equal to value $R$ computed in the signing oracle.
  The probability that $R$ equals a specific $R'$ is $1/|\GG|$ since $R$ is uniformly distributed in $\GG$: for any $R \in \GG$ and any $s \in \ZZ_p$, there is exactly one $c \in \ZZ_p$ such that $R = g^s \cdot X^{-c}$.
  Moreover, at the time of each signing query, the table $T_{\bdv}$ contains at most $q_s + q_h$ entries (from previous signing queries and hash queries).
  Therefore, the probability that $(R, m)$ matches any existing entry in $T_{\bdv}$ is at most $\frac{q_s + q_h}{|\GG|}$.

  By union bound over all $q_s$ signing queries:
  \[
    \Pr[E] \leq \frac{q_s(q_s + q_h)}{|\GG|} \leq \frac{2 q_s(q_s + q_h)}{2^\secpar}
  \]
  where we used $|\GG| = p \geq 2^{\secpar-1}$.

  \paragraph{From Algorithm $\bdv$ to Algorithm $\cdv$}
  The only difference between $\bdv$ and $\cdv$ is that $\cdv$ has access to an external random oracle $\pcoracle{H}$, and uses a modified oracle $\pcoracle{H}_{\cdv}$ that:
  \begin{itemize}
    \item Returns the programmed value $T[y]$ if it exists
    \item Otherwise queries the external oracle $\pcoracle{H}(y)$
  \end{itemize}

  This is a purely syntactic change. In $\bdv$, non-programmed queries sample fresh randomness directly; in $\cdv$, they get it from the external oracle.
  Since both sources provide independent uniform values from $\ZZ_p$, the adversary's view is identical. Therefore:
  \[
    \Pr[\game{\algo{Single}}{\algo{InpGen}}[\bdv] = 1] = \Pr[\game{\algo{Single}}{\algo{InpGen}}[\cdv] = 1]
  \]

  \paragraph{Conclusion.}
  Finally, we need to prove that when $\cdv$ succeeds (i.e., doesn't abort), the output satisfies $g^{s^*} = R^* \cdot X^{\algo{H}(R^*, m^*)}$ where $\algo{H}$ is the external oracle. We know that $g^{s^*} = R^* \cdot X^{c^*}$ where $c^* = \pcoracle{H}_{\cdv}(R^*, m^*)$. We claim that $\pcoracle{H}_{\cdv}(R^*, m^*) = \pcoracle{H}(R^*, m^*)$.
  
  Assume for contradiction that $\pcoracle{H}_{\cdv}(R^*, m^*) \neq \pcoracle{H}(R^*, m^*)$. This means $T[R^*, m^*] \neq \pcoracle{H}(R^*, m^*)$, so $T[R^*, m^*]$ must have been assigned during a $\pcoracle{Sign}_{\cdv}$ query. But whenever $\pcoracle{Sign}_{\cdv}$ sets $T[R, m] = c$, it also adds $m$ to $\mathcal{Q}$. Therefore $m^* \in \mathcal{Q}$. However, $\cdv$ asserts that $m^* \notin \mathcal{Q}$, which is a contradiction. Thus $c^* = \pcoracle{H}(R^*, m^*)$.
  
  Combining the bounds:
  \[
    \left|\advantage{Single}{\cdv, \algo{InpGen}} - \advantage{EUF-CMA}{\adv, \algo{SchnorrSig}[\grgen, \algo{H}]}\right| \leq \frac{2 q_s(q_s + q_h)}{2^\secpar}
  \]

\end{mysolution}
\fi

\begin{exercise}\label{ex:schnorr-euf-cma}
  Prove \autoref{thm:schnorr-euf-cma}. Use \autoref{lem:schnorr-to-single} and the forking lemma (\autoref{lem:fork}).
\end{exercise}

\ifsolutions
\begin{mysolution}
  We construct a \ppt algorithm $\ddv$ that solves the discrete logarithm problem using a \ppt EUF-CMA adversary $\adv$ against Schnorr signatures.
  
  Given a discrete logarithm instance $((\GG, p, g), X)$ where $X = g^x$ for unknown $x$, algorithm $\ddv$ works as follows:
  \begin{enumerate}
    \item Run the forking game $\game{\algo{Fork}}{\algo{InpGen}}[\cdv]$ with algorithm $\cdv$ from \autoref{lem:schnorr-to-single}
    \item If the forking game returns $((R^*, m^*), s_1, (R^*, m^*), s_2) \neq \bot$, then we have:
    \begin{itemize}
      \item From the first execution: By \autoref{lem:schnorr-to-single}, we have $g^{s_1} = R^* \cdot X^{h_1}$ where $h_1 = \algo{H}(R^*, m^*)$
      \item From the second execution: By \autoref{lem:schnorr-to-single}, we have $g^{s_2} = R^* \cdot X^{h_2}$ where $h_2 = \algo{H}'(R^*, m^*)$
      \item By the assertion $T[z] \neq T'[z]$ in $\game{\algo{Fork}}{\algo{InpGen}}$ with $z = (R^*, m^*)$: we have $h_1 \neq h_2$
    \end{itemize}
    \item From the two verification equations:
    \[
      g^{s_1} = R^* \cdot X^{h_1} \text{ and } g^{s_2} = R^* \cdot X^{h_2}
    \]
    Dividing the first by the second:
    \[
      g^{s_1 - s_2} = X^{h_1 - h_2}
    \]
    \item Since $h_1 \neq h_2$ and we work in $\ZZ_p$, we can compute:
    \[
      x = \frac{s_1 - s_2}{h_1 - h_2} \bmod p
    \]
    \item Output $x$
  \end{enumerate}
  
  \textbf{Success Probability:} Algorithm $\ddv$ succeeds in solving the discrete logarithm problem exactly when the forking game returns non-$\bot$ outputs. Therefore:
  \[
    \advantage{DLog}{\ddv, \grgen} = \advantage{Fork}{\cdv, \algo{InpGen}}
  \]
  
  By the forking lemma:
  \begin{align*}
    \advantage{Fork}{\cdv, \algo{InpGen}} &\geq \advantage{Single}{\cdv, \algo{InpGen}} \left( \frac{\advantage{Single}{\cdv, \algo{InpGen}}}{q_h + 1} - \frac{1}{2^\secpar} \right)\\
    &\geq  \frac{\advantage{Single}{\cdv, \algo{InpGen}}^2}{q_h + 1} - \frac{1}{2^\secpar}\\
  \end{align*}
  because $\advantage{Single}{\cdv, \algo{InpGen}} \leq 1$.
  
  By \autoref{lem:schnorr-to-single}, we have:
  \[
    \left|\advantage{Single}{\cdv, \algo{InpGen}} - \advantage{EUF-CMA}{\adv, \algo{SchnorrSig}[\grgen, \algo{H}]}\right| \leq \frac{2 q_s(q_s + q_h)}{2^\secpar}
  \]
  
  This means:
  \[
    \advantage{EUF-CMA}{\adv, \algo{SchnorrSig}[\grgen, \algo{H}]} - \frac{2 q_s(q_s + q_h)}{2^\secpar} \leq \advantage{Single}{\cdv, \algo{InpGen}} \leq \advantage{EUF-CMA}{\adv, \algo{SchnorrSig}[\grgen, \algo{H}]} + \frac{2 q_s(q_s + q_h)}{2^\secpar}
  \]
  
  Therefore:
  \begin{align*}
    \advantage{DLog}{\ddv, \grgen} &\geq \frac{\advantage{Single}{\cdv, \algo{InpGen}}^2}{q_h + 1} - \frac{1}{2^\secpar}\\
    &\geq \frac{\left(\advantage{EUF-CMA}{\adv, \algo{SchnorrSig}[\grgen, \algo{H}]} - \frac{2 q_s(q_s + q_h)}{2^\secpar}\right)^2}{q_h + 1} - \frac{1}{2^\secpar}
  \end{align*}
  
  Rearranging:
  \[
    (q_h + 1) \left(\advantage{DLog}{\ddv, \grgen} + \frac{1}{2^\secpar}\right) \geq \left(\advantage{EUF-CMA}{\adv, \algo{SchnorrSig}[\grgen, \algo{H}]} - \frac{2 q_s(q_s + q_h)}{2^\secpar}\right)^2
  \]
  
  Therefore:
  \[
    \advantage{EUF-CMA}{\adv, \algo{SchnorrSig}[\grgen, \algo{H}]} \leq \frac{2 q_s(q_s + q_h)}{2^\secpar} + \sqrt{(q_h + 1) \left(\advantage{DLog}{\ddv, \grgen} + \frac{1}{2^\secpar}\right)}
  \]
  
  $q_s, q_h$ are polynomial in $\secpar$ since $\adv$ is \ppt
  Assuming the discrete logarithm problem is hard (i.e., $\advantage{DLog}{\ddv, \grgen}$ is negligible), we conclude that $\advantage{EUF-CMA}{\adv, \algo{SchnorrSig}[\grgen, \algo{H}]}$ is negligible.
\end{mysolution}
\fi

\begin{exercise}[Optional]
  Study the formalization of adaptor signatures in Gerhart, Schröder, Soni, and Thyagarajan~\cite{EC:GSST24}.
  Explain what gaps existed in previous formalizations and how this work addresses them.
\end{exercise}

\section{Signatures with Key Tweaking}\label{sec:signatures-key-tweaking}

Key tweaking is fundamental in hierarchical deterministic (HD) Bitcoin wallets~\cite{add:bip-hdwallets} and Taproot commitments~\cite{add:bip-taproot}, enabling the derivation of multiple keys from a single master key and the commitment of additional data to public keys.

\subsection{Syntax}

\begin{definition}[Tweaking Scheme for Signatures]
  A tweaking scheme $\algo{TwSch}$ for a signature scheme $\algo{Sig}$ is a tuple $(\mathcal{T}, \algo{TwSK}, \algo{TwPK})$ where:
  \begin{itemize}
    \item $\mathcal{T}$ is a set of allowed tweaks, parameterized by the public parameters $\params$ output by $\algo{Setup}$.
    \item $\algo{TwSK}(\sk, t) \rightarrow \tilde{\sk}$ is a deterministic polynomial-time algorithm that takes a secret key $\sk$ and a tweak $t \in \mathcal{T}$ as input and outputs a tweaked secret key $\tilde{\sk}$.
    \item $\algo{TwPK}(\pk, t) \rightarrow \tilde{\pk}$ is a deterministic polynomial-time algorithm that takes a public key $\pk$ and a tweak $t \in \mathcal{T}$ as input and outputs a tweaked public key $\tilde{\pk}$.
  \end{itemize}
\end{definition}

\subsection{Key Tweaking for Schnorr Signatures}

\begin{definition}[SchnorrTS]\label{def:schnorrts}
  The tweaking scheme $\algo{SchnorrTS}$ for Schnorr signatures is defined as follows:
  \begin{itemize}
    \item The tweak set is $\mathcal{T} \defeq \ZZ_p \times \ZZ_p^*$ where $\ZZ_p^* = \ZZ_p \setminus \{0\}$.
    \item For $(\alpha, \beta) \in \mathcal{T}$, the secret key tweaking algorithm is:
    \[
      \algo{TwSK}(x, (\alpha, \beta)) \defeq \alpha + \beta \cdot x \bmod p
    \]
    for every $x \in \ZZ_p$.
    \item For $(\alpha, \beta) \in \mathcal{T}$, the public key tweaking algorithm is:
    \[
      \algo{TwPK}(X, (\alpha, \beta)) \defeq g^\alpha \cdot X^\beta
    \]
    for every $X \in \GG$.
  \end{itemize}
\end{definition}

\begin{definition}[Schnorr Signature with Key Prefixing]\label{def:schnorr-kp}
  The Schnorr signature scheme with key prefixing $\algo{SchnorrSigKP}[\grgen, \algo{H}]$ is identical to $\algo{SchnorrSig}[\grgen, \algo{H}]$ except that the hash is computed as:
  \[
    c = \algo{H.Eval}(\kappa, (R, X, m))
  \]
  instead of $c = \algo{H.Eval}(\kappa, (R, m))$, where $X$ is the public key.
\end{definition}

\subsection{Security with Tweaked Keys}

The security notion of \emph{existential unforgeability under chosen message attack with tweaked keys (EUF-CMA-TK)} extends the standard EUF-CMA notion to capture security when an adversary can obtain signatures under tweaked keys.
Informally, the adversary is given a public key and can request signatures on messages of their choice, but now with an additional capability: for each signing query, they can specify a tweak in addition to the message.
The signing oracle will return a signature that verifies under the tweaked public key (i.e., a signature created with the correspondingly tweaked secret key).
The adversary wins by outputting $(m^*, t^*, \sigma^*)$ such that the signature verifies under the tweaked public key for $t^*$, and the pair $(m^*, t^*)$ was never queried to the signing oracle.

The formal definition of EUF-CMA-TK is left as an exercise (see \autoref{ex:euf-cma-tk-def}).

\begin{theorem}[SchnorrSigKP with SchnorrTS is EUF-CMA-TK secure]\label{thm:schnorrkp-euf-cma-tk}
  Let $\grgen$ be a group generation algorithm for which the discrete logarithm problem is hard and let $\algo{H}$ be a hash function with output space $\ZZ_p$.
  Then the Schnorr signature scheme with key prefixing $\algo{SchnorrSigKP}[\grgen, \algo{H}]$ with tweaking scheme $\algo{SchnorrTS}$ is EUF-CMA-TK secure in the random oracle model for $\algo{H}$.
\end{theorem}

The proof is left as an exercise (see \autoref{ex:schnorrkp-to-single}).

\subsection{Exercises}

\begin{exercise}
  Define correctness for a signature scheme with tweaking.
  Specifically, modify the standard signature scheme correctness definition (Definition~\ref{def:sig-correctness}) to ensure that signatures created with a tweaked secret key verify correctly under the corresponding tweaked public key.
\end{exercise}

\ifsolutions
\begin{mysolution}
  A signature scheme $\algo{Sig} = (\algo{Setup}, \algo{KeyGen}, \algo{Sign}, \algo{Verify})$ with tweaking scheme $\algo{TwSch} = (\mathcal{T}, \algo{TwSK}, \algo{TwPK})$ is \emph{correct with tweaking} if:
  \begin{enumerate}
    \item The signature scheme $\algo{Sig}$ is correct (as per the standard definition).
    \item For all $\secpar \in \NN$, all $\params \in \algo{Setup}(1^\secpar)$, all $(\sk, \pk) \in \algo{KeyGen}(\params)$, all tweaks $t \in \mathcal{T}$, and all messages $m \in \mathcal{M}$:
    \[
      \Pr[\algo{Verify}(\params, \algo{TwPK}(\pk, t), m, \algo{Sign}(\params, \algo{TwSK}(\sk, t), m)) = 1] = 1
    \]
    where the probability is taken over the randomness of $\algo{Sign}$.
  \end{enumerate}
  
  In other words, signatures created with a tweaked secret key must always verify correctly under the corresponding tweaked public key.
\end{mysolution}
\fi

\begin{exercise}[Optional]
  Argue that the tweaking scheme $\algo{SchnorrTS}$ (Definition~\ref{def:schnorrts}) is correct with the Schnorr signature scheme (Definition~\ref{def:schnorr}).
  That is, show that signatures created with a tweaked secret key verify correctly under the corresponding tweaked public key.
\end{exercise}

\ifsolutions
\begin{mysolution}
  For SchnorrTS with Schnorr signatures, if $\sk = x$ and $\pk = X = g^x$, then for tweak $(\alpha, \beta) \in \mathcal{T}$:
  \begin{itemize}
    \item Tweaked secret key: $\tilde{x} = \algo{TwSK}(x, (\alpha, \beta)) = \alpha + \beta \cdot x$
    \item Tweaked public key: $\tilde{X} = \algo{TwPK}(X, (\alpha, \beta)) = g^\alpha \cdot X^\beta = g^\alpha \cdot g^{\beta x} = g^{\alpha + \beta x} = g^{\tilde{x}}$
  \end{itemize}
  
  The key observation is that the discrete logarithm relationship is preserved: $\tilde{X} = g^{\tilde{x}}$.
  This means that $(\tilde{x}, \tilde{X})$ is a valid key pair that could have been output by $\algo{KeyGen}$.
  By the correctness of Schnorr signatures (Lemma~\ref{lem:schnorr-correctness}), signatures created with $\tilde{x}$ will verify under $\tilde{X}$.
\end{mysolution}
\fi

\begin{exercise}\label{ex:euf-cma-tk-def}
  Formalize the EUF-CMA-TK security notion by adapting the EUF-CMA security definition (Definition~\ref{def:euf-cma}).
  Your definition should capture the informal description given above.
\end{exercise}

\ifsolutions
\begin{mysolution}
  A signature scheme $\algo{Sig} = (\algo{Setup}, \algo{KeyGen}, \algo{Sign}, \algo{Verify})$ with tweaking scheme $\algo{TwSch} = (\mathcal{T}, \algo{TwSK}, \algo{TwPK})$ is \emph{existentially unforgeable under chosen message attack with tweaking (EUF-CMA-TK)} if for all \ppt adversaries $\adv$:
  \[
    \advantage{EUF-CMA-TK}{\adv, \algo{Sig}, \algo{TwSch}} \defeq \pr{\game{\Game~\algo{EUF-CMA-TK}}{\algo{Sig}, \algo{TwSch}} = 1} = \negl
  \]
  where $\game{\Game~\algo{EUF-CMA-TK}}{\algo{Sig}, \algo{TwSch}}$ is defined as follows:

  \begin{figure}[htb]
    \begin{center}
      \begin{tcolorbox}
        \begin{pchstack}[center]
          \procedure{$\game{\Game~\algo{EUF-CMA-TK}}{\algo{Sig}, \algo{TwSch}}$}{%
            \params \gets \algo{Setup}(1^\secpar) \\
            (\sk,\pk) \gets \algo{KeyGen}(\params) \\
            \mathcal{Q} \defeq \emptyset \\
            (m^*, \gamechange{$t^*$}, \sigma^*) \gets \adv^{\pcoracle{Sign}}(\params, \pk) \\
            \pcreturn \gamechange{$(m^*, t^*)$} \notin \mathcal{Q} \wedge \\
            \t \algo{Verify}(\params, \gamechange{$\algo{TwPK}(\pk, t^*)$}, m^*, \sigma^*) = 1
          }
          \pchspace
          \procedure{Oracle $\pcoracle{Sign}(m, \gamechange{$t$})$}{%
            \gamechange{$\tilde{\sk} \gets \algo{TwSK}(\sk, t)$} \\
            \sigma \gets \algo{Sign}(\params, \gamechange{$\tilde{\sk}$}, m) \\
            \mathcal{Q} \defeq \mathcal{Q} \cup \{\gamechange{$(m, t)$}\} \\
            \pcreturn \sigma
          }
        \end{pchstack}
      \end{tcolorbox}
    \end{center}
    \caption{The EUF-CMA-TK security game for signatures with tweaking. Changes from the standard EUF-CMA game are highlighted.}
    \label{fig:euf-cma-tk}
  \end{figure}
\end{mysolution}
\fi

\begin{exercise}
  Give a \ppt adversary that breaks the EUF-CMA-TK security of Schnorr signatures with the SchnorrTS tweaking scheme. Hints:
  \begin{itemize}
    \item You only need to use the signing oracle once.
    \item Pick a tweak $t$ and look at the Schnorr verification equation for a signature $(R, s)$ under public key $X$. Compute $s'$ by adding a value to $s$ such that $(R, s')$ is a valid signature for $\algo{TwPK}(X, t)$.
  \end{itemize}
\end{exercise}

\ifsolutions
\begin{mysolution}
  The adversary $\adv$ that breaks EUF-CMA-TK security is shown in Figure~\ref{fig:adversary-euf-cma-tk}.

  \begin{figure}[htb]
    \begin{center}
      \begin{tcolorbox}[width=7cm]
        \begin{pchstack}[center]
          \procedure{Adversary $\adv^{\pcoracle{Sign}}(\params, X)$}{%
            m \sample \{0,1\}^* \\
            \tau_0 \defeq (0, 1) \\
            (R, s) \gets \pcoracle{Sign}(m, \tau_0) \\
            c \defeq \algo{H}(R, m) \\
            s' \defeq s + c \\
            \tau^* \defeq (1, 1) \\
            \pcreturn (m, \tau^*, (R, s'))
          }
        \end{pchstack}
      \end{tcolorbox}
    \end{center}
    \caption{Adversary breaking EUF-CMA-TK security of Schnorr signatures with SchnorrTS tweaking.}
    \label{fig:adversary-euf-cma-tk}
  \end{figure}

  \textbf{Why this attack succeeds:}
  
  When querying the signing oracle with $\tau_0 = (0, 1)$:
  \begin{itemize}
    \item The tweaked public key is $\algo{TwPK}(X, (0, 1)) = g^0 \cdot X^1 = X$ (the original key).
    \item The signature $(R, s)$ satisfies: $g^s = R \cdot X^c$ where $c = \algo{H}(R, m)$.
  \end{itemize}
  
  For the forgery with $\tau^* = (1, 1)$:
  \begin{itemize}
    \item The tweaked public key is $\algo{TwPK}(X, (1, 1)) = g^1 \cdot X = g \cdot X$.
    \item The game verifies: $g^{s'} = g^{s + c} \stackrel{?}{=} R \cdot \algo{TwPK}(X, (1, 1))^c$.
  \end{itemize}
  
  The verification succeeds because:
  \begin{align*}
    g^{s + c} &= g^s \cdot g^c \\
    &= R \cdot X^c \cdot g^c \\
    &= R \cdot (X \cdot g)^c \\
    &= R \cdot \algo{TwPK}(X, (1, 1))^c
  \end{align*}
  
  Since $(m, (0, 1)) \neq (m, (1, 1))$, this is a valid forgery that breaks EUF-CMA-TK security.
\end{mysolution}
\fi

\begin{exercise}
  Explain why we cannot directly apply the EUF-CMA security proof of $\algo{SchnorrSig}$ to prove EUF-CMA-TK security for $\algo{SchnorrSig}$ with SchnorrTS tweaking.
  
  Hint: Examine what Lemma~\ref{lem:schnorr-to-single} guarantees about $\cdv$'s output and why this guarantee holds. Then consider whether a similar guarantee could be established for a reduction $\cdv'$ that uses an adversary $\adv$ breaking EUF-CMA-TK for $\algo{SchnorrSig}$ with SchnorrTS.
\end{exercise}

\ifsolutions
\begin{mysolution}
  The EUF-CMA security proof for $\algo{SchnorrSig}$ (Lemma~\ref{lem:schnorr-to-single}) requires that when the reduction algorithm $\cdv$ outputs $(z, \mathit{aux})$ with $z = (R^*, m^*)$ and $\mathit{aux} = s^*$, we have:
  \[
    g^{s^*} = R^* \cdot X^{\algo{H}(R^*, m^*)}
  \]
  
  The proof of this property relies on showing that the programmed oracle value $T[R^*, m^*]$ equals the external oracle $\algo{H}(R^*, m^*)$.
  This is proven by contradiction: if they differ, then $T[R^*, m^*]$ must have been set during a signing query, which means $m^* \in \mathcal{Q}$.
  But the reduction asserts $m^* \notin \mathcal{Q}$, giving us the contradiction.
  
  In the EUF-CMA-TK setting, this proof breaks down:
  \begin{itemize}
    \item The adversary outputs $(m^*, t^*, \sigma^*)$ for some tweak $t^*$.
    \item The winning condition checks $(m^*, t^*) \notin \mathcal{Q}$, not $m^* \notin \mathcal{Q}$.
    \item This means the adversary could have queried the signing oracle with $m^*$ with a different tweak $t' \neq t^*$.
    \item If $m^*$ was queried to the signing oracle with tweak $t'$, and the adversary's forgery reuses the same $R$ from that query, then $T[R^*, m^*]$ has been assigned to a uniformly random value, which is unequal to $\algo{H}(R^*, m^*)$ with overwhelming probability.
  \end{itemize}
  
  Since $\algo{H}_{\cdv}(R^*, m^*) \neq \algo{H}(R^*, m^*)$ (with overwhelming probability), we cannot use the forking lemma's guarantee that $\algo{H}(R^*, m^*) \neq \algo{H}'(R^*, m^*)$ to extract the secret key from the two signatures.
\end{mysolution}
\fi

\begin{exercise}\label{ex:schnorrkp-to-single}
  Prove the following lemma, which is an adaptation of Lemma~\ref{lem:schnorr-to-single} for the EUF-CMA-TK security of $\algo{SchnorrSigKP}$ with SchnorrTS.
  
  \begin{lemma}\label{lem:schnorrkp-to-single}
    Let $\adv$ be an adversary against the EUF-CMA-TK security of $\algo{SchnorrSigKP}$ with SchnorrTS$[\grgen, \algo{H}]$ in the random oracle model for $\algo{H}$ making at most $q_s$ queries to $\pcoracle{Sign}$ and at most $q_h$ queries to $\algo{H}$.
    
    Let $\algo{InpGen}$ be the algorithm which on input $1^\secpar$ runs $(\GG, p, g) \gets \grgen(1^\secpar)$, draws $X \sample \GG$, and returns $((\GG, p, g), X)$.
    
    There exists algorithm $\cdv$ that takes as input $(\gparam, X) \gets \algo{InpGen}(1^\secpar)$, has access to a random oracle $\algo{H}$, makes at most $q_h + 1$ queries to $\algo{H}$, and satisfies
    \[
      \left|\advantage{Single}{\cdv, \algo{InpGen}} - \advantage{EUF-CMA-TK}{\adv, \algo{SchnorrSigKP}, \algo{SchnorrTS}}\right| \leq \frac{q_s(q_s + q_h)}{2^\secpar}
    \]
    with $\advantage{Single}{\cdv, \algo{InpGen}}$ as defined in the forking lemma (Lemma~\ref{lem:fork}).
    
    Moreover, when $\cdv$ returns a non-$\bot$ output $(z, \mathit{aux})$, then $z = (R^*, \algo{TwPK}(X, t^*), m^*, t^*)$, $\mathit{aux} = s^*$, and
    \[
      g^{s^*} = R^* \cdot \algo{TwPK}(X, t^*)^{\algo{H}(R^*, \algo{TwPK}(X, t^*), m^*)}
    \]
  \end{lemma}
  
  Adapt the proof of Lemma~\ref{lem:schnorr-to-single} to prove this lemma.
  
  \textbf{Note:} Applying the forking lemma to $\cdv$ would allow extracting the discrete logarithm via $g^{s_1 - s_2} = (g^\alpha \cdot X^\beta)^{h_1 - h_2}$ where $(\alpha, \beta) = t^*$, thus establishing the EUF-CMA-TK security of $\algo{SchnorrSigKP}$ with SchnorrTS.
\end{exercise}

\ifsolutions
\begin{mysolution}
  We adapt the proof of Lemma~\ref{lem:schnorr-to-single} by modifying the games to handle tweaking and key prefixing.
  
  \begin{figure}[htbp]
    \begin{center}
      \begin{tcolorbox}[width=\textwidth]
        \begin{pchstack}[center]
          \begin{pcvstack}
            \procedure[headlinesep=1pt]{$\game{\algo{EUF-CMA-TK}}{\algo{SchnorrSigKP}, \algo{SchnorrTS}}$}{%
              \gparam \gets \grgen(1^\secpar) \\
              x \sample \ZZ_p \\
              X \defeq g^x \\
              \params \defeq (\gparam, \{0,1\}^*, \GG \times \ZZ_p) \\
              T \defeq \emptyset \\
              \mathcal{Q} \defeq \emptyset \\
              (m^*, t^*, \sigma^*) \gets \adv^{\pcoracle{Sign}, \pcoracle{H}}(\params, X) \\
              (R^*, s^*) \defeq \sigma^* \\
              \tilde{X}^* \defeq \algo{TwPK}(X, t^*) \\
              c^* \defeq \pcoracle{H}(R^*, \tilde{X}^*, m^*) \\
              \pcreturn (m^*, t^*) \notin \mathcal{Q} \wedge \\
              \t g^{s^*} = R^* \cdot \tilde{X}^{*c^*}
            }
            \pcvspace
            \procedure[headlinesep=1pt]{Oracle $\pcoracle{Sign}(m, t)$}{%
              \tilde{x} \defeq \algo{TwSK}(x, t) \\
              \tilde{X} \defeq \algo{TwPK}(X, t) \\
              r \sample \ZZ_p \\
              R \defeq g^r \\
              c \defeq \pcoracle{H}(R, \tilde{X}, m) \\
              s \defeq r + c \cdot \tilde{x} \bmod p \\
              \mathcal{Q} \defeq \mathcal{Q} \cup \{(m, t)\} \\
              \pcreturn (R, s)
            }
            \pcvspace
            \procedure[headlinesep=1pt]{Oracle $\pcoracle{H}(y)$}{%
              \pcif T[y] = \bot \pcthen \\
              \t T[y] \sample \ZZ_p \\
              \pcreturn T[y]
            }
          \end{pcvstack}
          \pchspace
          \begin{pcvstack}
            \procedure[headlinesep=1pt]{$\bdv(\gamechange{$\gparam, X$}; \rho)$}{%
              \gamechange{$\params \defeq (\gparam, \{0,1\}^*,$} \\
              \gamechange{$\t \GG \times \ZZ_p)$} \\
              T \defeq \emptyset \\
              \mathcal{Q} \defeq \emptyset \\
              (m^*, t^*, \sigma^*) \gets \\
              \t \adv^{\pcoracle{Sign}\gamechange{$_{\bdv}$}, \pcoracle{H}\gamechange{$_{\bdv}$}}(\params, X) \\
              (R^*, s^*) \defeq \sigma^* \\
              \tilde{X}^* \defeq \algo{TwPK}(X, t^*) \\
              c^* \defeq \pcoracle{H}_{\bdv}(R^*, \tilde{X}^*, m^*) \\
              \gamechange{$\pcassert (m^*, t^*) \notin \mathcal{Q}$} \\
              \gamechange{$\pcassert g^{s^*} = R^* \cdot \tilde{X}^{*c^*}$} \\
              \gamechange{$\pcreturn ((R^*, \tilde{X}^*, m^*, t^*), s^*)$}
            }
            \pcvspace
            \procedure[headlinesep=1pt]{Oracle $\pcoracle{Sign}_{\bdv}(m, t)$}{%
              \tilde{X} \defeq \algo{TwPK}(X, t) \\
              \gamechange{$s \sample \ZZ_p$} \\
              \gamechange{$c \sample \ZZ_p$} \\
              \gamechange{$R \defeq g^s \cdot \tilde{X}^{-c}$} \\
              \gamechange{$\pcassert T[R, \tilde{X}, m] = \bot$} \\
              \gamechange{$T[R, \tilde{X}, m] \defeq c$} \\
              \mathcal{Q} \defeq \mathcal{Q} \cup \{(m, t)\} \\
              \pcreturn (R, s)
            }
            \pcvspace
            \procedure[headlinesep=1pt]{Oracle $\pcoracle{H}_{\bdv}(y)$}{%
              \pcif T[y] = \bot \pcthen \\
              \t T[y] \sample \ZZ_p \\
              \pcreturn T[y]
            }
          \end{pcvstack}
          \pchspace
          \begin{pcvstack}
            \procedure[headlinesep=1pt]{$\cdv^{\gamechange{$\pcoracle{H}$}}(\gparam, X; \rho)$}{%
              \params \defeq (\gparam, \{0,1\}^*, \\
              \t \GG \times \ZZ_p) \\
              T \defeq \emptyset \\
              \mathcal{Q} \defeq \emptyset \\
              \pccomment{Track $(t, \tilde{X})$ pairs} \\
              \gamechange{$\mathcal{T} \defeq \emptyset$} \\
              (m^*, t^*, \sigma^*) \gets \\
              \t \adv^{\pcoracle{Sign}\gamechange{$_{\cdv}$}, \pcoracle{H}\gamechange{$_{\cdv}$}}(\params, X) \\
              (R^*, s^*) \defeq \sigma^* \\
              \tilde{X}^* \defeq \algo{TwPK}(X, t^*) \\
              c^* \defeq \pcoracle{H}_{\cdv}(R^*, \tilde{X}^*, m^*) \\
              \pcassert (m^*, t^*) \notin \mathcal{Q} \\
              \pcassert g^{s^*} = R^* \cdot \tilde{X}^{*c^*} \\
              \pccomment{Check for tweaking collision} \\
              \gamechange{$\pcif \exists (t, \tilde{X}) \in \mathcal{T}:$} \\
              \gamechange{$\t\t \tilde{X} = \tilde{X}^* \wedge t \neq t^* \pcthen$} \\
              \gamechange{$\t (\alpha, \beta) \defeq t; (\alpha^*, \beta^*) \defeq t^*$} \\
              \gamechange{$\t x \defeq (\alpha - \alpha^*) / (\beta^* - \beta) \bmod p$} \\
              \gamechange{$\t \pcreturn ((\bot, \bot, \bot, \bot), x)$} \\
              \pcreturn ((R^*, \tilde{X}^*, m^*, t^*), s^*)
            }
            \pcvspace
            \procedure[headlinesep=1pt]{Oracle $\pcoracle{Sign}_{\cdv}(m, t)$}{%
              \tilde{X} \defeq \algo{TwPK}(X, t) \\
              \gamechange{$\mathcal{T} \defeq \mathcal{T} \cup \{(t, \tilde{X})\}$} \\
              s \sample \ZZ_p \\
              c \sample \ZZ_p \\
              R \defeq g^s \cdot \tilde{X}^{-c} \\
              \pcassert T[R, \tilde{X}, m] = \bot \\
              T[R, \tilde{X}, m] \defeq c \\
              \mathcal{Q} \defeq \mathcal{Q} \cup \{(m, t)\} \\
              \pcreturn (R, s)
            }
            \pcvspace
            \procedure[headlinesep=1pt]{Oracle $\pcoracle{H}_{\cdv}(y)$}{%
              \pcif T[y] = \bot \pcthen \\
              \gamechange{$\t T[y] \gets \pcoracle{H}(y)$} \\
              \pcreturn T[y]
            }
          \end{pcvstack}
        \end{pchstack}
      \end{tcolorbox}
    \end{center}
    \caption{Security reduction for SchnorrSigKP with SchnorrTS: EUF-CMA-TK game (left), algorithm $\bdv$ (middle), and algorithm $\cdv$ (right). Changes from the previous game are marked with $\gamechange{$\cdot$}$.}
    \label{fig:schnorrkp-reduction}
  \end{figure}
  
  The games are shown in Figure~\ref{fig:schnorrkp-reduction}.
  
  \textbf{Proof that $g^{s^*} = R^* \cdot \algo{TwPK}(X, t^*)^{\algo{H}(R^*, \algo{TwPK}(X, t^*), m^*)}$:} 
  When $\cdv$ succeeds, we have $g^{s^*} = R^* \cdot \tilde{X}^{*c^*}$ where $\tilde{X}^* = \algo{TwPK}(X, t^*)$ and $c^* = \pcoracle{H}_{\cdv}(R^*, \tilde{X}^*, m^*)$.
  We claim that $\pcoracle{H}_{\cdv}(R^*, \tilde{X}^*, m^*) = \algo{H}(R^*, \tilde{X}^*, m^*)$.
  
  Assume for contradiction that $\pcoracle{H}_{\cdv}(R^*, \tilde{X}^*, m^*) \neq \algo{H}(R^*, \tilde{X}^*, m^*)$.
  This means $T[R^*, \tilde{X}^*, m^*]$ was assigned during a $\pcoracle{Sign}_{\cdv}$ query.
  But whenever $\pcoracle{Sign}_{\cdv}$ sets $T[R, \tilde{X}, m] = c$ for some $(m, t)$ with $\tilde{X} = \algo{TwPK}(X, t)$, it also adds $(m, t)$ to $\mathcal{Q}$.
  Since $T[R^*, \tilde{X}^*, m^*]$ was set, there must exist some signing query where the key for $T$ was $(R^*, \tilde{X}^*, m^*)$.
  This means $R = R^*$, $\tilde{X} = \tilde{X}^*$, and $m = m^*$ for that query.
  From $\tilde{X} = \tilde{X}^*$ and $\tilde{X} = \algo{TwPK}(X, t)$, $\tilde{X}^* = \algo{TwPK}(X, t^*)$, we have $\algo{TwPK}(X, t) = \algo{TwPK}(X, t^*)$.
  If $t \neq t^*$, then $\cdv$ would have detected this collision and directly extracted the discrete logarithm.
  Otherwise $t = t^*$, so $(m^*, t^*) = (m, t) \in \mathcal{Q}$.
  However, $\cdv$ asserts that $(m^*, t^*) \notin \mathcal{Q}$, which is a contradiction.
  Thus $c^* = \algo{H}(R^*, \tilde{X}^*, m^*)$.
  
  \textbf{Note on collision extraction:} If $(t, \tilde{X}) \in \mathcal{T}$ with $t = (\alpha, \beta) \neq t^* = (\alpha^*, \beta^*)$ but $\algo{TwPK}(X, t) = \algo{TwPK}(X, t^*)$, then $g^\alpha \cdot X^\beta = g^{\alpha^*} \cdot X^{\beta^*}$.
  This implies $g^\alpha \cdot g^{\beta x} = g^{\alpha^*} \cdot g^{\beta^* x}$, so $(\beta - \beta^*)x = \alpha^* - \alpha$.
  If $\beta = \beta^*$, then $\alpha = \alpha^*$, contradicting $t \neq t^*$.
  Thus $\beta \neq \beta^*$ and we can extract $x = (\alpha^* - \alpha)/(\beta - \beta^*) \bmod p$.
  
  The bound $|\advantage{Single}{\cdv, \algo{InpGen}} - \advantage{EUF-CMA-TK}{\adv, \algo{SchnorrSigKP}, \algo{SchnorrTS}}| \leq \frac{q_s(q_s + q_h)}{2^\secpar}$ follows from the same analysis as in Lemma~\ref{lem:schnorr-to-single}, where the error term accounts for the probability of collisions in the programmed oracle.
\end{mysolution}
\fi

\begin{exercise}[Optional]
  Read about signatures with re-randomizable keys in Section 3 of Fleischhacker et al.~\cite{PKC:FKMSSS16}, which presents a notion closely related to key tweaking.
  Compare their security model with the EUF-CMA-TK model discussed in this section and discuss the practical impact.
  How can they prove security for Schnorr signatures with re-randomized keys without requiring public key prefixing?
\end{exercise}

\section{Interactive Aggregate Signatures}\label{sec:interactive-aggregate-signatures}

An interactive aggregate signature (IAS) scheme allows $n$ signers, each with their own key pair $(sk_i, pk_i)$ and message $m_i$, to jointly produce a single short signature that proves $m_i$ was signed under $pk_i$ for every $i \in \{1, \ldots, n\}$.

The following exercises are based on the DahLIAS paper~\cite{add:nick2025dahlias}, which presents a discrete logarithm-based IAS scheme with constant-size signatures.

\subsection{Exercises}

\begin{exercise}
  Read the abstract and the main security theorem (Theorem 2) of DahLIAS.
  What does the security theorem tell us about the relationship between breaking DahLIAS and solving the underlying computational problems?
\end{exercise}

\begin{exercise}
  Read the technical introduction (Section 1.3).
  Why does the straightforward adaptation of the MuSig2 two-round technique described in the text not yield a secure IAS scheme?
\end{exercise}

\begin{exercise}
  Read about IAS syntax and security (Sections 3.1 and 3.3).
  What is the difference between EUF-CMA and co-EUF-CMA security for IAS?
\end{exercise}

\begin{exercise}
  What are the applications where strong binding (Section 6) matters?
  Explain why this property is important in these contexts.
\end{exercise}

\begin{exercise}[Optional]
  Read the proof sketch (Section 5.2).
  Outline the main proof technique and explain how the security reduction works.
\end{exercise}



%%% Appendix
\appendix
\crefalias{section}{appendix}

\section{Notation}\label{sec:notation}

\subsection{Sets and Numbers}

\begin{itemize}
  \item $\NN$ denotes the set of natural numbers $\{0, 1, 2, \ldots\}$.
  \item $\ZZ$ denotes the set of integers.
  \item $\ZZ_p$ denotes the set of integers modulo $p$, i.e., $\{0, 1, \ldots, p-1\}$.
  \item $\ZZ_p^*$ denotes the set of integers modulo $p$ excluding zero, i.e., $\{1, 2, \ldots, p-1\}$.
  \item $[n]$ denotes the set $\{1, 2, \ldots, n\}$ for a positive integer $n$.
  \item $[a, b]$ denotes the set $\{a, a+1, \ldots, b\}$ for integers $a \leq b$.
  \item $\{0, 1\}^n$ denotes the set of all binary strings of length $n$.
  \item $\{0, 1\}^*$ denotes the set of all binary strings of finite length.
\end{itemize}

\subsection{Probability and Sampling}

\begin{itemize}
  \item $x \sample S$ denotes sampling $x$ uniformly at random from the finite set $S$.
  \item $x \gets \algo{A}(y)$ denotes running algorithm $\algo{A}$ on input $y$ and assigning the output to $x$.
  \item $\Pr[E]$ denotes the probability of event $E$.
  \item $\defeq$ denotes a definition, i.e., the left side is defined to equal the right side.
\end{itemize}

\subsection{Cryptographic Notation}

\begin{itemize}
  \item $\secpar$ denotes the security parameter.
  \item $\secparam$ denotes the security parameter in unary representation.
  \item $\negl$ denotes a negligible function in the security parameter.
  \item \ppt stands for probabilistic polynomial-time.
\end{itemize}

\subsection{Groups and Fields}

\begin{itemize}
  \item $\gparam$ denotes group parameters, where $\GG$ denotes a cyclic group, $g$ denotes the generator and $p$ denotes the order of a group.
  \item $1_\GG$ denotes the identity element of group $\GG$.
\end{itemize}

\subsection{Logical Operators}

\begin{itemize}
  \item $\wedge$ denotes logical AND.
  \item $\vee$ denotes logical OR.
  \item $\neg$ denotes logical NOT.
  \item $\oplus$ denotes bitwise XOR.
\end{itemize}

\subsection{Other Notation}

\begin{itemize}
  \item $\|$ denotes concatenation (e.g., $x \| y$ is the concatenation of $x$ and $y$).
  \item $|S|$ denotes the cardinality (size) of set $S$.
  \item $\bot$ denotes an error or undefined value.
  \item $\emptyset$ denotes the empty set.
  \item $\log_g(h)$ denotes the discrete logarithm of $h$ with respect to base $g$.
  \item $O(\cdot)$ denotes asymptotic upper bound (big-O notation).
\end{itemize}


\section{Appendix: Proofs of Negligible Function Properties}\label{sec:appendix-negl-proofs}

This appendix contains the proofs of properties (2)-(4) from Lemma~\ref{lem:negl}.

\begin{proof}[Proof of Lemma~\ref{lem:negl}(2)]
  Let $f_1, f_2$ be negligible functions.
  We need to show that $f_1 + f_2$ is negligible.
  
  Let $c > 0$ be arbitrary. We must find $N$ such that $(f_1 + f_2)(\secpar) < \secpar^{-c}$ for all $\secpar > N$.
  
  Since $f_1$ and $f_2$ are negligible, for the constant $c + 1 > 0$, there exist $N_1, N_2 \in \NN$ such that:
  \[
  \forall \secpar > N_1: f_1(\secpar) < \secpar^{-(c+1)} \quad \text{and} \quad \forall \secpar > N_2: f_2(\secpar) < \secpar^{-(c+1)}
  \]
  
  Let $N = \max(N_1, N_2, 2)$. Then for all $\secpar > N$:
  \[
    f_1(\secpar) + f_2(\secpar) < \secpar^{-(c+1)} + \secpar^{-(c+1)} = 2 \cdot \secpar^{-(c+1)} = \frac{2}{\secpar} \cdot \secpar^{-c} \leq \secpar^{-c}
  \]
  where the last inequality holds because $\frac{2}{\secpar} \leq 1$ when $\secpar \geq 2$, which is guaranteed by our choice of $N$.
\end{proof}

\begin{proof}[Proof of Lemma~\ref{lem:negl}(3)]
  Let $f$ be a negligible function with $f(\secpar) \ge 0$ and let $k > 0$ be a constant.
  We need to show that $f + k$ is not negligible.
  
  Assume for contradiction that $f + k$ is negligible. 
  Then for the constant $c = 1$, there exists $N \in \NN$ such that for all $\secpar > N$:
  \[
    f(\secpar) + k < \secpar^{-1}
  \]
  
  Since $f(\secpar) \ge 0$, we have $f(\secpar) + k \ge k$ for all $\secpar$.
  
  Choose $\secpar_0 = \max(N + 1, \lceil \frac{2}{k} \rceil)$. Then $\secpar_0 > N$ and $\secpar_0^{-1} \le \frac{k}{2}$.
  
  By our assumption, we must have:
  \[
    k \le f(\secpar_0) + k < \secpar_0^{-1} \le \frac{k}{2}
  \]
  
  This gives us $k < \frac{k}{2}$, which is a contradiction.
  
  Therefore, $f + k$ is not negligible.
\end{proof}

\begin{proof}[Proof of Lemma~\ref{lem:negl}(4)]
  Let $f$ be a non-negligible function and $g$ be a negligible function.
  Assume for contradiction that $f - g$ is negligible.
  Then by property (2), $(f - g) + g = f$ would be negligible (as the sum of two negligible functions).
  This contradicts the assumption that $f$ is non-negligible.
\end{proof}

\section{Appendix: Notation}\label{sec:appendix-notation}

%%% Acknowledgements
\section*{Acknowledgements}
We thank Nadav Kohen and Pieter Wuille for valuable feedback and discussions.
We also thank Claude for assistance with revisions and exercises.

%%% Changelog
\iffull
  \phantomsection
  \addcontentsline{toc}{section}{Changelog}

  \section*{Changelog}\label{sec:changelog}
  \begin{description}
    \item[2025-08-25]\
          \begin{itemize}
            \item First draft.
          \end{itemize}
  \end{description}
  % TODO Why is spacing wrong without this?
  \vspace{-2em}
\fi

%%% Bibliography
\phantomsection
\addcontentsline{toc}{section}{References}
\printbibliography\label{sec:bib}

\end{document}
