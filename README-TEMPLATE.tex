\setcounter{section}{-1}
\section{Template Usage Notes}
This is a brief overview.
See the respective package documentation for more details. (Hint: Try \texttt{texdoc -l <pkg>} if you use TeXLive or check out \url{https://texdoc.org/}.)
The source code of this file (\texttt{template-readme.tex}) may be instructive, too.


\subsection{Compilation}
It's recommended to use \texttt{latexmk}.
A configuration file is included, so simply running \texttt{latexmk} is enough.
Use \texttt{latexmk -C} to clean auxiliary files, which are all stored in the \texttt{latex.out} directory.


\subsection{Toggles}
A few toggles are provided at the top of \texttt{main.tex}:
\begin{quote}
  \begin{description}
    \item [\texttt{\textbackslash{}fulltrue}:]
          Full version of the paper. (Authors can also use \verb|\iffull| where appropriate.)
    \item [\texttt{\textbackslash{}notestrue}:]
          Enable notes and todos
    \item [\texttt{\textbackslash{}labelstrue}:]
          Print labels
    \item [\texttt{\textbackslash{}anonymoustrue}:]
          Hide authors
    \item [\texttt{\textbackslash{}spacetrue}:]
          Enable hacks to save space
    \item [\texttt{\textbackslash{}camerareadytrue}:]
          Disable some tweaks that the publisher may not like
  \end{description}
\end{quote}


\subsection{Theorem-live Environments}
The \texttt{llncs} class defines the environments
\texttt{corollary}, \texttt{definition}, \texttt{lemma}, \texttt{proposition}, and \texttt{theorem},
as well as \texttt{proof} and \texttt{claim} with a different formatting.
This template adds \texttt{assumption}.

\begin{theorem}[My Theorem]\label{thm:my-thm}
  \begin{equation}\label{eq:my-thm}
    C = g^mh^r.
  \end{equation}
  \begin{equation}\label{eq:my-thm2}
    \Pr[\bad] \le \negl.
  \end{equation}
\end{theorem}
\begin{proof}[Proof (handwavy)]
  Believe, me, this \cref{eq:my-thm}, it's true.
  But here's an equation to demonstrate that \verb|\qedhere| works:
  \[ c = m+rx . \qedhere \]
\end{proof}

\begin{lemma}\label{thm:a-lemma}
  $10\,\btcsym$ is a lot of money.
\end{lemma}
\begin{proof}
  Follows from the theorem.
\end{proof}

\begin{assumption}[VERY-HARD-PROB]\label{ass:very-hard}
  It's hard, you know?
\end{assumption}

\begin{remark}\label{rem:add-env}
  There are additional environments
  \texttt{case}, \texttt{conjecture}, \texttt{example}, \texttt{exercise}, \texttt{note}, \texttt{problem}, \texttt{property}, \texttt{question}, \texttt{remark}, and \texttt{solution}.
  These have a different formatting (like this \lcnamecref{rem:add-env}).
\end{remark}


\subsection{Citations and Bibliography (\texttt{biblatex})}
Use \texttt{get-cryptobib.sh} to add the latest \href{https://cryptobib.di.ens.fr/}{crypto.bib} as a git subtree.
The same script is used to update \texttt{crypto.bib}.

This template defaults to BibLaTeX's modern Biber backend, which can handle UTF-8 and which will enable all BibLaTeX features.
But since Biber tends to be slow and will need over a minute to parse the huge \texttt{crypto.bib},\footnote{See \url{https://github.com/plk/biber/issues/371} for background.}
we use a wrapper around Biber that automatically extracts only the cited entries from \texttt{crypto.bib} to a temporary BibTeX file that will then be processed by Biber.
All of this should happen automatically and transparently if you use \texttt{latexmk}.

You can use \verb|\textcite| to typeset author names automatically, \eg, \textcite{CCS:BelNev06,EPRINT:NicRufSeu20}.

It's a good idea to define the \texttt{shorthand} field in the bib file for BIPs, RFCs and similar documents, \eg, \verb|shorthand = {BIP340}|. By the way, have you heard about xonly keys~\cite{add:bip-schnorr}?


\subsection{References (\texttt{cleveref})}
\begin{enumerate}
  \item\label{item:one}
        Use \verb|\cref{label,maybe-another-label}| to insert a reference, \eg, \cref{sec:intro,thm:my-thm,ass:very-hard}.
        The prefix, \eg, ``Theorem'' or ``Appendix'' will be added automatically.
  \item
        References to equations, list items, and line numbers (\cref{line:false,item:one,eq:my-thm}) are not capitalized.
        Use \verb|\Cref| to force uppercase, \eg, at the beginning of a sentence.
        Then you get:
        \Cref{line:false,item:one}. But: \Cref{item:one}. And: \Cref{eq:my-thm,eq:my-thm2,item:one}.
\end{enumerate}


\subsection{Notes and Todos (\texttt{todonotes})}
Set \verb|\notestrue| to display them.
\newuser{readme}{Template:}{orange!20}
\readmenote{
  You can define a user with \texttt{\textbackslash{}newuser}, \eg,  \texttt{\textbackslash{}newuser\textbraceleft{}s\textbraceright{}\textbraceleft{}Satoshi\textbraceright{}\textbraceleft{}orange\textbraceright{}}, see the source code.
  This will create the user-specific command \texttt{\textbackslash{}snote} for inserting notes.

  A note can span multiple paragraphs.
}


\subsection{Macros, \eg, $\btcsym$, $\algo{Sign}$, $\pk$, $\defeq$, and $\xmark$ Are Mostly Robust and Can Thus be Used in Movable Arguments like Section Headings}
If not, try to put a \verb|\protect| in front of them when using them.


\subsection{Pseudocode (\texttt{cryptocode})}
See the source code.
\begin{figure}[tbhp]
  \begin{center}
    % If you omit the width, then it will be 100%.
    % Try to specify boxsep=1mm or smaller if space is tight
    \begin{tcolorbox}[width=10cm]
      \begin{pchstack}[center]
        \begin{pcvstack}
          \procedure[linenumbering]{$\algo{Alg}(\secpar)$}{%
            y \gets \ZZ_p \\
            \pcreturn \pcfalse \label{line:false}
          }
          \pcvspace
          \procedure[headlinesep=1pt]{$\game{\Game~\algo{Coll}}{\algo{H}}$}{%
            \mathellipsis
          }
        \end{pcvstack}
        \pchspace[1em]
        \begin{pcvstack}
          \procedure{$\algo{Foo}(\secpar)$}{%
            x \gets 5\\
            \gamechange{\pcassert \pcfalse}
          }
          \pcvspace
          \procedure[headlinesep=1pt]{$\oracle{Bar}(\sk, \pk)$}{%
            \pcbox{x \sample \ZZ_p \pcsc \pcreturn \sk + x} \\
            y \sample \ZZ_p
          }
        \end{pcvstack}
      \end{pchstack}
    \end{tcolorbox}
  \end{center}
  \caption{My great scheme\label{fig:scheme}}  % \label goes inside \caption
\end{figure}

\clearpage
